\documentclass{memoir}
\usepackage[T1]{fontenc}
\usepackage[utf8x]{inputenc}
\usepackage{amsmath}
\usepackage[nodayofweek]{datetime} % For proper ISO 8601 date and time formatting
\usepackage{siunitx,pgfplots}
\sisetup{%
  locale=UK,
  zero-decimal-to-integer % do not show .0 on whole numbers
}

\pgfplotsset{%
  compat=newest,
    %grid=major,
    %grid style={dashed,gray!30},
    %scale only axis,
  table/col sep=comma,
  width=\linewidth,
  mark end/.style={%
    % From: https://tex.stackexchange.com/questions/116690/pgfplots-marks-mandatory-for-1st-and-last-point
    scatter,
    scatter src=x,
    scatter/@pre marker code/.code={%
      \pgfmathtruncatemacro\usemark{%
        (\coordindex==(1))
      }
      \ifnum\usemark=0
      \pgfplotsset{mark=none}
      \fi
    },
    scatter/@post marker code/.code={}
  },
  x-tick-siunitx/.style={%
    xticklabel = \pgfmathparse{\tick*1}\num{\pgfmathresult},
  },
  y-tick-siunitx/.style={%
    yticklabel = \pgfmathparse{\tick*1}\num{\pgfmathresult},
  },
  y-div/.style={%
    x-tick-siunitx,
    y-tick-siunitx,
    ylabel = Diversity,
  },
  initial-mut/.style={%
    y-div,
    xlabel = {Initial Mutation},
  },
  initial-sim/.style={%
    y-div,
    xlabel = {Initial Similarity},
  },
  initial-mut-sim-root/.style={%
    legend columns=-1,
    legend entries={Fitness-based, Hammming distance, \dia{}},
  },
  initial-sim-root/.style={%
    initial-mut-sim-root,
  },
  initial-mut-root/.style={%
    initial-mut-sim-root,
  },
  dynamic-root/.style={%
    legend columns=-1,
    legend entries={Greedy, Ancestor Elitism, Single Parent Elitism, MEEE},
  },
  dynamic/.style={%
    mark end,
    y-div,
    xlabel = Generation,
    each nth point=5,
  },
  fitness/.style={%
    mark end,
    x-tick-siunitx,
    y-tick-siunitx,
    ylabel = Max fitness,
    xlabel = Generation,
    title  = Fitness,
    each nth point=5,
  },
}


\title{Errata\\
\Large Trait-based Diversity Measurement in Genetic Algorithms using Artificial Neural Networks}

\author{Elias~Khazen~Obeid \and
        Kent~Munthe~Caspersen \and
      Martin~Bjeldbak~Madsen\\
    \scshape Aalborg University, Denmark}

\date{Updated \formatdate{20}{6}{2014}}

\begin{document}
\pagestyle{empty}
\maketitle
\thispagestyle{empty}
\newcommand{\perc}[1]{\SI{#1}{\percent}}
\newcommand{\di}{Neural Network Trait Diversity}
\newcommand{\dia}{NNTD}
\newcommand{\fit}{\phi}

\newcommand{\mail}[1]{\href{mailto:#1}{#1}}
\newcommand{\var}[1]{\ensuremath{\mathit{#1}}}
\newcommand{\citpls}[1]{\footnote{\textcolor{red}{ Citation needed here.}}}
\newcommand{\set}[1]{\ensuremath{\left\{ #1 \right\}}}
\newcommand{\setof}[2]{\ensuremath{\left\{ #1 \mid #2 \right\}}}
\newcommand{\bigO}[1]{\ensuremath{\operatorname{O}\left(#1\right)}}

\newcommand{\correctlyresize}[2]{\resizebox{#1}{!}{#2}}
\newcommand{\inputresize}[1]{\correctlyresize{\linewidth}{\input{#1}}}
\newcommand{\inputresizeto}[2]{\correctlyresize{#1}{\input{#2}}}

\definecolor{maroon}{HTML}{85144B}
\definecolor{navy}{HTML}{001F3F}
\definecolor{blue}{HTML}{0074D9}
\definecolor{aqua}{HTML}{7FDBFF}
\definecolor{teal}{HTML}{39CCCC} 
\definecolor{red}{HTML}{FF4136}
\definecolor{blue}{HTML}{0074D9}
\definecolor{black}{HTML}{111111}
\definecolor{purple}{HTML}{B10DC9}
\definecolor{green}{HTML}{2ECC40}

\sisetup{
  locale=UK,
  zero-decimal-to-integer % do not show .0 on whole numbers
}

\pgfplotsset{%
  compat=newest,
    %grid=major,
    %grid style={dashed,gray!30},
    %scale only axis,
  table/col sep=comma,
  width=\linewidth,
  mark end/.style={%
    % From: https://tex.stackexchange.com/questions/116690/pgfplots-marks-mandatory-for-1st-and-last-point
    scatter,
    scatter src=x,
    scatter/@pre marker code/.code={%
      \pgfmathtruncatemacro\usemark{%
        (\coordindex==(1))
      }
      \ifnum\usemark=0
      \pgfplotsset{mark=none}
      \fi
    },
    scatter/@post marker code/.code={}
  },
  x-tick-siunitx/.style={%
    xticklabel = \pgfmathparse{\tick*1}\num{\pgfmathresult},
  },
  y-tick-siunitx/.style={%
    yticklabel = \pgfmathparse{\tick*1}\num{\pgfmathresult},
  },
  y-div/.style={%
    x-tick-siunitx,
    y-tick-siunitx,
    ylabel = Diversity,
  },
  initial-mut/.style={%
    y-div,
    xlabel = {Initial Mutation},
  },
  initial-sim/.style={%
    y-div,
    xlabel = {Initial Similarity},
  },
  initial-mut-sim-root/.style={%
    legend columns=-1,
    legend entries={Fitness-based, Hammming distance, \dia{}},
  },
  initial-sim-root/.style={%
    initial-mut-sim-root,
  },
  initial-mut-root/.style={%
    initial-mut-sim-root,
  },
  dynamic-root/.style={%
    legend columns=-1,
    legend entries={Greedy, Ancestor Elitism, Single Parent Elitism, Explore-Exploit},
  },
  dynamic/.style={%
    mark end,
    y-div,
    xlabel = Generation,
    each nth point=5,
  },
  fitness/.style={%
    mark end,
    x-tick-siunitx,
    y-tick-siunitx,
    ylabel = Fitness,
    xlabel = Generation,
    title  = Fitness,
    each nth point=5,
  },
}


The fixed versions of the errors we have found are in the slides.

\section*{Distribution of individuals into species}
The formula for distributing neural networks into species (equation (3) on page 5) allows an individual to belong to multiple species.
As an example, consider a neural network $\ind$ with two output neurons $\nnout_1$ and $\nnout_2$, where $\nnout_1 = \nnout_2$ on some input $\ran$.
The binary representation of $1$ is \texttt{01}, which means that $\ind \in \speciesi{1}{\ran}$ if for any $j$, $\nnout_1 \geq \nnout_j$. This is indeed satisfied since all output neurons have the same value.
The binary representation of $3$ is \texttt{11}, which means that $\ind \in \speciesi{3}{\ran}$ if for any $j$, $\nnout_1 \geq \nnout_j \wedge \nnout_2 \geq \nnout_j$. This is indeed satisfied since all output neurons have the same value. We propose the following definition that corrects the above problem:

If $b_{m}b_{m-1}\dots b_1$ is the binary representation of a number $i$, we define the species \speciesi{i}{\ran} to contain any individual $\ind \in \indset$, that given $\ran$ as input satisfies
\begin{equation*}
  \forall j \in \left\{1, 2,\dots,m\right\}\left(b_j \rightarrow \left(\nnout_j = h\right) \land \neg b_j \rightarrow \left(\nnout_j < h\right)\right),
\end{equation*}
where $\nnout_k \in \nnoutset$ is the value of the $k$th output neuron of neural network \ind, and $h = \max\left\{\nnout_1, \nnout_2, \dots, \nnout_m\right\}$.

\section*{Initial similarity for 8-bit XOR}
The fitness-based plot for 8-bit XOR in figure 3 of page 9 is accidentally the same as the fitness-based plot for Snake. 
%This happened due to a typo in the way the plots are made. - I don't think we have to excuse why this happened
The 8-bit XOR plot should instead be as shown in figure~\ref{fig:initial-similarity-xor}.

\begin{figure}[htbp]
  \centering
  \resizebox{0.5\linewidth}{!}{%
  \begin{tikzpicture}
  \begin{axis}[
      initial-sim,
      title  = 8-bit XOR,
    ]
    \addplot[mark=*, color=blue]
    table[y index=1, x index=0] {initial_similarity_xor.csv};
    \addplot[mark=square*, color=red] % Hamming
    table[y index=2, x index=0] {initial_similarity_xor.csv};
    \addplot[mark=triangle*, color=green]
    table[y index=3, x index=0] {initial_similarity_xor.csv};
  \end{axis}
\end{tikzpicture}

}
  \caption{The fixed plot of initial similarity belonging to the 8-bit XOR problem}\label{fig:initial-similarity-xor}
\end{figure}

\section*{Fitness plots}
NNTD was used as the diversity measure for the Mass Extinction Explore Exploit replacement rule (MEEE) in the all the graphs of figures 5, 6, and 7.
\end{document}
