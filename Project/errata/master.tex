\documentclass{memoir}
\usepackage[T1]{fontenc}
\usepackage[utf8x]{inputenc}
\usepackage{amsmath}
\usepackage[nodayofweek]{datetime} % For proper ISO 8601 date and time formatting
\usepackage{siunitx,pgfplots}
\sisetup{%
  locale=UK,
  zero-decimal-to-integer % do not show .0 on whole numbers
}

\pgfplotsset{%
  compat=newest,
    %grid=major,
    %grid style={dashed,gray!30},
    %scale only axis,
  table/col sep=comma,
  width=\linewidth,
  mark end/.style={%
    % From: https://tex.stackexchange.com/questions/116690/pgfplots-marks-mandatory-for-1st-and-last-point
    scatter,
    scatter src=x,
    scatter/@pre marker code/.code={%
      \pgfmathtruncatemacro\usemark{%
        (\coordindex==(1))
      }
      \ifnum\usemark=0
      \pgfplotsset{mark=none}
      \fi
    },
    scatter/@post marker code/.code={}
  },
  x-tick-siunitx/.style={%
    xticklabel = \pgfmathparse{\tick*1}\num{\pgfmathresult},
  },
  y-tick-siunitx/.style={%
    yticklabel = \pgfmathparse{\tick*1}\num{\pgfmathresult},
  },
  y-div/.style={%
    x-tick-siunitx,
    y-tick-siunitx,
    ylabel = Diversity,
  },
  initial-mut/.style={%
    y-div,
    xlabel = {Initial Mutation},
  },
  initial-sim/.style={%
    y-div,
    xlabel = {Initial Similarity},
  },
  initial-mut-sim-root/.style={%
    legend columns=-1,
    legend entries={Fitness-based, Hammming distance, \dia{}},
  },
  initial-sim-root/.style={%
    initial-mut-sim-root,
  },
  initial-mut-root/.style={%
    initial-mut-sim-root,
  },
  dynamic-root/.style={%
    legend columns=-1,
    legend entries={Greedy, Ancestor Elitism, Single Parent Elitism, MEEE},
  },
  dynamic/.style={%
    mark end,
    y-div,
    xlabel = Generation,
    each nth point=5,
  },
  fitness/.style={%
    mark end,
    x-tick-siunitx,
    y-tick-siunitx,
    ylabel = Max fitness,
    xlabel = Generation,
    title  = Fitness,
    each nth point=5,
  },
}


\title{Errata\\
\Large Trait-based Diversity Measurement in Genetic Algorithms using Artificial Neural Networks}

\author{Elias~Khazen~Obeid \and
        Kent~Munthe~Caspersen \and
      Martin~Bjeldbak~Madsen\\
    \scshape Aalborg University, Denmark}

\date{Updated \formatdate{20}{6}{2014}}

\begin{document}
\pagestyle{empty}
\maketitle
\thispagestyle{empty}

The fixed versions of the errors we have found are in the slides.

\section*{Bucket definition function}
The bucket definition function in equation (3) on page 5 allows an individual to belong to multiple species. This is possible if the bit string of its species can be broken down into another species, e.g., every individual in species \texttt{11} will also belong to species \texttt{01} and \texttt{10}. The function does not take the other bits of the species into account. We propose the following similar function definition instead
\begin{equation*}
  \forall j \in \left\{m, m-1,\dots,1\right\}\left(b_j \implies \left(o_j = h\right) \land \neg b_j \implies\left(o_j < h\right)\right),
\end{equation*}
where $h = \max\left\{o_1, o_2, \dots, o_b\right\}$, $m$ is the amount of bits in any species' bit string (constant for all species), $b_j$ is the resulting species' bit string with index $j$, and $o_j$ is the value of output neuron $j$ in the given network.

\section*{Initial similarity for 8-bit XOR}
The fitness-based line of the last plot in figure 3 of page 9 is accidentally the same as the fitness-based line of the snake data set. This happened due to a typo in the way the plots are made. The 8-bit XOR plot should instead be as shown in figure~\ref{fig:initial-similarity-xor}.

\begin{figure}[htbp]
  \centering
  \resizebox{0.5\linewidth}{!}{%
  \begin{tikzpicture}
  \begin{axis}[
      initial-sim,
      title  = 8-bit XOR,
    ]
    \addplot[mark=*, color=blue]
    table[y index=1, x index=0] {initial_similarity_xor.csv};
    \addplot[mark=square*, color=red] % Hamming
    table[y index=2, x index=0] {initial_similarity_xor.csv};
    \addplot[mark=triangle*, color=green]
    table[y index=3, x index=0] {initial_similarity_xor.csv};
  \end{axis}
\end{tikzpicture}

}
  \caption{The fixed plot of initial similarity belonging to the 8-bit XOR problem}\label{fig:initial-similarity-xor}
\end{figure}
\end{document}
