\documentclass[a4paper]{IEEEconf}
\synctex=1

\usepackage[utf8]{inputenc}
\usepackage[T1]{fontenc}
\usepackage[english]{babel}
\usepackage[cmex10]{amsmath}
\interdisplaylinepenalty=2500
\usepackage[nocompress]{cite}
\usepackage{graphicx}
\usepackage{algorithm}
\usepackage[noend]{algpseudocode}
\algrenewcommand\algorithmicindent{0.8em}%
\usepackage[colorinlistoftodos]{todonotes}
\presetkeys{todonotes}{inline}{}
\usepackage{caption}
\usepackage{subcaption}
\usepackage{booktabs}
\usepackage{hyperref}
\usepackage{tikz}
\usetikzlibrary{positioning}
\usepackage{pgfplots}
\usepackage{cleveref}
\usepackage{mathtools}
\usepackage{dblfloatfix}
\usepackage{fixltx2e}
\usepackage{nag}
\usepackage[activate={true,nocompatibility},final,tracking=true,kerning=true,spacing=true,factor=1100,stretch=10,shrink=10]{microtype}
% activate={true,nocompatibility} - activate protrusion and expansion
% final - enable microtype; use ``draft'' to disable
% tracking=true, kerning=true, spacing=true - activate these techniques
% factor=1100 - add 10% to the protrusion amount (default is 1000)
% stretch=10, shrink=10 - reduce stretchability/shrinkability (default is 20/20)
%
\definecolor{maroon}{HTML}{85144B}
\definecolor{navy}{HTML}{001F3F}
\definecolor{blue}{HTML}{0074D9}
\definecolor{aqua}{HTML}{7FDBFF}
\definecolor{teal}{HTML}{39CCCC}

\usepackage[locale=US]{siunitx}
\newcommand{\perc}[1]{\SI{#1}{\percent}}
\newcommand{\di}{Simpsons Diversity Index NN}
\newcommand{\dia}{SDINN}
\newcommand{\fit}{\phi}
%\sisetup{group-separator = {,},
%         per-mode = symbol,
%         inter-unit-product = \ensuremath{\cdot},
%         number-unit-product = \text{ },
%         binary-units = true}


\newcommand{\mail}[1]{\href{mailto:#1}{#1}}
\newcommand{\var}[1]{\ensuremath{\mathit{#1}}}
\newcommand{\citpls}[1]{\footnote{\textcolor{red}{ Citation needed here.}}}
\newcommand{\set}[1]{\ensuremath{\left\{ #1 \right\}}}
\newcommand{\setof}[2]{\ensuremath{\left\{ #1 \mid #2 \right\}}}

\newcommand{\correctlyresize}[2]{\resizebox{#1}{!}{#2}}
\newcommand{\inputresize}[1]{\correctlyresize{\linewidth}{\input{#1}}}
\newcommand{\inputresizeto}[2]{\correctlyresize{#1}{\input{#2}}}

\pgfplotsset{%
    compat=newest,
    grid=major,
    grid style={dashed,gray!30},
    scale only axis,
    table/col sep=comma,
    initial-mut/.style={%
      ylabel = {Diversity},
      xlabel = {Initial Mutation},
    },
    initial-sim/.style={%
      ylabel = {Diversity},
      xlabel = {Initial Similarity},
    },
    initial-mut-sim-root/.style={%
      legend columns=-1,
      legend entries={Fitness-based, Hammming distance, \dia{}},
    },
    initial-sim-root/.style={%
      initial-mut-sim-root,
      initial-sim,
    },
    initial-mut-root/.style={%
      initial-mut-sim-root,
      initial-mut,
    },
  }

\begin{document}

\title{Measuring Diversity in Steady State Genetic Algorithms using Artificial Neural Networks}

%\author{%
%  Kent Munthe Caspersen \\
%  \email{kcasp11@student.aau.dk}
%  \and
%  Elias Khazen Obeid \\
%  \email{eobeid11@student.aau.dk}
%  \and
%  Martin Bjeldbak Madsen \\
%  \email{mbma11@student.aau.dk} \\
%  \begin{affiliation}
%    Department of Computer Science\\
%    Aalborg University\\
%    9220 Aalborg
%  \end{affiliation}
%}

\maketitle

\listoftodos

\begin{abstract}
Controlling diversity in a genetic algorithm's population is crucial for finding the global optimum. When diversity is overlooked, premature convergence is the consequence, which possibly only leads to local optima. We propose a behaviour-based diversity measurement for genetic algorithms with artificial neural networks, which we call \di{} (\dia{}).

Experiments are conducted to compare \dia{} to the genotypic diversity measure of Hamming distance, and a phenotypic fitness-based diversity measure. We argue why both of these measures have weaknesses that \dia{} overcomes. Our experiments show that \dia{} consistently mirrors the intuition of behavioural diversity in populations, better than Hamming distance and fitness-based diversity measurement methods, which seem less predictable. Interestingly, the experiments also show that there might be a connection between the genetic and behavioural diversity among individuals.
\end{abstract}


\section{Introduction}
For many problems in computer science, a greedy strategy will not always lead to the best result. A greedy strategy chooses the next step by considering how promising each step is at that given moment. As an example, if a mountain climber wants to reach the Mount Everest, a greedy strategy would be to move in any direction he think is the best, as long as he climbs upwards and never downwards. Using this strategy, he is definitely able to climb up a peak, but once he reaches one, he might discover an even greater peak he could not see before. Unfortunately, his strategy is to never back down, so now he is stuck at what we call a local optimum. What he wanted was the global optimum of Mount Everest.
 
Genetic algorithms (GAs) can be used to search for a global optimum in many types of problems, such as optimization problems. A GA manages a number of individuals, which constitute a population. Each of the individuals represents a candidate solution to the same problem. Neural networks, among other data structures, can be chosen as individuals.
In this article, we will only use neural networks as individuals, and hence use the terms \emph{neural network} and\emph{individual} interchangeably.
A neural network takes a number of inputs, do some internal calculations, and then yield a number of outputs.
The neural network as its whole can be interpreted as a solution to the problem.
For instance, for the particular problem of adding two integers, the neural network can take two integers $s$ and $t$ as input, and output a single value, which ideally should equal $s + t$.

Most often, the individuals in a GA do not solve the problem in question adequately.
However, different individuals may have good solutions to different aspects of the problem.
Individuals maintained by the GA procreate to form offspring that combine the best traits from both its parents. Intuitively, the combination of these traits constitutes a better solution to the given problem. Each individual has a fitness value, which indicates how adequately it solves the problem in question.

To identify a constructive combination of traits, a diverse set of individuals is required. If diversity is not maintained, only a local optimum of the population's available traits will be explored, and a global optimum might never be found~\cite{ursem2002diversity,Darwen00doesextra}.

We believe it is essential that a diversity measure reflects the difference in traits among individuals. By concentrating on neural networks as individuals, we develop a new diversity measure that, as we will see, is not affected by pitfalls in traditional diversity measures.

Since a \emph{trait} is a rather vague term, we introduce a clear definition of traits among neural networks: ``\emph{two neural networks have different traits, if they for some input produce different outputs}''. We now argue why we think that neither fitness-based nor genotypic diversity measures catch a diversity among traits, or \emph{trait diversity}.

Consider two individuals who try to find the highest peak on an elevation map, where the fitness of each individual is based on the height of the peak they return. The two individuals might have completely different strategies, causing them to get stuck on two different peaks. If both peaks happen to have the same height, the two individuals will have the same fitness, and hence a fitness-based diversity measure will return a low diversity, even though the two individuals have different traits.

Two individuals of different genotypes can have the same behaviour, which means they yield the same output on any input. An example of such two neural networks is shown in \cref{fig:entire-eqnetwork}. No matter what input they receive, their output will always be the same. They are genotypically diverse, because their genetic structure is different, but they are not trait diverse, because they produce the same output. The genetic structure could be represented as bit string as shown in \cref{fig:entire-eqnetwork}.
%
\begin{figure*}
  \begin{subfigure}{0.5\textwidth}
    \centering
    \inputresizeto{0.5\linewidth}{drawings/eqnetworks/eqnetworks3}
    \caption{An artificial neural network with connections and weights.}\label{fig:eqnetwork}
  \end{subfigure}
  \begin{subfigure}{0.5\textwidth}
    \centering
    \inputresizeto{0.5\linewidth}{drawings/eqnetworks/eqnetworks4}
    \caption{An artificial neural network equivalent to \cref{fig:eqnetwork}.}\label{fig:eqnetwork2}
  \end{subfigure}
  \caption{Networks with same phenotype, but different genotypes. The binary representation assumes that each weight is represented by four bits.}\label{fig:entire-eqnetwork}
\end{figure*}

%
%Many existing methods which aim to increase diversity, require a lot of computational power \citpls{}. We will develop a method that is computationally inexpensive, yet maintains a high diversity in a population. Common methods for measuring the phenotypic diversity of a population only take into account the fitness of each individual \citpls{}. We develop a method for measuring phenotypic diversity that takes into account the different traits among individuals. 
We develop a new method for measuring phenotypic diversity in GAs using neural networks as individuals. We claim that our method better reflects different traits among the individuals. Furthermore, we use our method to explore how different replacement rules affect the diversity of a population.

This paper is organized as follows. \Cref{sec:preliminaries} introduces the concepts used in our diversity measure, and can be skipped if the reader knows about genetic algorithms and neural networks. \Cref{sec:nntd} describes our diversity measure, which is experimentally evaluated and compared to other measures in \cref{sec:experiments}. In \cref{sec:conclusion} we evaluate \dia.

\subsection{Related work}
Concepts and applications of GAs are described in \cite{Cobb93geneticalgorithms,DeJong:1975:ABC:907087,Luke2013Metaheuristics,Syswerda:1989:UCG:645512.657265,ursem2002diversity,fogarty,Whitley:1989:GAS:93126.93169,1250187}. The use of diversity maintenance in GAs is discussed in \cite{diaz2007empirical,Zitzler00comparisonof,Darwen00doesextra,1266373}, and measures of population diversity are described in \cite{Nguyen:2006:ASPGP,simpson1949measurement}. Combining GAs and neural networks is described in \cite{masterThesisGANN}.

\section{Preliminaries}
\label{sec:preliminaries}

Here we present the needed background knowledge to understand the work explained in the following sections. We begin with an introduction to artificial neural networks, followed by an introduction to genetic algorithms (GAs). 

In GAs, a high diversity among the individuals is important. It is often argued that the weakness of GAs is the fall in diversity over generations, which can result in premature convergence~\cite{diaz2007empirical, 1266373,Zitzler00comparisonof}. After presenting neural networks and GAs,  we present methods used to overcome a decreasing diversity. Methods like crowding and use of special selection procedures are presented. Next, we introduce different methods of measuring diversity.

%We begin by presenting methods which propose solutions to overcome the decreasing nature of diversity in the populations.

%In this section we present the structure of artificial neural networks and concept of genetic algorithms. Furthermore, we introduce the concept of diversity measurement for genetic algorithms.

\subsection{Artificial Neural Network}
An artificial neural network is a graph structure that, from the outside, can be seen as a black box, that given the values $x_1, x_2, \dots, x_n$ outputs the values $y_1, y_2, \dots, y_m$. With the right internal structure, a neural network can be used for a variety of purposes, e.g.\ face recognition, where the intensities of different pixels in an image are used as input, and a single output value $y_1$ is produced, where $y_1 = 1$ if the image was of a face and $0$ if not. We now describe the structure and inner workings of neural networks to understand how these output values are calculated based on input values.

\subsubsection{Neurons}
Nodes in the graph of a neural network are called neurons. Three classes of neurons exist: input, hidden, and output. Input neurons receive $x_1, x_2, \dots, x_n$ and output the same value. Hidden and output neurons take a number of values as input from edges exiting other neurons, applies a weight to each value, sums them, then applies a function to produce a single output value. The function applied is called the transfer function, and is the same for all hidden and output neurons in the network. This is recursively expressed by 
% Note: bias/threshold is not defined below, add if needed
\begin{equation*}\label{eq:weightcalc}
  y_i =
  \begin{cases}
    \var{x}_i                     & \text{if } i \text{ is an input neuron} \\
    \theta\left(\sum_{j=1}^n w_{ji} y_{j} \right) & \text{otherwise} %- \theta_i \right)
  \end{cases}
\end{equation*}
%
where $x_i$ is the input to input neuron $i$, $n$ is the amount of neurons, $y_i$ is the output value of neuron $i$, $w_{ji}$ is the weight of the edge from neuron $j$ to $i$ ($0$ if no connection exists), $y_j$ is the output of neuron $j$, and $\theta$ is the transfer function, usually defined to be the sigmoid function taking the form
%
\begin{equation*}
  \theta(t) = \frac{1}{1+e^{-t}}
\end{equation*}
%
In a feedforward network, neurons are placed in one or more layers in an acyclic graph structure, where each neuron in layer $i$ is connected to every neuron in layer $i + 1$. The value output by the last layer, or the output layer, becomes the output of the neural network. \Cref{fig:ann} illustrates the generic graph structure of a feedforward neural network with a single hidden layer.
%
\begin{figure}[htpb]
  \centering
  \includestandalone[mode=buildnew, width=\linewidth]{drawings/ANN/ANN}
  \caption{Structure of a neural network.}
  \label{fig:ann}
\end{figure}
%
\subsubsection{Training a Neural Network}
Any applica\-tion of a neural network requires a suitable number of layers, neurons, and weights between neurons, to adequately solve a given problem. How many hidden neurons and layers to have is a highly debated subject, see \cite{sarle1997}.

The weights on edges connecting neurons are decided by a process called training. Well known training algorithms, such as backpropagation, typically require that for each input to the neural network, the correct output is already known. This kind of learning is called supervised learning. For the application of face recognition, this means that the pictures used for training, each has a predicate indicating whether or not it is a picture of a face.

\todo{We need to fix the following two paragraphs. Change them to a shorter, more concise argument against backpropagation in general terms. Cite articles that argument against BP in GA.}
For some purposes, the desired output of a neural network is not known beforehand. Consider for instance a computer game, where a player is controlled by a neural network. The neural network takes values indicating the state of the game as input, such as the observed position of the players and enemies. For each input, the neural network returns a single value indicating which action the player should take, e.g.\ move left, move right, or jump. Given a state of the game, we might not be able to say whether an action output by the neural network is right or wrong. It might be wrong to move closer to an enemy if he eliminates you, but it might be right if the next action is successfully to eliminate the enemy.

From this, it is clear that a backpropagation algorithm is not appropriate for training a neural network to control the actions of an artificial intelligent player in a computer game. For this purpose we propose another approach. We can tell how good a neural network performs, or how fit it is, by simulating a game being played using the neural network to control the player and determining the score achieved in the game, or any other function indicating how well the artificial player performed according to some criteria using that particular neural network. By being able to measure the fitness of a particular neural network, we can use genetic algorithms to create and search for the best performing neural network.

\subsection{Genetic algorithms} 
Genetic algorithms are optimization algorithms, which imitate the process of natural selection in the search of a global optimum.

\subsubsection{Individuals and chromosomes}
GAs maintain a list of \emph{individuals}, which together form a \emph{population}. Each individual represents a possible solution to the optimization problem in question and has a fitness value, denoting how adequately the individual can solve the optimization problem. Individuals in GAs can take on any form of data structure, as long as they wholly represent a possible solution to the problem. An individual is encoded by its \emph{chromosome}, which is typically represented by a bit string. A substring of this bit string is called a \emph{gene}. The GA manipulates the bit strings to form new individuals. Therefore, using a GA requires a way of encoding/decoding between an individual's chromosome and a solution to the given problem. We refer to the individuals' chromosomes or bit strings synonymously.

The population used by a GA typically has a fixed number of individuals, each initialized with a random chromosome when the GA is run. That is, the bit string representing the chromosome is initialized with random bits. As the GA iterates, new individuals are made by combining and modifying chromosomes from existing individuals of the population. In steady state GAs, some of the new individuals will replace older individuals according to a replacement rule. In contrast, a generational GA will choose only from offspring when forming the next generation~\cite{fogarty, Syswerda:1989:UCG:645512.657265, Whitley:1989:GAS:93126.93169}. We will focus on steady state GAs, using the term \emph{generation} \generation{n} to denote the content of the population after $n - 1$ iterations. A replacement rule then determines which individuals of the $n$th generation \generation{n}, and their offspring \offspring{n} are selected to form the next generation \generation{n+1}.

\subsubsection{Neural networks as individuals}
%bitstrings
%weight on connections
%each input and each output
%amount of hidden neurons

As we have discussed, neural networks are ideal for decision making and hence are appropriate as individuals in a GA. To support various GA operators, individuals are encoded as bit strings.

% Uncommented this paragraph friday 2014-3-21, as it just says what a NN is? -Martin
%Each possible input an individual can get will be received through an input neuron. Likewise, each possible action an individual can perform is formulated through the output neurons. The network is constructed with connections between neurons with associated weights. These weights are used to calculate an action given the actual input.

When using a GA, all individuals are neural networks with the same architecture.
That is, the neural networks differ only in their weights between neurons and the bias of each neuron.
Each individual is therefore represented only in terms of the weights and biases.
For each GA, any weight and bias is encoded with a fixed number of bits $n$ and $m$, respectively.
The bit string is constructed in an ordered manner, such that the first $n$ bits represent the weight for the first connection between the first input neuron and the first hidden neuron, the next $n$ bits represent the weight for the connection between the first input neuron to the second hidden neuron, and so forth. \cref{fig:entire-eqnetwork} shows an example of two neural networks and the bit string that encodes each of them.
If biases are used by the GA, these comes after all the encoded weights.
They are ordered such that the first $m$ bits encodes the bias of the first neuron, the next $m$ bits encodes the bias of the second neuron, and so forth.
In this way, we encode a neural network, which is manipulated by different GA operators.
From now on, we will refer to the bit string encoding of a neural network as the individual's \emph{chromosome}.

% ALso removed this friday 2014-3-21, no need to reference it twice. Possibly if there was an example on how to encode the network in the example?
%An illustration of neural networks is presented in \cref{fig:ann}, and should give a good idea of how the bit string is concatenated.

%\[
%  \underbrace{w_{1,1}}_{n} w_{1,2} \ldots w_{i,j}
%\]

%\[
%  \ldots \underbrace{w_{i,j-1}}_{n} w_{i,j} w_{i,j+1} \ldots
%\]

%where $w_{i,j}$ represents the weight of the connection between the $i'th$ and $j'th$ neuron in bits. Each weight is concatenated with the next weight. The length of any weight is of size $n$.
%
If two chromosomes have different bit strings, we say that they have different genotypes. If the neural network they encode produces a different output for some input, we say they have different phenotypes.

%\subsubsection{Genes}
%Each individual consists of many \emph{genes} as part of its chromosome. Genes constitute the DNA of the individual. These genes are an encoding of some attribute or skill the individual has. Because we defined neural networks to be individuals, the weights and node biases consist of the DNA.

% This subsubsection is already described in individuals and hcromosomes
%\subsubsection{Populations}
%A genetic algorithm manages a collection of many individuals, known as a population. Individuals in the first population are usually initialized randomly with a fixed population size. The goal for these individuals is to solve a problem optimally. Each individual in the population has a fitness level that defines the individual's ability to solve a given problem.
%Initially, there are no generations. Creating a new population from a previous population increases the amount of generations by one, hopefully yielding a net increase in average fitness.

\subsubsection{Crossover and mutation}
In natural evolution, a pair of individuals come together to produce one or more children, having genes from both parents. In GAs, the process of procreation is done by performing a \emph{crossover} of the two parent individuals' chromosomes. Parts of each parent's bit strings are used to create the child individual.

\emph{Mutations} can also occur randomly at any point in time upon creating a child individual. If genes are encoded as bit strings, then a mutation arbitrarily toggles one or more bits. This ensures that any gene can be formed.

\subsubsection{Fitness functions}
A \emph{fitness function} must be defined to calculate the desirability of each individual. This function is used to define the most fit individuals in a population. By giving more fit individuals a greater chance of reproducing, the intuition is that more fit individuals will be created, having the best traits from each of their parents.


\subsubsection{Crowding}
Many methods have been proposed to overcome the problem of gradually decreasing diversity through generations. These methods include inserting random immigrants, which are new, randomly initialized individuals~\cite{Cobb93geneticalgorithms}, using complex population structures to lower the gene flow, and the use of special selection procedures~\cite{ursem2002diversity}. The latter is known as crowding.  How crowding is performed can be described formally by a replacement rule.

\subsubsection{Replacement rules}
A \emph{replacement rule} determines how a number of competing individuals are replaced by only a subset of themselves, e.g.\ when the individuals of the $i$th generation $G_i$, and their offspring $O_i$, created by some selection method, compete to be selected for the next generation $G_{i+1}$. A commonly used replacement rule is to make $G_i$ contain the $n$ most fit individuals from $G_{i-1} \cup O_{i-1}$, where $\mid G_x\mid = n$ for all $x$ (see~\cite{masterThesisGANN}). We will refer to this as the \emph{naive replacement rule}. It is easy to see that this method causes a low diversity over time. Since the best individuals are more likely to be selected for procreation, much of the offspring will have similar traits that yield a high fitness value, and thus be a part of the next generation, just like their parents. Naturally, this increases the chance of the same traits being carried on into the next generation.

Another well known replacement rule is the probabilistic crowding algorithm~\cite{Mengshoel_and_Goldberg:1999}
%
\[
  p(x) = \frac{f\left(x\right)}{f\left(x\right)+f\left(y\right)}
\]
%
where $p(x)$ denotes the chance that individual $x$ replaces individuals $\set{x, y}$, and $f(i)$ is the fitness of individual $i$. This rule can easily be generalized for larger family sizes. This replacement rule favours the most fit individuals to be selected.

A replacement rule can favour more diverse individuals, for instance by dictating that for any $a$ individuals having produced $b$ offspring individuals, we only select $a$ individuals of these similar $a + b$ individuals for the next generation, instead of selecting all $a + b$ individuals.

\todo{Måske skal vi fjerne forklaringen af replacement rules i dette afsnit, og lade ``Replacement rules'' beskrive dem i stedet for. Hvad synes I? --Martin}

%\paragraph{Roulette wheel selection}
%Also known as fitness proportionate selection, uses the fitness value of each individual to associate a probability of being selected to procreate. The probabilities are calculated to give the most fit individual the largest probability to be selected. We define $\fit_i$ to be the fitness of individual $i$, and the probability for selection is then calculated by $p_i = \frac{\fit_i}{\Phi}$, where $\Phi = \sum_{j=1}^{I} \fit_j$, and $I$ is the total number of individuals~\cite{tang1996genetic, koza1992genetic}. Using this method, the most fit individuals have the highest probability to be selected, while the least fit individuals have the lowest probability.
%
%Imagine a roulette wheel that is spun. The individual that is most fit might cover \perc{38} of the entire wheel, while the rest of the individuals combined have \perc{62} chance. It is obvious that the fittest individual will be selected most often.
%
%\paragraph{Stochastic Universal Sampling}
%
%This policy is similar to roulette wheel selection, with one exception. When individuals are selected for procreation, pointers are used to choose the individuals, instead of randomly choosing an individual. The number of pointers, $P$, is equal to the number of individuals for the next generation, and the pointers are equally spaced. The first pointer is placed at a random position in the range $\left[0, \frac{1}{P}\right]$, and the space between each of the following pointers is equal to $\frac{1}{P}$. Each pointer then points to an individual, and these individuals are selected for procreation, possibly skipping individuals~\cite{baker1987reducing}.
%
%\paragraph{Reward-based selection}
%
%Individuals have an associated reward, which is computed as the sum of the individual's reward and the reward of its parents. If the individual is selected for the next generation, then the individual and its parents receive a reward. The probability for an individual to be selected is proportional to the cumulative reward. There are different functions to calculate a reward~\cite{loshchilov2011not}.
%
%\paragraph{Tournament selection}
%
%As the name suggest, this policy works as any other tournament. It involves running several tournaments, and the winner of each tournament is chosen to procreate. The individuals compete to solve the given problem optimally, and the winner is selected for breeding~\cite{miller1996genetic}.
%

\subsection{Diversity measures}
\label{sec:diversitymeasures}
It is often argued that the weakness of GAs is the fall in diversity over generations, often resulting in premature convergence~\cite{diaz2007empirical, 1266373,Zitzler00comparisonof}.

Here, we summarise key points regarding genotypic and phenotypic diversity measures. %Afterwards, we present our own proposal for a diversity measure (\dia), and we clearly define what diversity means with this method.

\subsubsection{Genotypic diversity measures}
The genotypic diversity of a set of individuals is determined by how different their genetic structures are. To measure this type of diversity, methods to compute the distance between any two individuals' encoded bit strings are required.

The diversity between a set of bit strings can then be expressed as the average distance between any two bit strings. Summation can also be used instead of averaging, which is merely a convenient optimization.

The \emph{Hamming distance} between two bit strings $A$ and $B$ of equal length is the number of indexes $i$, such that $A[i] \neq B[i]$. The \emph{Levenshtein distance} between these two bit strings is the number of bits that must be inserted, deleted or substituted to change $A$ into $B$.

As shown in \cref{fig:entire-eqnetwork}, two networks of different genotypes may have the exact same behaviour. One can argue that the two genotypes shown in \cref{fig:entire-eqnetwork} are in fact not that different, since only two bits are different in each substring. The Levenshtein distance measure catches this intuition better than Hamming distance. For example, the distance between between the bit strings
%
\begin{align}
  &01010101\label{eq:bit1} \\
  &10101010\label{eq:bit2}
\end{align}
%
is 8 when using Hamming distance, and 2 when using Levenshtein distance, because transforming \cref{eq:bit1} into \cref{eq:bit2} is done by deleting the first bit and prepending a $0$, totalling 2 operations~\cite{1250187}.

If we define \indset{} to be a set of neural networks, the complexity of calculating $h_{ij}$, where $h_{ij}$ is the Hamming distance between two neural networks $\ind_i, \ind_j \in \indset$ is \bigO{\bitstringl}, which is linear in the length of an encoded network's bit string $l$. The complexity of computing Hamming distance for all $\ind \in \indset$ is thus \bigO{\indsetl^2 \cdot \bitstringl}. 

The complexity of calculating $v_{ij}$, where $v_{ij}$ is the Levenshtein distance between two neural networks $\ind_i, \ind_j \in \indset$ is \bigO{\bitstringl^2}\cite{Freeman:2006:CLN:1220835.1220895}. Therefore, computing the Levenshtein distance between all individuals in \indset{} yields the complexity \bigO{\indsetl^2 \cdot \bitstringl^2}.

\subsubsection{Phenotypic diversity measures}
A \emph{phenotypic diversity} measure is concerned with the individuals' behavioural differences, and can be calculated based on their fitness values. Such diversity measures include computing the standard deviation of fitness values, the average number of unique fitness values in a population, and entropy-based methods, see~\cite{1250187, 1266373}.

One advantage of fitness-based diversity measures is that no extra computations are associated with calculating the diversity, because the fitness values already have been calculated by the GA to assess how fit each individual is~\cite{Nguyen:2006:ASPGP}.

Taking the precomputation of fitness values into account gives these diversity measures an advantage when it comes to complexity. To calculate the number of distinct individuals, we can use a hash table. If we assume it unlikely to have a clash between hash values and choose to ignore this, we can achieve a complexity linear in the size of the population, or \bigO{\indsetl} for fitness-based diversity measures.

\subsubsection{Other measurements}
Some diversity measures exist that are neither genotypic nor phenotypic. For instance the \emph{Ancestral ID} method, which assigns a unique ID to each individual in the initial population. Every mutated individual receives a new unique ID while every child gets the ID of one of its parents. The diversity is then based on the uniqueness of IDs in a population~\cite{1250187}.


\section{\di{}}
In the following, we present a diversity measure for GAs using neural networks, based on the Simpsons Diversity Index (SDI), which we call \di{} (\dia{}). We will thus use the term individual and neural network interchangeably. Let $F = \set{f_1, f_2, \dots, f_n}$, where each $f_j$ for $0 \leq j \leq n$ denote the set of neural networks contained in a population, which all have the same architecture of $a$ input and $b$ output neurons. 

\di{} is calculated with respect to a number of random inputs $\set{R_1, R_2, \dots, R_m}$, where each $R_i$ for $1 \leq i \leq m$ is an $a$-tuple of real values chosen randomly. For each sample $R_i$, a SDI is calculated. SDI is a diversity measure used in ecology to quantify the biodiversity of a habitat~\cite{simpson1949measurement}. The SDI is calculated with
%
\begin{equation*}\label{eq:sdi2}
  D_i = 1 - \left(\frac{\sum_{j=1}^{k}S_j\left(R_i\right)\left(S_j\left(R_i\right) - 1\right)}{\lvert F\rvert \left(\lvert F\rvert - 1\right)}\right) 
\end{equation*}
%
where $s$ is the total number of individuals of a particular species, and $S$ is the total number of individuals of all species. $D$ is calculated $m$ times, and the average of the $m$ SDI values will define the \dia{}. We classify species of individuals using $S_k(R_i)$ as the set of neural networks belonging to the $k$th species with respect to input $R_i$.

We distribute the neural networks into species based on which one of their output neurons yields the highest value on input $R_i$. %This means that for any neural network belonging to species $S_k(R_i)$, the value of the $k$th output neuron will be greater than or equal to the value of any other output neuron given input $R_i$.
This definition implies that the number of species equals the number of output neurons $b$, e.g.\ $1 \leq k \leq b$. We distribute neural networks into a set of species as follows
%
\begin{equation*}\label{eq:species}
  S_k\left(R_i\right) = \setof{f_j}{\forall l \in \set{1, 2, \dots, b} \left(\sigma_{kji} \geq \sigma_{lji}\right)}
\end{equation*}
%

where $\sigma_{xyz} \in \set{0, 1}$ is the output of the $x$th neuron in neural network $f_y$ on input $R_z$.

This species classification implies that a single network can only be part of one species per input. One disadvantage of \dia{} is that it relies on random inputs, which means that fewer random inputs implies less statistical significance.

\section{Experiments}
%In this section, we evaluate our proposed method of a better diversity measurement (NNTD) by using it to measure diversity of the replacement rules described in \cref{sec:replacementrules}. This is to compare each replacement rules' influence on diversity using NNTD.
Three problems instances are used to evaluate our diversity measure. Two problems discrete, and one continuous. A standard implementation of GAs, ANNs, replacement policies, and diversity measures is used (see~\cite{mbm:kmc:ekoGA} for the authors' implementation details). 

\subsection{Parameter settings}

\begin{table}
  \centering
  \begin{tabular}{l S}
    \toprule
    Parameter & {Specification} \\
    \midrule
    Number of runs & 1000 \\
    Generations per run (continuous) & 2000 \\
    Generations per run (discrete) & 51 \\
    Population size & 100 \\
    Selection method & {rank-based} \\
    \bottomrule
  \end{tabular}
  \caption{GA parameters used throughout experimenting.}
  \label{tab:gaparam}
\end{table}

Three crossover methods are used: one-point crossover, two-point crossover, and uniform crossover. Upon creating a new population, there is an equal chance for each to be chosen.

Half of the offspring is created from crossover between two selected parents in population $p$, the other half created solely by mutation of the individuals also in $p$. Of the individuals in $p$'s offspring belonging to the half created by crossover, there is a \perc{10} chance that the individual also will be selected for mutation. For each bit belonging to the individuals selected for mutation, there is a \perc{5} chance that that will be mutated. Of the bits to be mutated, each bit has a \perc{50} chance it will remain the same, and a \perc{50} chance to be inverted.

\subsection{Problems}
\label{sec:problems}
We hereby explain the three discrete problems we perform our experiments on.

\subsubsection{XOR}
We use a neural network to approximate the XOR between two 8-bit strings.
To evaluate fitness of a neural network, we choose 1000 random instances of two 8-bit strings.
For each instance, we determine how many bits the neural network calculates correct.
The fitness is the defined as the average number of bits calculated correct.
Fitness values will thus lie in the range $[0-8]$.
Since the XOR between two random bits will be evenly distributed between 0 and 1,
randomly guessing a solution to the 8-bit XOR problem, is expected to yield a fitness of 4.
The network has 16 input neurons, 16 hidden neurons, 8 output neurons, 2 bits per weight, and 1 bit per bias for each hidden and output neuron.
Any output neuron $i$ represents the XOR between the two input neurons $i$ and $i+8$. 
We have used as few neurons and bits to represent weights and biases as possible, while still being able to verify that the maximum fitness value of 8 is achievable.
The same random seed is always used for generating the 1000 problem instances.

\subsubsection{Leaf classification}
We use a neural network to classify the Leaf data set~\cite{Bache+Lichman:2013, leafdataset}.
The neural network is given 16 properties about an unknown leaf and has to decide which of 40 types of leaves it is.
Fitness is evaluated based on how many instances out of the entire data set the neural network correctly classifies.
The implementation consists of 16 input neurons, 10 hidden and 40 output neurons. The output neuron with the highest value decides the classification. Each weight is encoded by 9 bits and neurons have no bias.

\subsubsection{Snake}
Snake is a game found on old Nokia cell phones, where you are moving a snake around a grid to pick up pieces of food.
Every time a piece of food is collected, both the length of the snake and your score increases by 1.
You loose if the snake head hits its body or one of the edges of the grid.
At all times, the grid contains only a single piece of food.
The game becomes harder as the length of the snake increases, and as the snake is constantly moving, it becomes hard to not trap yourself.

We use a neural network to play a game of Snake in a $10\times10$ grid with an initial snake length of 5 units.
We have defined the fitness of a neural network to be $f + \frac{f}{s/1000}$,
where $f$ is the amount of food it collects and $s$ is the total number of steps the snake is alive. The game is constrained such that the snake can only change its direction \SI{90}{\degree} per step. The neural network has 6 input neurons, each receiving a bit of information about the game state:
\begin{enumerate}
  \item \set{-1, 0, 1} Whether food is to the left, vertically aligned, or to the right of the snake head.
  \item \set{-1, 0, 1} Whether food is above, horizontally aligned, or below the snake head.
  \item \set{0, 1} If the snake dies if its next move is up
  \item \set{0, 1} If the snake dies if its next move is down
  \item \set{0, 1} If the snake dies if its next move is right 
  \item \set{0, 1} If the snake dies if its next move is left
\end{enumerate}
4 output neurons are used. The neuron with the highest value determines whether to move the snake right, left, up, or down.
The neural network uses 5 hidden neurons, 9 bits per weight and neurons have no bias.
For every game of Snake, we always use the same random seed to decide the positions where pieces of food will spawn.

\subsubsection{Criteria for selection}
The XOR function is interesting since it is the simplest boolean function that is not linearly separable.
This fact has made it quite popular in NN research communities\cite{masterThesisGANN}.
A classification problem like Leaf is interesting because it is radically different from the XOR problem.
XOR is a simple and well defined function, whereas classifying leaves is more complex and depends on observations in nature, which may contain noise.
The Snake game differs in that it is an agent decision problem, where not just a single, but a sequence of decisions determines the outcome, where each decision changes the intermediate state.

\subsection{Diversity measurements}
To test our NNTD method, we introduce two variables, both of which only have an effect during the very first generation of a population: initial similarity and initial mutation of a population. 

Initial similarity describes how equal the individuals initially are. It can take on a range between \num{0} and \num{1}, where \num{0} means that all individuals are random, and \num{1} denotes that \perc{100} all individuals have the exact same genotype. With a value of \num{0.5}, one individual is cloned so that half of the population has the same genetic makeup and the other half is initialized to be completely random. A population with the value of \num{1} for initial similarity should have the lowest diversity, and a completely randomly initialize population (initial similarity of \num{0}) should have the highest diversity, since every individual is randomly chosen.

Initial mutation requires an initial similarity set to \num{1}, e.g.\ the initial population consists of a single, cloned individual. This is due to the fact that mutating individuals already randomly initialized will not yield a different diversity of individuals. The initial mutation rate simply signifies the mutation percentage of all bits in every individual of the population. A population consisting of the same cloned individual with an initial mutation rate of \num{1} should have the highest diversity, since every bit in every, initially similar individual, will have a \perc{100} chance to be selected for mutation, resulting in a randomly initialized population.

Experiments on NNTD using both of these two variables are shown in \cref{fig:initial-mutation-similarity}. Each line represents the average diversity measure of \num{100} runs at a variable intervals of \num{0.05}. As we can see by the chart, diversity increases as expected for both variables. Diversity is maximum at a value of \num{7.2} for a completely random population. To reach this maximum, it requires an initial mutation rate of a mere \perc{2} and an initial similarity of \perc{0}.

\begin{figure}[htpb]
  \centering
  \inputresize{drawings/initial-mutation-similarity/graph}
  \caption{Diversity measurements with NNTD, given increased ranges of initial similarity and initial mutation rates.}\label{fig:initial-mutation-similarity}
\end{figure}

\begin{algorithm}
  \caption{Procedure AESP}\label{alg:aesp}
  \begin{algorithmic}[1]
  \Procedure{aesp}{$G_k, O_k$}
    \State $G_{k+1} \gets G_k$
    \ForAll{$o \in O_k$}
      \If {$o.\mathtt{num\_parents} = 1$}
        \If {$\beta(o) > \beta(o.p_1)$}
          \State $G_{k+1} \gets \left(G_{k+1} \setminus \set{o.p_1}\right) \cup \set{o}$
        \EndIf
      \Else
        \If {$\beta(o) > \max\left(\beta(o.p_1), \beta(o.p_2)\right)$}
        \State  $G_{k+1} \gets$ \begin{varwidth}[t]{\linewidth}$(G_{k+1} \setminus \{o.p_1, o.p_2\})$\par
          \hskip\algorithmicindent $\cup \set{o} \cup \set{\rho}$
        \end{varwidth}
        \EndIf
      \EndIf
    \EndFor
  \EndProcedure
  \end{algorithmic}
\end{algorithm}



\bibliographystyle{abbrv}
\bibliography{bibliography}


% Appendices can be added with \appendix[Proof of Khazens's equation]
\clearpage
\appendix
\section{Artificial Neural Networks and Genetic Algorithms}
\label{ap:ann-ga}

\subsection{Artificial Neural Network}
An artificial neural network is a graph structure that, from the outside, can be seen as a black box, that given the values $x_1, x_2, \dots, x_n$ outputs the values $y_1, y_2, \dots, y_m$. With the right internal structure, a neural network can be used for a variety of purposes, e.g.\ face recognition, where the intensities of different pixels in an image are used as input, and a single output value $y_1$ is produced, where $y_1 = 1$ if the image was of a face and $0$ if not. We now describe the structure and inner workings of neural networks to understand how these output values are calculated based on input values.

\subsubsection{Neurons}
Nodes in the graph of a neural network are called neurons. Three classes of neurons exist: input, hidden, and output. Input neurons receive $x_1, x_2, \dots, x_n$ and output the same value. Hidden and output neurons take a number of values as input from edges exiting other neurons, applies a weight to each value, sums them, then applies a function to produce a single output value. The function applied is called the transfer function, and is the same for all hidden and output neurons in the network. This is recursively expressed by 
% Note: bias/threshold is not defined below, add if needed
\begin{equation*}\label{eq:weightcalc}
  y_i =
  \begin{cases}
    \var{x}_i                     & \text{if } i \text{ is an input neuron} \\
    \theta\left(\sum_{j=1}^n w_{ji} y_{j} \right) & \text{otherwise} %- \theta_i \right)
  \end{cases}
\end{equation*}
%
where $x_i$ is the input to input neuron $i$, $n$ is the amount of neurons, $y_i$ is the output value of neuron $i$, $w_{ji}$ is the weight of the edge from neuron $j$ to $i$ ($0$ if no connection exists), $y_j$ is the output of neuron $j$, and $\theta$ is the transfer function, usually defined to be the sigmoid function taking the form
%
\begin{equation*}
  \theta(t) = \frac{1}{1+e^{-t}}
\end{equation*}
%
In a feedforward network, neurons are placed in one or more layers in an acyclic graph structure, where each neuron in layer $i$ is connected to every neuron in layer $i + 1$. The value output by the last layer, or the output layer, becomes the output of the neural network. \Cref{fig:ann} illustrates the generic graph structure of a feedforward neural network with a single hidden layer.
%
\begin{figure}[htpb]
  \centering
  \includestandalone[mode=buildnew, width=\linewidth]{drawings/ANN/ANN}
  \caption{Structure of a neural network.}
  \label{fig:ann}
\end{figure}
%
\subsubsection{Training a Neural Network}
Any applica\-tion of a neural network requires a suitable number of layers, neurons, and weights between neurons, to adequately solve a given problem. How many hidden neurons and layers to have is a highly debated subject, see \cite{sarle1997}.

The weights on edges connecting neurons are decided by a process called training. Well known training algorithms, such as backpropagation, typically require that for each input to the neural network, the correct output is already known. This kind of learning is called supervised learning. For the application of face recognition, this means that the pictures used for training, each has a predicate indicating whether or not it is a picture of a face.

\todo{We need to fix the following two paragraphs. Change them to a shorter, more concise argument against backpropagation in general terms. Cite articles that argument against BP in GA.}
For some purposes, the desired output of a neural network is not known beforehand. Consider for instance a computer game, where a player is controlled by a neural network. The neural network takes values indicating the state of the game as input, such as the observed position of the players and enemies. For each input, the neural network returns a single value indicating which action the player should take, e.g.\ move left, move right, or jump. Given a state of the game, we might not be able to say whether an action output by the neural network is right or wrong. It might be wrong to move closer to an enemy if he eliminates you, but it might be right if the next action is successfully to eliminate the enemy.

From this, it is clear that a backpropagation algorithm is not appropriate for training a neural network to control the actions of an artificial intelligent player in a computer game. For this purpose we propose another approach. We can tell how good a neural network performs, or how fit it is, by simulating a game being played using the neural network to control the player and determining the score achieved in the game, or any other function indicating how well the artificial player performed according to some criteria using that particular neural network. By being able to measure the fitness of a particular neural network, we can use genetic algorithms to create and search for the best performing neural network.

\subsection{Genetic algorithms} 
Genetic algorithms are optimization algorithms, which imitate the process of natural selection in the search of a global optimum.

\subsubsection{Individuals and chromosomes}
GAs maintain a list of \emph{individuals}, which together form a \emph{population}. Each individual represents a possible solution to the optimization problem in question and has a fitness value, denoting how adequately the individual can solve the optimization problem. Individuals in GAs can take on any form of data structure, as long as they wholly represent a possible solution to the problem. An individual is encoded by its \emph{chromosome}, which is typically represented by a bit string. A substring of this bit string is called a \emph{gene}. The GA manipulates the bit strings to form new individuals. Therefore, using a GA requires a way of encoding/decoding between an individual's chromosome and a solution to the given problem. We refer to the individuals' chromosomes or bit strings synonymously.

The population used by a GA typically has a fixed number of individuals, each initialized with a random chromosome when the GA is run. That is, the bit string representing the chromosome is initialized with random bits. As the GA iterates, new individuals are made by combining and modifying chromosomes from existing individuals of the population. In steady state GAs, some of the new individuals will replace older individuals according to a replacement rule. In contrast, a generational GA will choose only from offspring when forming the next generation~\cite{fogarty, Syswerda:1989:UCG:645512.657265, Whitley:1989:GAS:93126.93169}. We will focus on steady state GAs, using the term \emph{generation} \generation{n} to denote the content of the population after $n - 1$ iterations. A replacement rule then determines which individuals of the $n$th generation \generation{n}, and their offspring \offspring{n} are selected to form the next generation \generation{n+1}.

\subsubsection{Neural networks as individuals}
%bitstrings
%weight on connections
%each input and each output
%amount of hidden neurons

As we have discussed, neural networks are ideal for decision making and hence are appropriate as individuals in a GA. To support various GA operators, individuals are encoded as bit strings.

% Uncommented this paragraph friday 2014-3-21, as it just says what a NN is? -Martin
%Each possible input an individual can get will be received through an input neuron. Likewise, each possible action an individual can perform is formulated through the output neurons. The network is constructed with connections between neurons with associated weights. These weights are used to calculate an action given the actual input.

When using a GA, all individuals are neural networks with the same architecture.
That is, the neural networks differ only in their weights between neurons and the bias of each neuron.
Each individual is therefore represented only in terms of the weights and biases.
For each GA, any weight and bias is encoded with a fixed number of bits $n$ and $m$, respectively.
The bit string is constructed in an ordered manner, such that the first $n$ bits represent the weight for the first connection between the first input neuron and the first hidden neuron, the next $n$ bits represent the weight for the connection between the first input neuron to the second hidden neuron, and so forth. \cref{fig:entire-eqnetwork} shows an example of two neural networks and the bit string that encodes each of them.
If biases are used by the GA, these comes after all the encoded weights.
They are ordered such that the first $m$ bits encodes the bias of the first neuron, the next $m$ bits encodes the bias of the second neuron, and so forth.
In this way, we encode a neural network, which is manipulated by different GA operators.
From now on, we will refer to the bit string encoding of a neural network as the individual's \emph{chromosome}.

% ALso removed this friday 2014-3-21, no need to reference it twice. Possibly if there was an example on how to encode the network in the example?
%An illustration of neural networks is presented in \cref{fig:ann}, and should give a good idea of how the bit string is concatenated.

%\[
%  \underbrace{w_{1,1}}_{n} w_{1,2} \ldots w_{i,j}
%\]

%\[
%  \ldots \underbrace{w_{i,j-1}}_{n} w_{i,j} w_{i,j+1} \ldots
%\]

%where $w_{i,j}$ represents the weight of the connection between the $i'th$ and $j'th$ neuron in bits. Each weight is concatenated with the next weight. The length of any weight is of size $n$.
%
If two chromosomes have different bit strings, we say that they have different genotypes. If the neural network they encode produces a different output for some input, we say they have different phenotypes.

%\subsubsection{Genes}
%Each individual consists of many \emph{genes} as part of its chromosome. Genes constitute the DNA of the individual. These genes are an encoding of some attribute or skill the individual has. Because we defined neural networks to be individuals, the weights and node biases consist of the DNA.

% This subsubsection is already described in individuals and hcromosomes
%\subsubsection{Populations}
%A genetic algorithm manages a collection of many individuals, known as a population. Individuals in the first population are usually initialized randomly with a fixed population size. The goal for these individuals is to solve a problem optimally. Each individual in the population has a fitness level that defines the individual's ability to solve a given problem.
%Initially, there are no generations. Creating a new population from a previous population increases the amount of generations by one, hopefully yielding a net increase in average fitness.

\subsubsection{Crossover and mutation}
In natural evolution, a pair of individuals come together to produce one or more children, having genes from both parents. In GAs, the process of procreation is done by performing a \emph{crossover} of the two parent individuals' chromosomes. Parts of each parent's bit strings are used to create the child individual.

\emph{Mutations} can also occur randomly at any point in time upon creating a child individual. If genes are encoded as bit strings, then a mutation arbitrarily toggles one or more bits. This ensures that any gene can be formed.

\subsubsection{Fitness functions}
A \emph{fitness function} must be defined to calculate the desirability of each individual. This function is used to define the most fit individuals in a population. By giving more fit individuals a greater chance of reproducing, the intuition is that more fit individuals will be created, having the best traits from each of their parents.




\end{document}
