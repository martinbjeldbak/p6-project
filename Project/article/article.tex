\documentclass[a4paper]{IEEEconf}
\synctex=1

\usepackage[utf8]{inputenc}
\usepackage[T1]{fontenc}
\usepackage[english]{babel}
\usepackage[cmex10]{amsmath}
\interdisplaylinepenalty=2500
\usepackage[nocompress]{cite}
\usepackage{graphicx}
\usepackage{algorithm}
\usepackage[noend]{algpseudocode}
\algrenewcommand\algorithmicindent{0.8em}%
\usepackage[colorinlistoftodos]{todonotes}
\presetkeys{todonotes}{inline}{}
\usepackage{caption}
\usepackage{subcaption}
\usepackage{booktabs}
\usepackage{hyperref}
\usepackage{tikz}
\usetikzlibrary{positioning}
\usepackage{pgfplots}
\usepackage{cleveref}
\usepackage{mathtools}
\usepackage{nag}
\usepackage[activate={true,nocompatibility},final,tracking=true,kerning=true,spacing=true,factor=1100,stretch=10,shrink=10]{microtype}
% activate={true,nocompatibility} - activate protrusion and expansion
% final - enable microtype; use ``draft'' to disable
% tracking=true, kerning=true, spacing=true - activate these techniques
% factor=1100 - add 10% to the protrusion amount (default is 1000)
% stretch=10, shrink=10 - reduce stretchability/shrinkability (default is 20/20)
%
\definecolor{maroon}{HTML}{85144B}
\definecolor{navy}{HTML}{001F3F}
\definecolor{blue}{HTML}{0074D9}
\definecolor{aqua}{HTML}{7FDBFF}
\definecolor{teal}{HTML}{39CCCC}

\usepackage[locale=US]{siunitx}
\newcommand{\perc}[1]{\SI{#1}{\percent}}
\newcommand{\di}{Simpsons Diversity Index NN}
\newcommand{\dia}{SDINN}
\newcommand{\fit}{\phi}
%\sisetup{group-separator = {,},
%         per-mode = symbol,
%         inter-unit-product = \ensuremath{\cdot},
%         number-unit-product = \text{ },
%         binary-units = true}


\newcommand{\mail}[1]{\href{mailto:#1}{#1}}
\newcommand{\var}[1]{\ensuremath{\mathit{#1}}}
\newcommand{\citpls}[1]{\footnote{\textcolor{red}{ Citation needed here.}}}
\newcommand{\set}[1]{\ensuremath{\left\{ #1 \right\}}}
\newcommand{\setof}[2]{\ensuremath{\left\{ #1 \mid #2 \right\}}}

\newcommand{\correctlyresize}[2]{\resizebox{#1}{!}{#2}}
\newcommand{\inputresize}[1]{\correctlyresize{\linewidth}{\input{#1}}}
\newcommand{\inputresizeto}[2]{\correctlyresize{#1}{\input{#2}}}

\pgfplotsset{%
    compat=newest,
    grid=major,
    grid style={dashed,gray!30},
    scale only axis,
    initial-mut/.style={%
      ylabel = {Diversity},
      xlabel = {Initial Mutation},
    },
    initial-sim/.style={%
      ylabel = {Diversity},
      xlabel = {Initial Similarity},
    },
    initial-mut-sim-root/.style={%
      legend columns=-1,
      legend entries={Fitness-based, Hammming distance, \dia{}},
    },
    initial-sim-root/.style={%
      initial-mut-sim-root,
      initial-sim,
    },
    initial-mut-root/.style={%
      initial-mut-sim-root,
      initial-mut,
    },
  }

\begin{document}

\title{Measuring Diversity in Steady State Genetic Algorithms using Artificial Neural Networks}

%\author{%
%  Kent Munthe Caspersen \\
%  \email{kcasp11@student.aau.dk}
%  \and
%  Elias Khazen Obeid \\
%  \email{eobeid11@student.aau.dk}
%  \and
%  Martin Bjeldbak Madsen \\
%  \email{mbma11@student.aau.dk} \\
%  \begin{affiliation}
%    Department of Computer Science\\
%    Aalborg University\\
%    9220 Aalborg
%  \end{affiliation}
%}

\maketitle

\listoftodos

\begin{abstract}
  \IEEEPARstart{W}{ith} neural networks, one can\ldots
\end{abstract}


\section{Introduction}
Neural networks are often chosen as individuals in a population within genetic algorithms. These individuals procreate to form a new generation by creating offspring. Offspring consists of some combination of parent individuals and usually replace the individuals that are worst off in the preceding generation, simulating natural selection.


\subsection{Motivation}
\label{sec:motivation}

%Diversity is important. Avoid retardation.
%Local optimum. The best breed. Low diversity.
%Maintain high diversity. Slow search for optimum.

Genetic algorithms rely on the individuals in a population to be able to evolve. Each individual tries to solve a given problem in its own way. Naturally, this implies that it is important to maintain a divers set of individuals to be able to keep evolving and trying different solutions to the problem. Otherwise, if diversity is not maintained within the population, a global optimal solution might never be found.
\cite{ursem2002diversity}

If diversity is not taken into account, it will of course quickly become low. The population will contain more and more similar individuals, which actually stops the much needed evolution. This approach produces a local optimal solution, by letting the best of the individuals breed and produce offspring, without any kind of intervention to try to maintain diversity. 

The benefit of maintaining a divers population, is a higher likelihood that the outcome will be a global optimal solution. The downside of maintaining diversity, is that the search for a solution becomes much slower. We propose an approach where the diversity is maximised and the time searching for a solution is minimised.


\subsection{Artificial Neural Network}
An artificial neural network is a graph structure that can from the outside be seen as a black box, that given the values $x_1, x_2, \dots, x_n$ outputs the values $y_1, y_2, \dots, y_m$. With the right internal structure, a neural network can be used for a variety of purposes, e.g.\ face recognition, where the intensities of different pixels in an image are used as input, and a single output value $y_1$ is produced, where $y_1 = 1$ if the image was of a face and $0$ if not. We now describe the structure and inner workings of neural networks to understand how these output values are calculated based on input values.

\subsubsection{Neurons}
Nodes in the graph of a neural network are called neurons. A neuron takes a number of values as input from edges exiting other neurons, applies a weight to each value, sums them, then finally applies a function to produce a single output value. The function applied is called the transfer function, and is the same for all hidden neurons in the network. This is recursively expressed by the recurrence
% Note: bias/threshold is not defined below, add if needed
\begin{equation}
  y_i =
  \begin{cases}
    \var{input}_i & \text{if } i \text{ is an input neuron} \\
    f\left(\sum_{j=1}^n w_{ji} y_{j} \right) & \text{else} %- \theta_i \right)
  \end{cases}
\end{equation}
where $\var{input}_i$ is the value given to input neuron $i$, $n$ is the amount of neurons, $y_i$ is the output value of neuron $i$, $w_{ji}$ is the weight of the edge from neuron $j$ to $i$ ($0$ if no connection exists), $y_j$ is the output of neuron $j$, and $f$ is the transfer function, usually defined to be the sigmoid function taking the form
\begin{equation*}
  f(t) = \frac{1}{1+e^{-t}}
\end{equation*}

In a feedforward network, neurons are placed in different layers, where each neuron takes the output of all neurons from the previous layer as input. Neurons in the first layer are called input neurons and receive values as input to the neural network.  The neurons in the last layer are called output neurons. The value output by these neurons becomes the output of the neural network. Any neuron in between these layers is called a hidden neuron. \Cref{fig:ann} illustrates the generic graph structure of a feedforward neural network with a single hidden layer.

\begin{figure}[htpb]
  \centering
  \includestandalone[mode=buildnew, width=\linewidth]{drawings/ANN/ANN}
  \caption{Structure of a neural network.}
  \label{fig:ann}
\end{figure}

\subsubsection{Training a neural network}
Any application of a neural network requires that a correct number of layers, neurons, and weights between neurons are found to adequately solve the problem. In many applications, a single layer of $k$ hidden neurons works well, where $k = 0.5\left(\mid\var{inputNeurons}\mid \times \mid\var{outputNeurons}\mid\right)$\citpls. The weights on edges connecting neurons are decided by a process called training. Well known training algorithms, such as backpropagation, typically require that for each input to the neural network, the output is already known. This kind of learning is called supervised learning. For the application of face recognition, this means that the pictures used for training, each has a predicate indicating whether or not it is a picture of a face.\footnote{Should we explain backpropagation more in detail?}

For some purposes, the desired output of a neural network is not known given a number of input values. Consider for instance a computer game, where a player is controlled by a neural network. The neural network takes as input values indicating the state of the game, such as the position of the players and enemies around him. For each input, the neural network returns a single value indicating which action the player should take, e.g.\ move left, move right, or jump. Given a state of the game, we might not be able to say whether an action output by the neural network is right or wrong. It might be wrong to move closer to an enemy if he eliminates you, but it might be right if the next action is successfully to eliminate the enemy.

From this, it is clear that a backpropagation algorithm is not appropriate for training a neural network to control the actions of a artificial intelligent player in a computer game. For this purpose we propose another approach. We can tell how good a neural network performs, or how fit it is, by simulating a game being played using the neural network to control the player and determining the score achieved in the game, or any other function indicating how well the artificial player performed according to some criteria using that particular neural network. By now being able to measure the fitness of a particular neural network, we can use genetic algorithms to create and search for the best performing neural network.

\subsection{Genetic algorithms}
Genetic algorithms belong to the class of evolutionary algorithms. This class of algorithms is inspired by natural evolution as seen in biology. Genetic algorithms are optimization algorithms which imitate the process of natural selection in search of global maximum.

\subsubsection{Individuals and chromosomes}
A genetic algorithm (GA) maintains a list of \emph{individuals} which together forms a \emph{population}. Each individual represents a possible solution to the optimization problem and has a fitness value, which denotes how adequately the individual can solve the optimization problem. How an individual solves the optimization problem is determined by its \emph{chromosome}, which is typically represented by a bit string. Therefore, using a GA requires a way of decoding a chromosome into a solution to the optimization problem. The population used by a genetic algorithm typically has a fixed number of individuals, who are all initialized randomly when the GA is initially run. That is, the bit string representing the chromosome is initialized with random bits. As the GA iterates, new individuals are made by combining and modifying chromosomes from existing individuals of the population. Some of the newly created individuals will replace the older individuals using a replace policy that over time aims to maximize the fitness of the best individuals. After each iteration of the GA, where a new population is formed, we say that we have another \emph{generation} of individuals.

\subsubsection{Individuals}
Individuals in evolutionary computational algorithms have a sets of traits and behaviors that define each individual. They can take on any form of data structure, as long as they wholly represent a possible solution to the problem. 

%\subsubsection{Genes}
%Each individual consists of many \emph{genes} as part of its chromosome. Genes constitute the DNA of the individual. These genes are an encoding of some attribute or skill the individual has. Because we defined neural networks to be individuals, the weights and node biases consist of the DNA.

% This subsubsection is already described in individuals and hcromosomes
%\subsubsection{Populations}
%A genetic algorithm manages a collection of many individuals, known as a population. Individuals in the first population are usually initialized randomly with a fixed population size. The goal for these individuals is to solve a problem optimally. Each individual in the population has a fitness level that defines the individual's ability to solve a given problem.
%Initially, there are no generations. Creating a new population from a previous population increases the amount of generations by one, hopefully yielding a net increase in average fitness.

\subsubsection{Crossovers and mutations}
In natural evolution, a pair of individuals come together to produce one or more new child individuals, with genes from both of the parent individuals. The process of procreation is done by performing a \emph{crossover} of the two parent individuals' genes.

% vvvvvv <= below a crossover method has been defined, when there are many different kinds - Martin
%A crossover point is defined and one part of each of the parents is copied and combined to form a new set of genes for the child individual.

%illustration of crossover and mutation

Mutations can occur randomly at any point in time upon creating a child individual. If genes are encoded as bit strings, then a mutation arbitrarily toggles one of the bits. This ensures that the population can evolve if no progress would be made if the genes did not allow it.

\subsubsection{Fitness functions}
A fitness function must be defined to calculate the desirability for each individual. This function is used to define the most fit individuals in a population. The higher the fitness level is of a given individual, the higher the chance it has to be chosen to reproduce with another individual. The intuition behind this is that choosing to very fit individuals to crossover will create an even better individual with the best traits of each of its parents.

\subsubsection{Neural networks as individuals}
%bitstrings
%weight on connections
%each input and each output
%amount of hidden neurons

Neural networks can be used to solve many types of problems, e.g.\ classification and decision making.
Before we can use neural networks as individuals in a GA, we must define a method for encoding a neural network as a bit string, which will be manipulated by the GA\@. For any GA, we will assume that every individual in its population will have the same architecture. That is, the number of neurons, the size of each layer, and how neurons are connected is the same.

% Uncommented this paragraph friday 2014-3-21, as it just says what a NN is? -Martin
%Each possible input an individual can get will be received through an input neuron. Likewise, each possible action an individual can perform is formulated through the output neurons. The network is constructed with connections between neurons with associated weights. These weights are used to calculate an action given the actual input.

Thus, neural networks differ only in their weights between neurons and the bias of each neuron. Each individual is therefore represented only in terms of the weights and biases. For each GA, any weight and bias is encoded with a fixed number of bits $q$ and $w$, respectively. The bit string is constructed in an ordered manner, such that the first $q$ bits represent the weight of the connection between the first input neuron and the first hidden neuron, the next $q$ bits represent the weight of the connection between the first input neuron and the second hidden neuron, and so forth. \Cref{fig:entire-eqnetwork} shows an example of two neural networks and the bit string that encodes each of them. If biases are used by the GA, these are encoded right after the weights, and ordered such that the first $w$ bits encode the bias of the first neuron, the next $w$ bits encode the bias of the second neuron, and so forth. 
We limit the weights and biases of any neural network to lie in the range $[-5,5]$.
This is because we use the sigmoid activation function, for which the value $f(x)$ changes only a little when $x \leq -5$ and $x  \geq 5$. If a weight or bias of a neural network is encoded by the $u$ bits $b_1 b_2 \cdots b_u$, we decode its real value using the formula
\[
w = (b_12^0 + b_22^1 + \cdots + b_{u-1}2^{u-2})(1-2b_u)\frac{5}{2^u-1}
\]
%\[
%unsigned = b_12^0 + b_22^1 + \cdots + b_{u-1}2^{u-2}\\
%signed = unsigned(1-2b_{u})\\
%weight = 5\frac{signed}{2^u-1}\\
%\]
The first $u-1$ bits represent an increasing sequence of powers of two. The last bit $b_u$ negates the entire value if set (by the factor $(1-2b_u)$). The value is finally normalized to the range $[-5, 5]$ (by the factor $\frac{5}{2^u-1}$).
The number of bits $u$ used to encode a weight or bias is dependent on the problem in question.
If two chromosomes have different bit strings, we say that they have different genotypes. If the neural network they encode produces a different output for some input, we say they have different phenotypes.
% ALso removed this friday 2014-3-21, no need to reference it twice. Possibly if there was an example on how to encode the network in the example?
%An illustration of neural networks is presented in \cref{fig:ann}, and should give a good idea of how the bit string is concatenated.

%\[
%  \underbrace{w_{1,1}}_{n} w_{1,2} \ldots w_{i,j}
%\]

%\[
%  \ldots \underbrace{w_{i,j-1}}_{n} w_{i,j} w_{i,j+1} \ldots
%\]

%where $w_{i,j}$ represents the weight of the connection between the $i'th$ and $j'th$ neuron in bits. Each weight is concatenated with the next weight. The length of any weight is of size $n$.
%


If two chromosomes have different bit strings, we say that they have different genotypes.
If the behaviour they encode are different, that is, the neural network they encode produces a different output for some input,  we say they have different phenotypes.

\subsection{Measuring diversity}
In GAs, a great diversity among the individuals is important.
It is often argued that the weakness of GAs is the fall in diversity over generations, which causes the GA to do a simple local hill climbing.

Common methods for measuring diversity focuses on either genotypic or phenotypic diversity.
When chromosomes encode neural networks, one drawback of measuring diversity by comparing genotypes is that two individuals of different genotypes can have the same phenotype. Thus a high diversity does not necessarily imply that individuals with many different properties are represented in the population.
%\todo{include a picture showing two different neural networks that always produces the same output}.
When basing diversity measures on phenotypes, a formula is needed for calculating how different two phenotypes are. 
Usually, the phenotypic diversity is measured only based on the fitness value of each phenotype.
%http://citeseerx.ist.psu.edu/viewdoc/download?doi=10.1.1.104.912&rep=rep1&type=pdf
The advantage is that it requires no extra computational power. 
The drawback is that two individuals with the same fitness value might still have different traits, that is, the way of achieving these fitness values. We think that a high diversity in a population should reflect many different traits among the individuals. In the following, we propose a method for measuring diversity based on based on the different traits of the individuals as well as a method that increases this diversity measure. 

\section{Preliminaries}
\label{sec:preliminaries}

In the following sections we assume that the reader is familiar with neural networks and genetic algorithms. Otherwise, see \cref{ap:ann-ga} for explanations of these. 

In GAs, a high diversity among the individuals is important. It is often argued that the weakness of GAs is the fall in diversity over generations, which can result in premature convergence~\cite{diaz2007empirical, 1266373,Zitzler00comparisonof}.

We begin by presenting crowding methods, which use special selection procedures to overcome a decreasing diversity. Next, we introduce different methods of measuring diversity, and finish this section by explaining our own proposal for measuring diversity.

%We begin by presenting methods which propose solutions to overcome the decreasing nature of diversity in the populations.

%In this section we present the structure of artificial neural networks and concept of genetic algorithms. Furthermore, we introduce the concept of diversity measurement for genetic algorithms.

\subsubsection{Crowding}
\label{sec:crowding} 
A simple way to compose the $i$'th generation, denoted $G_i$, is by selecting the $n$ best individuals from the union of $G_{i-1}$ and $O_{i-1}$, where $O_{i-1}$ is the offspring produced from $G_{i-1}$ by selecting a number of individuals, where more fit individuals are more likely to be selected.
It is easy to see that this method causes a low diversity over time - since the best individuals are more likely to be selected for procreation, much of the offspring will have similar properties that yields a high fitness value and thus be a part of the next generation just like their parents. Now the chance is even greater that the same properties will be spread out even further in the next generation.

Many methods have been proposed to overcome the problem that diversity decreases through generations - these methods include inserting new random individuals into the population (called random immigrants), removing a number of individuals from the population, using complex population structures to lower the gene flow, and the use of special selection procedures\cite{ursem2002diversity}. The latter is known as crowding. 
An example of crowding is to dictate that a child only can be added to the population if one of its parents is removed.
Crowding has been found to work well on test functions as well as other applications\cite{crowding}. 

\subsubsection{Replacement rules}
A \emph{replacement rule} can be used when two or more individuals are very similar, and one wishes to replace them all by just a single individual. This will decrease the similarity of individuals and thus increase diversity.

An example is the probabilistic crowding replacement rule\cite{Mengshoel_and_Goldberg:1999}

\[p_x = \frac{f(x)}{f(x)+f(y)}\]

where $p_x$ denotes the chance that individual $x$ replaces the similar individuals $\{x, y\}$, $f(x)$ is the fitness of $x$, and $f(y)$ is the fitness value of $y$. This rule can easily be generalized for larger family size. 
In an attempt to keep a low diversity, the rule can be applied between a two parents and their offspring, to select two individuals that will replace the two parents in the next generation.













%\paragraph{Roulette wheel selection}
%Also known as fitness proportionate selection, uses the fitness value of each individual to associate a probability of being selected to procreate. The probabilities are calculated to give the most fit individual the largest probability to be selected. We define $\fit_i$ to be the fitness of individual $i$, and the probability for selection is then calculated by $p_i = \frac{\fit_i}{\Phi}$, where $\Phi = \sum_{j=1}^{I} \fit_j$, and $I$ is the total number of individuals~\cite{tang1996genetic, koza1992genetic}. Using this method, the most fit individuals have the highest probability to be selected, while the least fit individuals have the lowest probability.
%
%Imagine a roulette wheel that is spun. The individual that is most fit might cover \perc{38} of the entire wheel, while the rest of the individuals combined have \perc{62} chance. It is obvious that the fittest individual will be selected most often.
%
%\paragraph{Stochastic Universal Sampling}
%
%This policy is similar to roulette wheel selection, with one exception. When individuals are selected for procreation, pointers are used to choose the individuals, instead of randomly choosing an individual. The number of pointers, $P$, is equal to the number of individuals for the next generation, and the pointers are equally spaced. The first pointer is placed at a random position in the range $\left[0, \frac{1}{P}\right]$, and the space between each of the following pointers is equal to $\frac{1}{P}$. Each pointer then points to an individual, and these individuals are selected for procreation, possibly skipping individuals~\cite{baker1987reducing}.
%
%\paragraph{Reward-based selection}
%
%Individuals have an associated reward, which is computed as the sum of the individual's reward and the reward of its parents. If the individual is selected for the next generation, then the individual and its parents receive a reward. The probability for an individual to be selected is proportional to the cumulative reward. There are different functions to calculate a reward~\cite{loshchilov2011not}.
%
%\paragraph{Tournament selection}
%
%As the name suggest, this policy works as any other tournament. It involves running several tournaments, and the winner of each tournament is chosen to procreate. The individuals compete to solve the given problem optimally, and the winner is selected for breeding~\cite{miller1996genetic}.
%

\subsection{Measuring diversity}
In GAs, a high diversity among the individuals is important~\cite{1266373,Zitzler00comparisonof}. It is often argued that the weakness of GAs is the fall in diversity over generations, which can result in premature convergence~\cite{diaz2007empirical}.

Two types of diversity measures are generally used, namely genotypic and phenotypic~\cite{Nguyen:2006:ASPGP,1250187}. \emph{Genotypic diversity} is concerned with how different the encoding of individuals are, while \emph{phenotypic diversity} is concerned with how much the individuals' behavior differ. One drawback of measuring genotypic diversity is that two individuals of different genotypes can still have the same phenotype. For instance, the two different neural networks in \cref{fig:entire-eqnetwork} will always have identical outputs on the same input, even though their genetic makeup is completely different. Thus a high genotypic diversity does not necessarily imply that individuals with many different properties are represented in the population.
%

\begin{figure*}
  \begin{subfigure}{0.5\textwidth}
    \centering
    \includestandalone[mode=buildnew, width=0.6\linewidth]{drawings/eqnetworks/eqnetworks}
    \caption{An artificial neural network with connections and weights.}\label{fig:eqnetwork}
  \end{subfigure}
  \begin{subfigure}{0.5\textwidth}
    \centering
    \includestandalone[mode=buildnew, width=0.6\linewidth]{drawings/eqnetworks/eqnetwork2}
    \caption{An artificial neural network, which is equivalent to \cref{fig:eqnetwork}.}\label{fig:eqnetwork2}
  \end{subfigure}
  \caption{Networks with same phenotype, but different genotypes. The binary representation assumes that each weight is represented by four bits.}\label{fig:entire-eqnetwork}
\end{figure*}

%
When measuring phenotypic diversity, a formula is needed for calculating how different two phenotypes are. Usually, the phenotypic diversity is measured only based on the fitness value of each phenotype~\cite{Nguyen:2006:ASPGP}, which has the advantage of requiring no extra computational power.  We see one drawback to this approach however; Two individuals having the same fitness value might have different ways of achieving these, which will not be reflected by the diversity.

In the following, we propose a method for measuring diversity, which tries to overcome the drawbacks of fitness based diversity measures. We also propose a selection policy that aims to increase this diversity measure.


\section{Neural Network Trait Diversity}
In the following, we will use the term individual and neural network interchangeably, since we represent an individual by a neural network.  Let $F = \set{f_1, f_2, \dots, f_n}$, denote the set of neural networks contained in a population, which all have the same architecture of $a$ input and $b$ output neurons. NNTD is calculated with respect to a number of random inputs $\set{R_1, R_2, \dots, R_m}$, where each $R_i$ for $1 \leq i \leq m$ is an $a$-tuple of real values chosen randomly. For each $R_i$, a Simpson's Diversity Index (SDI) is calculated. SDI is a diversity measure used in ecology to quantify the biodiversity of a habitat. NNTD is calculated as the average SDI for all $R_i$.
To calculate SDI, the total number of neural networks $\lvert F \lvert$ is needed, as well as a distribution of neural networks into a set of species. We classify species of individuals using $S_i(R)$ as the set of neural networks belonging to the $i$th species with respect to input $R$.

We distribute the neural networks into species based on which of their output neurons yield the highest value on input $R$. This means that for any neural network belonging to the species $S_i(R)$, the value of the $i$th output neuron will be greater than or equal to the value of any other output neuron given the input $R$.  This definition implies that the number of species equals the number of output neurons $b$. We distribute neural networks into a set of species as follows
%
\begin{equation*}\label{eq:species}
  S_i(R) = \setof{f_p}{\forall j \in \set{1, 2, \dots, b} \left(\sigma_{Rpi} \geq \sigma_{Rpj}\right)}
\end{equation*}
%
where $\sigma_{xyz} \in \set{0, 1}$ is the output of the $z$th output neuron in neural network $f_y$ on input $x$.

One disadvantage of NNTD is that it relies on random inputs, which means that fewer random inputs implies less statistical significance.

\section{Experiments}
In this section, we evaluate our developed diversity measurement method and comparison our AESP selection policy to crowding and xxx selection policies.

%During each generation, a defined number of crossover and mutation operators are applied.
Each of the experiments are set up to use the same configuration. Populations consist of \num{100} individuals, and upon creating a new population, rank-based selection is utilized on the current population $p$ to select individuals for crossover and mutation in population $p + 1$. Half of this new population is created from crossover between two selected parents in population $p$, the other half created solely by mutation of the individuals also in $p$. Of the individuals in population $p + 1$ belonging to the half created by crossover, there is a \perc{10} chance that the individual also will be selected for mutation. For each bit belonging to the individuals selected for mutation, there is a \perc{5} that bit will be selected for mutation. When a bit is selected for mutation, there is a \perc{50} chance it will stay the same, or be inverted.

% Amount of hidden neurons

\begin{algorithm}
  \caption{Procedure AESP}\label{alg:aesp}
\begin{algorithmic}%[1]
  \Procedure{AESP}{$G_k, O_k$}
  \State $G_{k+1} \gets G_k$
  \ForAll{$o \in O_k$}
    \If {$o.\mathtt{num\_parents} = 1$}
      \If {$\beta(o) > \beta(\pi_1(o))$}
        \State $G_{k+1} \gets (G_{k+1} \setminus \set{\pi_1(o)}) \cup \set{o}$
      \EndIf
    \Else
      \If {$\beta(o) > \mathtt{max}(\beta(\pi_1(o)), \beta(\pi_2(o)))$}
      \State $\mathtt{tmp} \gets (G_{k+1} \setminus \{\pi_1(o), \pi_2(o)\})$ 
        \State $G_{k+1} \gets \mathtt{tmp} \cup \set{o} \cup \set{\rho}$
      \EndIf
    \EndIf
  \EndFor
\EndProcedure
\end{algorithmic}
\end{algorithm}

\subsection{Diversity measurements}
To test our NNTD method, we introduce two variables, both of which only have an effect during the very first generation of a population: initial similarity and initial mutation of a population. 

Initial similarity describes how equal the individuals initially are. It can take on a range between \num{0} and \num{1}, where \num{0} means that all individuals are random, and \num{1} denotes that \perc{100} all individuals have the exact same genotype. With a value of \num{0.5}, one individual is cloned so that half of the population has the same genetic makeup and the other half is initialized to be completely random. A population with the value of \num{1} for initial similarity should have the lowest diversity, and a completely randomly initialize population (initial similarity of \num{0}) should have the highest diversity, since every individual is randomly chosen.

Initial mutation requires an initial similarity set to \num{1}, e.g.\ the initial population consists of a single, cloned individual. This is due to the fact that mutating individuals already randomly initialized will not yield a different diversity of individuals. The initial mutation rate simply signifies the mutation percentage of all bits in every individual of the population. A population consisting of the same cloned individual with an initial mutation rate of \num{1} should have the highest diversity, since every bit in every, initially similar individual, will have a \perc{100} chance to be selected for mutation, resulting in a randomly initialized population.

Experiments on NNTD using both of these two variables are shown in \cref{fig:initial-mutation-similarity}. Each point represents the average diversity measure of \num{100} runs at a variable intervals of \num{0.05}. As we can see by the chart, diversity increases as expected for both variables. Diversity is maximum at a value of \num{7.2} for a completely random population. To reach this maximum, it requires an initial mutation rate of a mere \perc{2} and an initial similarity of \perc{0}.

\begin{figure}[htpb]
  \centering
  \inputresize{drawings/initial-mutation-similarity/graph}
  \caption{Diversity measurements with NNTD, given increased ranges of initial similarity and initial mutation rates. Each point is the average diversity over \num{100} runs.}\label{fig:initial-mutation-similarity}
\end{figure}



\bibliographystyle{abbrv}
\bibliography{bibliography}


% Appendices can be added with \appendix[Proof of Khazens's equation]
\clearpage
\appendix
\section{Artificial Neural Networks and Genetic Algorithms}
\label{ap:ann-ga}

\subsection{Artificial Neural Network}
An artificial neural network is a graph structure that can from the outside be seen as a black box, that given the values $x_1, x_2, \dots, x_n$ outputs the values $y_1, y_2, \dots, y_m$. With the right internal structure, a neural network can be used for a variety of purposes, e.g.\ face recognition, where the intensities of different pixels in an image are used as input, and a single output value $y_1$ is produced, where $y_1 = 1$ if the image was of a face and $0$ if not. We now describe the structure and inner workings of neural networks to understand how these output values are calculated based on input values.

\subsubsection{Neurons}
Nodes in the graph of a neural network are called neurons. A neuron takes a number of values as input from edges exiting other neurons, applies a weight to each value, sums them, then finally applies a function to produce a single output value. The function applied is called the transfer function, and is the same for all hidden neurons in the network. This is recursively expressed by the recurrence
% Note: bias/threshold is not defined below, add if needed
\begin{equation}
  y_i =
  \begin{cases}
    \var{input}_i & \text{if } i \text{ is an input neuron} \\
    f\left(\sum_{j=1}^n w_{ji} y_{j} \right) & \text{else} %- \theta_i \right)
  \end{cases}
\end{equation}
where $\var{input}_i$ is the value given to input neuron $i$, $n$ is the amount of neurons, $y_i$ is the output value of neuron $i$, $w_{ji}$ is the weight of the edge from neuron $j$ to $i$ ($0$ if no connection exists), $y_j$ is the output of neuron $j$, and $f$ is the transfer function, usually defined to be the sigmoid function taking the form
\begin{equation*}
  f(t) = \frac{1}{1+e^{-t}}
\end{equation*}

In a feedforward network, neurons are placed in different layers, where each neuron takes the output of all neurons from the previous layer as input. Neurons in the first layer are called input neurons and receive values as input to the neural network.  The neurons in the last layer are called output neurons. The value output by these neurons becomes the output of the neural network. Any neuron in between these layers is called a hidden neuron. \Cref{fig:ann} illustrates the generic graph structure of a feedforward neural network with a single hidden layer.

\begin{figure}[htpb]
  \centering
  \includestandalone[mode=buildnew, width=\linewidth]{drawings/ANN/ANN}
  \caption{Structure of a neural network.}
  \label{fig:ann}
\end{figure}

\subsubsection{Training a neural network}
Any application of a neural network requires that a correct number of layers, neurons, and weights between neurons are found to adequately solve the problem. In many applications, a single layer of $k$ hidden neurons works well, where $k = 0.5\left(\mid\var{inputNeurons}\mid \times \mid\var{outputNeurons}\mid\right)$\citpls. The weights on edges connecting neurons are decided by a process called training. Well known training algorithms, such as backpropagation, typically require that for each input to the neural network, the output is already known. This kind of learning is called supervised learning. For the application of face recognition, this means that the pictures used for training, each has a predicate indicating whether or not it is a picture of a face.\footnote{Should we explain backpropagation more in detail?}

For some purposes, the desired output of a neural network is not known given a number of input values. Consider for instance a computer game, where a player is controlled by a neural network. The neural network takes as input values indicating the state of the game, such as the position of the players and enemies around him. For each input, the neural network returns a single value indicating which action the player should take, e.g.\ move left, move right, or jump. Given a state of the game, we might not be able to say whether an action output by the neural network is right or wrong. It might be wrong to move closer to an enemy if he eliminates you, but it might be right if the next action is successfully to eliminate the enemy.

From this, it is clear that a backpropagation algorithm is not appropriate for training a neural network to control the actions of a artificial intelligent player in a computer game. For this purpose we propose another approach. We can tell how good a neural network performs, or how fit it is, by simulating a game being played using the neural network to control the player and determining the score achieved in the game, or any other function indicating how well the artificial player performed according to some criteria using that particular neural network. By now being able to measure the fitness of a particular neural network, we can use genetic algorithms to create and search for the best performing neural network.


\subsection{Genetic algorithms} 
Genetic algorithms are optimization algorithms, which imitate the process of natural selection in the search of a global optimum.

\subsubsection{Individuals and chromosomes}
GAs maintain a list of \emph{individuals}, which together form a \emph{population}. Each individual represents a possible solution to the optimization problem in question and has a fitness value, which denotes how adequately the individual can solve the optimization problem. An individual is encoded by its \emph{chromosome}, which is typically represented by a bit string. We refer to the individuals' chromosomes and bit strings, synonymously. Therefore, using a GA requires a way of decoding an individuals chromosome into a solution to the optimization problem.

The population used by a GA typically has a fixed number of individuals, each initialized with a random chromosome when the GA is run. That is, the bit string representing the chromosome is initialized with random bits. As the GA iterates, new individuals are made by combining and modifying chromosomes from existing individuals of the population. In steady state GAs, some of the new individuals will replace older individuals according to some replacement rule. In contrast, a generational GA will choose only from offspring when forming the next generation~\cite{fogarty, Syswerda:1989:UCG:645512.657265, Whitley:1989:GAS:93126.93169}. We will focus on steady state GAs, using the term \emph{generation} \generation{n} to denote the content of the population after $n - 1$ iterations.

\subsubsection{Individuals}
Individuals in GAs have a set of traits and behaviours which define each individual. They can take on any form of data structure, as long as they wholly represent a possible solution to the problem. 

\subsubsection{Neural networks as individuals}
%bitstrings
%weight on connections
%each input and each output
%amount of hidden neurons

Neural networks can be used to solve many types of problems, e.g.\ classification and decision making.
Before we can use neural networks as individuals in a GA, we must define a method for encoding a neural network as a bit string, which will be manipulated by the GA\@. For any GA, we will assume that every individual in its population will have the same architecture. That is, the number of neurons, the size of each layer, and how neurons are connected is the same.

% Uncommented this paragraph friday 2014-3-21, as it just says what a NN is? -Martin
%Each possible input an individual can get will be received through an input neuron. Likewise, each possible action an individual can perform is formulated through the output neurons. The network is constructed with connections between neurons with associated weights. These weights are used to calculate an action given the actual input.

Thus, neural networks differ only in their weights between neurons and the bias of each neuron. Each individual is therefore represented only in terms of the weights and biases. For each GA, any weight and bias is encoded with a fixed number of bits $q$ and $w$, respectively. The bit string is constructed in an ordered manner, such that the first $q$ bits represent the weight of the connection between the first input neuron and the first hidden neuron, the next $q$ bits represent the weight of the connection between the first input neuron and the second hidden neuron, and so forth. \Cref{fig:entire-eqnetwork} shows an example of two neural networks and the bit string that encodes each of them. If biases are used by the GA, these are encoded right after the weights, and ordered such that the first $w$ bits encode the bias of the first neuron, the next $w$ bits encode the bias of the second neuron, and so forth. 
We limit the weights and biases of any neural network to lie in the range $[-5,5]$.
This is because we use the sigmoid activation function, for which the value $f(x)$ changes only a little when $x \leq -5$ and $x  \geq 5$. If a weight or bias of a neural network is encoded by the $u$ bits $b_1 b_2 \cdots b_u$, we decode its real value using the formula
\[
w = (b_12^0 + b_22^1 + \cdots + b_{u-1}2^{u-2})(1-2b_u)\frac{5}{2^u-1}
\]
%\[
%unsigned = b_12^0 + b_22^1 + \cdots + b_{u-1}2^{u-2}\\
%signed = unsigned(1-2b_{u})\\
%weight = 5\frac{signed}{2^u-1}\\
%\]
The first $u-1$ bits represent an increasing sequence of powers of two. The last bit $b_u$ negates the entire value if set (by the factor $(1-2b_u)$). The value is finally normalized to the range $[-5, 5]$ (by the factor $\frac{5}{2^u-1}$).
The number of bits $u$ used to encode a weight or bias is dependent on the problem in question.
If two chromosomes have different bit strings, we say that they have different genotypes. If the neural network they encode produces a different output for some input, we say they have different phenotypes.
% ALso removed this friday 2014-3-21, no need to reference it twice. Possibly if there was an example on how to encode the network in the example?
%An illustration of neural networks is presented in \cref{fig:ann}, and should give a good idea of how the bit string is concatenated.

%\[
%  \underbrace{w_{1,1}}_{n} w_{1,2} \ldots w_{i,j}
%\]

%\[
%  \ldots \underbrace{w_{i,j-1}}_{n} w_{i,j} w_{i,j+1} \ldots
%\]

%where $w_{i,j}$ represents the weight of the connection between the $i'th$ and $j'th$ neuron in bits. Each weight is concatenated with the next weight. The length of any weight is of size $n$.
%

%\subsubsection{Genes}
%Each individual consists of many \emph{genes} as part of its chromosome. Genes constitute the DNA of the individual. These genes are an encoding of some attribute or skill the individual has. Because we defined neural networks to be individuals, the weights and node biases consist of the DNA.

% This subsubsection is already described in individuals and hcromosomes
%\subsubsection{Populations}
%A genetic algorithm manages a collection of many individuals, known as a population. Individuals in the first population are usually initialized randomly with a fixed population size. The goal for these individuals is to solve a problem optimally. Each individual in the population has a fitness level that defines the individual's ability to solve a given problem.
%Initially, there are no generations. Creating a new population from a previous population increases the amount of generations by one, hopefully yielding a net increase in average fitness.

\subsubsection{Crossovers and mutations}
In natural evolution, a pair of individuals come together to produce one or more new children, each having genes from both of its parents. In GAs, the process of procreation is done by performing a \emph{crossover} of the two parent individuals' chromosomes. Parts of each parent's bit strings are used to create the child individual.

\emph{Mutations} can also occur randomly at any point in time upon creating a child individual. If genes are encoded as bit strings, then a mutation arbitrarily toggles one or more bits. This ensures that new genes, not previously present in the population, can be formed.

\subsubsection{Fitness functions}
A \emph{fitness function} must be defined to calculate the desirability for each individual. This function is used to define the most fit individuals in a population. By giving more fit individuals a greater chance of reproducing, the intuition is that more fit individuals will be created, having the best traits from each of their parents.




\end{document}
