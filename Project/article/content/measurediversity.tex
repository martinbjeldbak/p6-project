\subsection{Measuring diversity}
In GAs, a high diversity among the individuals is important. It is often argued that the weakness of GAs is the fall in diversity over generations, which can result in premature convergence~\cite{diaz2007empirical, 1266373,Zitzler00comparisonof}.

\subsubsection{Genotypic diversity measures}
The genotypic diversity of a set of individuals is determined by how different their bit string are.
It is only necessary to have a method for measuring the difference (distance) between any two bit strings.
The diversity between a set of bit strings can then be expressed as the average distance between any two bit strings.
Summation can also be used instead of averaging\cite{1250187}, which is merely a convenient optimization.

The \emph{Hamming distance} between two bit strings $A$ and $B$ of equal lengths, is the number of indexes $i$, such that $A[i] \neq B[i]$\cite{1250187}.

The \emph{Levenshtein distance} between two bit strings $A$ and $B$, is the number of bits that must be inserted, deleted or substituted to change $A$ into $B$\cite{1250187}.

The Hamming distance between the two strings

\begin{align}
    01010101 \\
    10101010
\end{align}

is 8. The Levenshtein distance is 2, because we can transform (1) into (2) by deleting the first bit and prepend a $0$.

\subsubsection{Phenotypic diversity measures}
\emph{Phenotypic diversity} is concerned with the individuals' behavioural differences, and can be calculated based on their fitness values\cite{1250187}.

\emph{Phenotypes} are groupings of individual having the same fitness values. 
The number of phenotypes can be used as a diversity measure, as well as the \emph{Standar deviation} expressed as

\[\sqrt{\frac{\sum_{i=1}^N{(f_i-\bar{f})^2}}{N-1}}\]

where $N$ is the population size and $f_i$ is the fitness of the $i$th individual\cite{1250187}.

One advantage of fitness based diversity measures is that no extra computations are associated with calculating the fitness values , since they are already calculated by the GA to give the more fit individuals a higher chance to produce offspring.\cite{Nguyen:2006:ASPGP}.

\subsubsection{Other measurements}
Some diversity measures exists that are neither genotypic nor phenotypic. For instance the \emph{Ancestral ID}\cite{1250187} method, which assigns a unique ID to each individual in the initial population.
Every mutated individual receives a new unique ID while every child gets the ID of one of its parents.
The diversity is then based on the uniqueness of IDs in a population.

\subsubsection{Our proposal}
We think it is essential that a diversity measure reflects the difference in traits among individuals.
Recall that we are only concerned with neural networks as individual. 
Since \emph{traits} is a rather vague term, we introduce a clear definition of traits among neural networks.

\emph{Two neural networks have different traits, if they for some input produce different outputs}

By trait diversity, we denote the differences in traits among neural networks.

A fitness based diversity measure does not catch the trait diversity. Consider two artificial intelligent (AI) players for the cell phone game ``Snake''. Their fitness can be calculated based on how much food they collect before they die. One AI player may have traits that makes it good at avoiding death by not hitting any walls or its own body. Sometimes, by chance, it hits a piece of food. The other AI player may have traits that makes it good at searching for food. The two AI players can have the same fitness value, because they collect the same amount of food before they die, and still have completely different traits.
Genotypic diversity measures does not catch the trait diversity either. This is because two individuals of different genotypes can yield the same output on any input. An example is the two neural networks in \cref{fig:entire-eqnetwork}. No matter what input they receive, their output will always be the same. They are genotypic diverse, but not trait diverse.

%
\begin{figure*}
  \begin{subfigure}{0.5\textwidth}
    \centering
    \inputresizeto{0.5\linewidth}{drawings/eqnetworks/eqnetworks3}
    \caption{An artificial neural network with connections and weights.}\label{fig:eqnetwork}
  \end{subfigure}
  \begin{subfigure}{0.5\textwidth}
    \centering
    \inputresizeto{0.5\linewidth}{drawings/eqnetworks/eqnetworks4}
    \caption{An artificial neural network equivalent to \cref{fig:eqnetwork}.}\label{fig:eqnetwork2}
  \end{subfigure}
  \caption{Networks with same phenotype, but different genotypes. The binary representation assumes that each weight is represented by four bits.}\label{fig:entire-eqnetwork}
\end{figure*}

%

To the best of our knowledge, no diversity measure exists that catches our definition of trait diversity.
In the following, we propose a method for measuring trait diversity which we call \emph{Neural Network Trait Diversity} (NNTD). 
NNTD aims to reflect the diversity of different traits among individuals. 