\subsection{Measuring diversity}
In GAs, a high diversity among the individuals is important. It is often argued that the weakness of GAs is the fall in diversity over generations, which can result in premature convergence~\cite{diaz2007empirical, 1266373,Zitzler00comparisonof}.

Two types of diversity measures are widely used, namely genotypic and phenotypic~\cite{Nguyen:2006:ASPGP,1250187}.
\emph{Genotypic diversity} is concerned with the differences in the individuals' encoding, while \emph{phenotypic diversity} is concerned with the individuals' behavioural differences, which is measured based on fitness values.
Some diversity measures exists that are neither genotypic or phenotypic. For instance the \emph{Ancestral id}\cite{INSERT CITATION}%http://citeseerx.ist.psu.edu/viewdoc/download?doi=10.1.1.124.1687&rep=rep1&type=pdf
, which identifies similar individuals by looking at their ancestors. We will turn our look to only genotypic and phenotypic diversity measures, as these methods are widely used.


\subsubsection{Genotypic diversity measures}
\todo{Explain different genotypic diversity measures (hamming distance, levenshtein)}
%THIS ARTICLE IS USEFUL HERE http://citeseerx.ist.psu.edu/viewdoc/download?doi=10.1.1.124.1687&rep=rep1&type=pdf

\subsubsection{Phenotypic diversity measures}
\todo{Explain different fitness based diversity measures (unique fitness values, standard deviation)}
%THIS ARTICLE IS USEFUL HERE http://citeseerx.ist.psu.edu/viewdoc/download?doi=10.1.1.124.1687&rep=rep1&type=pdf
A fitness based diversity measure has the advantage that fitness values are calculated for each individual created by a GA, thus it is associated with only limited additional computations\cite{Nguyen:2006:ASPGP}.

\subsubsection{Our proposal}
We think it is essential that a diversity measure reflects the difference in traits among individuals. Recall that we are only concerned with neural networks as individual. Since \emph{traits} is a rather vague term, we introduce a clear definition of traits among neural networks.

\emph{Two neural networks have different traits, if they for some input produce different outputs}

By trait diversity, we denote the differences in traits among neural networks.

A fitness based diversity measure does not catch the trait diversity. Consider two artificial intelligent (AI) players for the cell phone game ``Snake''. Their fitness can be calculated based on how much food they collect before they die. One AI player may have traits that makes it good at avoiding death by not hitting any walls or its own body. Sometimes, by chance, it hits a piece of food. The other AI player may have traits that makes it good at searching for food. The two AI players can have the same fitness value, because they collect the same amount of food before they die, and still have completely different traits.
Genotypic diversity measures does not catch the trait diversity either. This is becayse two individuals of different genotypes can yield the same output on any input. An example is the two neural networks in \cref{fig:entire-eqnetwork}. No matter what input they receive, their output will always be the same. They are genotypic diverse, but not trait diverse.

%
\begin{figure*}
  \begin{subfigure}{0.5\textwidth}
    \centering
    \inputresizeto{0.5\linewidth}{drawings/eqnetworks/eqnetworks3}
    \caption{An artificial neural network with connections and weights.}\label{fig:eqnetwork}
  \end{subfigure}
  \begin{subfigure}{0.5\textwidth}
    \centering
    \inputresizeto{0.5\linewidth}{drawings/eqnetworks/eqnetworks4}
    \caption{An artificial neural network equivalent to \cref{fig:eqnetwork}.}\label{fig:eqnetwork2}
  \end{subfigure}
  \caption{Networks with same phenotype, but different genotypes. The binary representation assumes that each weight is represented by four bits.}\label{fig:entire-eqnetwork}
\end{figure*}

%

To the best of our knowledge, no diversity measure exists that catches our definition of trait diversity.
In the following, we propose a method for measuring trait diversity which we call \emph{Neural Network Trait Diversity} (NNTD). 
NNTD aims to reflect the diversity of different traits among individuals. 