\subsection{Measuring diversity}
In GAs, a high diversity among the individuals is important. It is often argued that the weakness of GAs is the fall in diversity over generations, which causes the GA to do a simple local hill climbing \citpls{}.

Two types of diversity measures are generally used, namely genotypic and phenotypi\cite{Nguyen:2006:ASPGP}. Genotypic diversity is concerned with how different the encoding of individuals are, while phenotypic diversity is concerned with how much the individuals' behavior differ. One drawback of measuring genotypic diversity is that two individuals of different genotypes can still have the same phenotype. For instance, the two different neural networks in \cref{fig:entire-eqnetwork} will always have identical outputs on the same input. Thus a high genotypic diversity does not necessarily imply that individuals with many different properties are represented in the population.
%
\begin{figure*}
  \begin{subfigure}{0.5\textwidth}
    \centering
    \inputresizeto{0.5\linewidth}{drawings/eqnetworks/eqnetworks3}
    \caption{An artificial neural network with connections and weights.}\label{fig:eqnetwork}
  \end{subfigure}
  \begin{subfigure}{0.5\textwidth}
    \centering
    \inputresizeto{0.5\linewidth}{drawings/eqnetworks/eqnetworks4}
    \caption{An artificial neural network equivalent to \cref{fig:eqnetwork}.}\label{fig:eqnetwork2}
  \end{subfigure}
  \caption{Networks with same phenotype, but different genotypes. The binary representation assumes that each weight is represented by four bits.}\label{fig:entire-eqnetwork}
\end{figure*}

%
When measuring phenotypic diversity, a formula is needed for calculating how different two phenotypes are.
Usually, the phenotypic diversity is measured only based on the fitness value of each phenotype~\cite{Nguyen:2006:ASPGP}, which has the advantage of requiring no extra computational power. 
We see one drawback to this approach however; Two individuals having the same fitness value might have different ways of achieve these, which will not be reflected by the diversity. 

In the following, we propose a method for measuring diversity, which tries to overcome the drawbacks of fitness based diversity measures.
We also propose a selection policy that aims to increase this diversity measure.
