\subsection{Diversity measures}
\label{sec:diversitymeasures}
It is often argued that the weakness of GAs is the fall in diversity over generations, which often results in premature convergence~\cite{diaz2007empirical, 1266373,Zitzler00comparisonof}.

Here, we summarise key points regarding well-known diversity measures. %Afterwards, we present our own proposal for a diversity measure (\dia), and we clearly define what diversity means with this method.

\subsubsection{Genotypic diversity measures}
The genotypic diversity of a set of individuals is determined by how different their genetic structures are. To measure this type of diversity, methods to compute the distance between any two individuals' encoded bit strings are required.

The diversity between a set of bit strings can then be expressed as the average distance between any two bit strings. Summation can also be used instead of averaging, which is merely a convenient optimization.

The \emph{Hamming distance} between two bit strings $A$ and $B$ of equal length is the number of indexes $i$, such that $A[i] \neq B[i]$.

Another way to measure genotypic diversity, is by using the \emph{Levenshtein distance} between two bit strings $A$ and $B$. It is then the number of bits that must be inserted, deleted or substituted to change $A$ into $B$. For example, the Hamming distance distance between between the bit strings
%
\begin{align}
&01010101\label{eq:bit1} \\
&10101010\label{eq:bit2}
\end{align}
%
is 8 when using Hamming distance, and 2 if using the Levenshtein distance, because transforming \cref{eq:bit1} into \cref{eq:bit2} is done by deleting the first bit and prepending a $0$~\cite{1250187}, totalling 2 operations. Complexity of computing the Levenshtein distance between two individuals is $\bigO{\bitstringl^2}$ and $\bigO{\bitstringl}$ for Hamming distance, where \bitstringl{} is the length of each neural network's bit string.

At first glance, these complexities aren't worrying. But in a GA with a population size of \indsetl{} individuals, where \indset{} is the set of all neural networks (the population), computing the average Hamming and Levenshtein distances between all individuals in the population takes $\bigO{\indsetl^2 \bitstringl}$ and $\bigO{\indsetl^2 \bitstringl^2}$ time, respectively. 

As concluded by~\cite{Darwen00doesextra}, constant high genetic diversity does not guarantee better solutions, requiring other diversity measures to also be explored.

\subsubsection{Phenotypic diversity measures}
\emph{Phenotypic diversity} is concerned with the individuals' behavioural differences, and can be calculated based on their fitness values. Such diversity measures include computing the standard deviation of fitness values, the average number of unique fitness values in a population, and entropy-based methods (see~\cite{1250187, 1266373}).

One advantage of fitness-based diversity measures is that no extra computations are associated with calculating the diversity, because the fitness values already have been calculated by the GA to access how fit each individual is~\cite{Nguyen:2006:ASPGP}.

\subsubsection{Other measurements}
Some diversity measures exist that are neither genotypic nor phenotypic. For instance the \emph{Ancestral ID} method, which assigns a unique ID to each individual in the initial population. Every mutated individual receives a new unique ID while every child gets the ID of one of its parents. The diversity is then based on the uniqueness of IDs in a population~\cite{1250187}.

