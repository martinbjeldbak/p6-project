\subsection{Measuring diversity}
In GAs, a high diversity among the individuals is important. It is often argued that the weakness of GAs is the fall in diversity over generations, which can result in premature convergence~\cite{diaz2007empirical, 1266373,Zitzler00comparisonof}.

Two types of diversity measures are generally used, namely genotypic and phenotypic~\cite{Nguyen:2006:ASPGP,1250187}. The former concerns different ways individuals are represented, while the latter is concerned with behavioural differences. \emph{Genotypic diversity} is concerned with the difference of the individuals' encoding, while \emph{phenotypic diversity} is concerned with the individuals' behavioral differences (which we measure as the fitness).

One drawback of measuring genotypic diversity is that two individuals of different genotypes can still have the same phenotype. For instance, the two different neural networks in \cref{fig:entire-eqnetwork} will have identical outputs on the same input, even though their genetic makeup is different. As an example, consider two artificial intelligent (AI) players for the cell phone game ``Snake''. Their fitness can be calculated based on how much food they collect before they die. One AI player may have traits that makes it good at avoiding death by not hitting any walls or its own body. Sometimes, by chance, it hits a piece of food. The other AI player may have traits that makes it good at searching for food. The two AI players can have the same fitness value, because they collect the same amount of food before they die, and still have completely different traits. Therefore, fitness-based diversity measures do not always catch the difference in traits among individuals.

%
\begin{figure*}
  \begin{subfigure}{0.5\textwidth}
    \centering
    \inputresizeto{0.5\linewidth}{drawings/eqnetworks/eqnetworks3}
    \caption{An artificial neural network with connections and weights.}\label{fig:eqnetwork}
  \end{subfigure}
  \begin{subfigure}{0.5\textwidth}
    \centering
    \inputresizeto{0.5\linewidth}{drawings/eqnetworks/eqnetworks4}
    \caption{An artificial neural network equivalent to \cref{fig:eqnetwork}.}\label{fig:eqnetwork2}
  \end{subfigure}
  \caption{Networks with same phenotype, but different genotypes. The binary representation assumes that each weight is represented by four bits.}\label{fig:entire-eqnetwork}
\end{figure*}

%

When measuring phenotypic diversity, a formula is needed for calculating how different two phenotypes are. Usually, the phenotypic diversity is measured only based on the fitness value of each phenotype~\cite{Nguyen:2006:ASPGP}, which has the advantage of requiring no extra computational power.  We see one drawback to this approach however; two individuals with the same fitness value might have different ways of achieving these, which will not be reflected by the diversity.

In the following, we propose a method for measuring diversity, which tries to overcome the drawbacks of fitness based diversity measures. We also propose a selection policy that aims to increase this diversity measure.
