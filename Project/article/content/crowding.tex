\subsubsection{Crowding}
\label{sec:crowding} 
A simple way to compose the $i$'th generation, denoted $G_i$, is by selecting the $n$ best individuals from the union of $G_{i-1}$ and $O_{i-1}$, where $O_{i-1}$ is the offspring produced from $G_{i-1}$ by selecting a number of individuals, where more fit individuals are more likely to be selected.
It is easy to see that this method causes a low diversity over time - since the best individuals are more likely to be selected for procreation, much of the offspring will have similar properties that yields a high fitness value and thus be a part of the next generation just like their parents. Now the chance is even greater that the same properties will be spread out even further in the next generation.

Many methods have been proposed to overcome the problem that diversity decreases through generations - these methods include inserting new random individuals into the population (called random immigrants), removing a number of individuals from the population, using complex population structures to lower the gene flow, and the use of special selection procedures\cite{ursem2002diversity}. The latter is known as crowding. 
An example of crowding is to dictate that a child only can be added to the population if one of its parents is removed.
Crowding has been found to work well on test functions as well as other applications\cite{crowding}. 

\subsubsection{Replacement rules}
A \emph{replacement rule} can be used when two or more individuals are very similar, and one wishes to replace them all by just a single individual. This will decrease the similarity of individuals and thus increase diversity.

An example is the probabilistic crowding replacement rule\cite{Mengshoel_and_Goldberg:1999}

\[p_x = \frac{f(x)}{f(x)+f(y)}\]

where $p_x$ denotes the chance that individual $x$ replaces the similar individuals $\{x, y\}$, $f(x)$ is the fitness of $x$, and $f(y)$ is the fitness value of $y$. This rule can easily be generalized for larger family size. 
In an attempt to keep a low diversity, the rule can be applied between a two parents and their offspring, to select two individuals that will replace the two parents in the next generation.













%\paragraph{Roulette wheel selection}
%Also known as fitness proportionate selection, uses the fitness value of each individual to associate a probability of being selected to procreate. The probabilities are calculated to give the most fit individual the largest probability to be selected. We define $\fit_i$ to be the fitness of individual $i$, and the probability for selection is then calculated by $p_i = \frac{\fit_i}{\Phi}$, where $\Phi = \sum_{j=1}^{I} \fit_j$, and $I$ is the total number of individuals~\cite{tang1996genetic, koza1992genetic}. Using this method, the most fit individuals have the highest probability to be selected, while the least fit individuals have the lowest probability.
%
%Imagine a roulette wheel that is spun. The individual that is most fit might cover \perc{38} of the entire wheel, while the rest of the individuals combined have \perc{62} chance. It is obvious that the fittest individual will be selected most often.
%
%\paragraph{Stochastic Universal Sampling}
%
%This policy is similar to roulette wheel selection, with one exception. When individuals are selected for procreation, pointers are used to choose the individuals, instead of randomly choosing an individual. The number of pointers, $P$, is equal to the number of individuals for the next generation, and the pointers are equally spaced. The first pointer is placed at a random position in the range $\left[0, \frac{1}{P}\right]$, and the space between each of the following pointers is equal to $\frac{1}{P}$. Each pointer then points to an individual, and these individuals are selected for procreation, possibly skipping individuals~\cite{baker1987reducing}.
%
%\paragraph{Reward-based selection}
%
%Individuals have an associated reward, which is computed as the sum of the individual's reward and the reward of its parents. If the individual is selected for the next generation, then the individual and its parents receive a reward. The probability for an individual to be selected is proportional to the cumulative reward. There are different functions to calculate a reward~\cite{loshchilov2011not}.
%
%\paragraph{Tournament selection}
%
%As the name suggest, this policy works as any other tournament. It involves running several tournaments, and the winner of each tournament is chosen to procreate. The individuals compete to solve the given problem optimally, and the winner is selected for breeding~\cite{miller1996genetic}.
%
