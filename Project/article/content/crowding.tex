\subsubsection{Crowding}
Many methods have been proposed to overcome the problem of gradually decreasing diversity through generations. These methods include inserting random immigrants, which are new, randomly initialized individuals~\cite{Cobb93geneticalgorithms}, using complex population structures to lower the gene flow, and the use of special selection procedures~\cite{ursem2002diversity}. The latter is known as crowding.  How crowding is performed can be described formally by a replacement rule.

\subsubsection{Replacement rules}
A \emph{replacement rule} determines how a number of competing individuals are replaced by only a subset of themselves, e.g.\ when the individuals of the $i$th generation $G_i$, and their offspring $O_i$, compete to be selected for the next generation $G_{i+1}$. A commonly used replacement rule is to make $G_i$ contain the $n$ most fit individuals from $G_{i-1} \cup O_{i-1}$, where $\mid G_x\mid = n$ for all $x$, see~\cite{masterThesisGANN}. We will refer to this as the \emph{naive replacement rule}. It is easy to see that this method causes a low diversity over time. Since the best individuals are more likely to be selected for procreation, much of the offspring will have similar traits that yield a high fitness value, and thus be a part of the next generation, just like their parents. Naturally, this increases the chance of the same traits being carried on into the next generation.

Another well known replacement rule is the probabilistic crowding algorithm~\cite{Mengshoel_and_Goldberg:1999}
%
\[
  p_x = \frac{f(x)}{f(x)+f(y)}
\]
%
where $p_x$ denotes the chance that individual $x$ replaces individuals $\set{x, y}$, and $f(i)$ is the fitness of individual $i$. This rule can easily be generalized for larger family sizes. This replacement rule favours the most fit individuals to be selected.

A replacement rule can favour more diverse individuals, for instance by dictating that for any $a$ individuals having produced $b$ offspring individuals, we only select $a$ individuals of these similar $a + b$ individuals for the next generation, instead of selecting all $a + b$ individuals.

%\paragraph{Roulette wheel selection}
%Also known as fitness proportionate selection, uses the fitness value of each individual to associate a probability of being selected to procreate. The probabilities are calculated to give the most fit individual the largest probability to be selected. We define $\fit_i$ to be the fitness of individual $i$, and the probability for selection is then calculated by $p_i = \frac{\fit_i}{\Phi}$, where $\Phi = \sum_{j=1}^{I} \fit_j$, and $I$ is the total number of individuals~\cite{tang1996genetic, koza1992genetic}. Using this method, the most fit individuals have the highest probability to be selected, while the least fit individuals have the lowest probability.
%
%Imagine a roulette wheel that is spun. The individual that is most fit might cover \perc{38} of the entire wheel, while the rest of the individuals combined have \perc{62} chance. It is obvious that the fittest individual will be selected most often.
%
%\paragraph{Stochastic Universal Sampling}
%
%This policy is similar to roulette wheel selection, with one exception. When individuals are selected for procreation, pointers are used to choose the individuals, instead of randomly choosing an individual. The number of pointers, $P$, is equal to the number of individuals for the next generation, and the pointers are equally spaced. The first pointer is placed at a random position in the range $\left[0, \frac{1}{P}\right]$, and the space between each of the following pointers is equal to $\frac{1}{P}$. Each pointer then points to an individual, and these individuals are selected for procreation, possibly skipping individuals~\cite{baker1987reducing}.
%
%\paragraph{Reward-based selection}
%
%Individuals have an associated reward, which is computed as the sum of the individual's reward and the reward of its parents. If the individual is selected for the next generation, then the individual and its parents receive a reward. The probability for an individual to be selected is proportional to the cumulative reward. There are different functions to calculate a reward~\cite{loshchilov2011not}.
%
%\paragraph{Tournament selection}
%
%As the name suggest, this policy works as any other tournament. It involves running several tournaments, and the winner of each tournament is chosen to procreate. The individuals compete to solve the given problem optimally, and the winner is selected for breeding~\cite{miller1996genetic}.
%
