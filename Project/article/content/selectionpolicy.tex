\subsection{Selection policy}
Offspring are produced in each iteration of a GA from a generation of individuals, which, combined with the existing individuals must form the next generation according to a selection policy.

In some implementations, only the offspring is used for the next generation and all individuals of previous generations are discarded\footnote{ Give an example of such a selection policy}. Common implementations form the next generation by selecting the best individuals from both the existing population and the offspring created\cite{masterThesisGANN}. Such a policy leads to homogeneous individuals and thus a low diversity.

We propose a computationally inexpensive selection policy that forces a higher diversity, we call this selection policy \emph{Ancestor Elitism Selection Policy} (AESP).

Given a generation of individuals $G_k$ and their offspring $O_k$, AESP puts together the next generation, $G_{k+1}$ as follows:
%
\begin{algorithm}
  \caption{Procedure AESP}\label{alg:aesp}
\begin{algorithmic}[1]
  \Procedure{AESP}{$G_k, O_k$}
  \State $G_{k+1} \gets G_k$
  \ForAll{individual $o \in O_k$}
    \If {$o.\mathtt{num\_parents} = 1$}
      \If {$\beta(o) > \beta(\pi_1(o))$}
        \State $G_{k+1} \gets (G_{k+1} \setminus \set{\pi_1(o)}) \cup \set{o}$
      \EndIf
    \Else
      \If {$\beta(o) > \mathtt{max}(\beta(\pi_1(o)), \beta(\pi_2(o)))$}
      \State $\mathtt{tmp} \gets (G_{k+1} \setminus \{\pi_1(o), \pi_2(o)\})$ 
        \State $G_{k+1} \gets \mathtt{tmp} \cup \set{o} \cup \set{\rho}$
      \EndIf
    \EndIf
  \EndFor
\EndProcedure
\end{algorithmic}
\end{algorithm}
%
In the algorithm, $\beta(o)$ denotes the fitness of the individual $o$, $\pi_i(o)$ denotes the $i$'th parent of $o$, and $\rho$ denotes a random immigrant, which is a newly created individual with a random chromosome.
