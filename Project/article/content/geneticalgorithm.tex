\subsection{Genetic algorithms} 
Genetic algorithms are optimization algorithms which imitate the process of natural selection in the search of a global optimum.

\subsubsection{Individuals and chromosomes}
Genetic algorithms (GA) maintain a list of \emph{individuals}, which together form a \emph{population}. Each individual represents a possible solution to the optimization problem in question and has a fitness value, which denotes how adequately the individual can solve the optimization problem. An individual is encoded by its \emph{chromosome}, which is typically represented by a bit string. We refer to encoded individuals chromosomes and bit strings synonymously. Therefore, using a GA requires a way of decoding an individuals chromosome into a solution to the optimization problem.

The population used by a GA typically has a fixed number of individuals, each initialized with a random chromosome when the GA is run. That is, the bit string representing the chromosome is initialized with random bits. As the GA iterates, new individuals are made by combining and modifying chromosomes from existing individuals of the population. In steady state GAs, some of the new individuals will replace older individuals according to some replacement rule. In contrast, a generational GA will choose only from offspring when forming the next generation~\cite{fogarty, Syswerda:1989:UCG:645512.657265, Whitley:1989:GAS:93126.93169}. We will focus on steady state GAs, using the term \emph{generation} \generation{n} to denote the content of the population after $n - 1$ iterations.

\subsubsection{Individuals}
Individuals in GAs have a set of traits and behaviours which define each individual. They can take on any form of data structure, as long as they wholly represent a possible solution to the problem. 

\subsubsection{Neural networks as individuals}
%bitstrings
%weight on connections
%each input and each output
%amount of hidden neurons

As we have discussed, neural networks are ideal for decision making and hence are appropriate as individuals in a GA. To support various GA operators, individuals are encoded as bit strings.

% Uncommented this paragraph friday 2014-3-21, as it just says what a NN is? -Martin
%Each possible input an individual can get will be received through an input neuron. Likewise, each possible action an individual can perform is formulated through the output neurons. The network is constructed with connections between neurons with associated weights. These weights are used to calculate an action given the actual input.

When using a GA, all individuals are neural networks with the same architecture.
That is, the neural networks differ only in their weights between neurons and the bias of each neuron.
Each individual is therefore represented only in terms of the weights and biases.
For each GA, any weight and bias is encoded with a fixed number of bits $n$ and $m$, respectively.
The bit string is constructed in an ordered manner, such that the first $n$ bits represent the weight for the first connection between the first input neuron and the first hidden neuron, the next $n$ bits represent the weight for the connection between the first input neuron to the second hidden neuron, and so forth. \cref{fig:entire-eqnetwork} shows an example of two neural networks and the bit string that encodes each of them.
If biases are used by the GA, these comes after all the encoded weights.
They are ordered such that the first $m$ bits encodes the bias of the first neuron, the next $m$ bits encodes the bias of the second neuron, and so forth.
In this way, we encode a neural network, which is manipulated by different GA operators.
From now on, we will refer to the bit string encoding of a neural network as the individual's \emph{chromosome}.

% ALso removed this friday 2014-3-21, no need to reference it twice. Possibly if there was an example on how to encode the network in the example?
%An illustration of neural networks is presented in \cref{fig:ann}, and should give a good idea of how the bit string is concatenated.

%\[
%  \underbrace{w_{1,1}}_{n} w_{1,2} \ldots w_{i,j}
%\]

%\[
%  \ldots \underbrace{w_{i,j-1}}_{n} w_{i,j} w_{i,j+1} \ldots
%\]

%where $w_{i,j}$ represents the weight of the connection between the $i'th$ and $j'th$ neuron in bits. Each weight is concatenated with the next weight. The length of any weight is of size $n$.
%
If two chromosomes have different bit strings, we say that they have different genotypes. If the neural network they encode produces a different output for some input, we say they have different phenotypes.

%\subsubsection{Genes}
%Each individual consists of many \emph{genes} as part of its chromosome. Genes constitute the DNA of the individual. These genes are an encoding of some attribute or skill the individual has. Because we defined neural networks to be individuals, the weights and node biases consist of the DNA.

% This subsubsection is already described in individuals and hcromosomes
%\subsubsection{Populations}
%A genetic algorithm manages a collection of many individuals, known as a population. Individuals in the first population are usually initialized randomly with a fixed population size. The goal for these individuals is to solve a problem optimally. Each individual in the population has a fitness level that defines the individual's ability to solve a given problem.
%Initially, there are no generations. Creating a new population from a previous population increases the amount of generations by one, hopefully yielding a net increase in average fitness.

\subsubsection{Crossovers and mutations}
In natural evolution, a pair of individuals come together to produce one or more new children, each having genes from both of its parents. In GAs, the process of procreation is done by performing a \emph{crossover} of the two parent individuals' chromosomes. Parts of each parent's bit strings are used to create the child individual.

\emph{Mutations} can also occur randomly at any point in time upon creating a child individual. If genes are encoded as bit strings, then a mutation arbitrarily toggles one or more bits. This ensures that new genes not previously present in the population can be made.

\subsubsection{Fitness functions}
A \emph{fitness function} must be defined to calculate the desirability for each individual. This function is used to define the most fit individuals in a population. By giving more fit individuals a greater chance of reproducing, the intuition is that more fit individuals will be created, having the best traits from each of their parents.

