\subsection{Genetic algorithms} 
Genetic algorithms are optimization algorithms which imitate the process of natural selection in search of global maxima.

\subsubsection{Individuals and chromosomes}
Genetic algorithms (GA) maintain a list of \emph{individuals}, which together form a \emph{population}. Each individual represents a possible solution to the optimization problem and has a fitness value, which denotes how adequately the individual can solve the optimization problem. An individual is encoded by its \emph{chromosome}, which is typically represented by a bit string. Therefore, using a GA requires a way of decoding an individuals chromosome into a solution to the optimization problem.

The population used by a GA typically has a fixed number of individuals, each initialized with a random chromosome when the GA is run. That is, the bit string representing the chromosome is initialized with random bits. As the GA iterates, new individuals are made by combining and modifying chromosomes from existing individuals of the population. Over time, more fit individuals will replace the less fit individuals, using a replace policy that aims to maximize the fitness of the entire population. After each iteration of the GA, where a new population is formed, we say that we have another \emph{generation} of individuals.

\subsubsection{Individuals}
Individuals in GAs have a set of traits and behaviors which define each individual. They can take on any form of data structure, as long as they wholly represent a possible solution to the problem. 

\subsubsection{Neural networks as individuals}
%bitstrings
%weight on connections
%each input and each output
%amount of hidden neurons

As we have discussed, neural networks are ideal for decision making and hence are appropriate as individuals in a GA. To support various GA operators, individuals are encoded as bit strings.

% Uncommented this paragraph friday 2014-3-21, as it just says what a NN is? -Martin
%Each possible input an individual can get will be received through an input neuron. Likewise, each possible action an individual can perform is formulated through the output neurons. The network is constructed with connections between neurons with associated weights. These weights are used to calculate an action given the actual input.

When using a GA, all individuals are neural networks with the same architecture.
That is, the neural networks differ only in their weights between neurons and the bias of each neuron.
Each individual is therefore represented only in terms of the weights and biases.
For each GA, any weight and bias is encoded with a fixed number of bits $n$ and $m$, respectively.
The bit string is constructed in an ordered manner, such that the first $n$ bits represent the weight for the first connection between the first input neuron and the first hidden neuron, the next $n$ bits represent the weight for the connection between the first input neuron to the second hidden neuron, and so forth. \cref{fig:entire-eqnetwork} shows an example of two neural networks and the bit string that encodes each of them.
If biases are used by the GA, these comes after all the encoded weights.
They are ordered such that the first $m$ bits encodes the bias of the first neuron, the next $m$ bits encodes the bias of the second neuron, and so forth.
In this way, we encode a neural network, which is manipulated by different GA operators.
From now on, we will refer to the bit string encoding of a neural network as the individual's \emph{chromosome}.

% ALso removed this friday 2014-3-21, no need to reference it twice. Possibly if there was an example on how to encode the network in the example?
%An illustration of neural networks is presented in \cref{fig:ann}, and should give a good idea of how the bit string is concatenated.

%\[
%  \underbrace{w_{1,1}}_{n} w_{1,2} \ldots w_{i,j}
%\]

%\[
%  \ldots \underbrace{w_{i,j-1}}_{n} w_{i,j} w_{i,j+1} \ldots
%\]

%where $w_{i,j}$ represents the weight of the connection between the $i'th$ and $j'th$ neuron in bits. Each weight is concatenated with the next weight. The length of any weight is of size $n$.
%
If two chromosomes have different bit strings, we say that they have different genotypes. If the neural network they encode produces a different output for some input, we say they have different phenotypes.

%\subsubsection{Genes}
%Each individual consists of many \emph{genes} as part of its chromosome. Genes constitute the DNA of the individual. These genes are an encoding of some attribute or skill the individual has. Because we defined neural networks to be individuals, the weights and node biases consist of the DNA.

% This subsubsection is already described in individuals and hcromosomes
%\subsubsection{Populations}
%A genetic algorithm manages a collection of many individuals, known as a population. Individuals in the first population are usually initialized randomly with a fixed population size. The goal for these individuals is to solve a problem optimally. Each individual in the population has a fitness level that defines the individual's ability to solve a given problem.
%Initially, there are no generations. Creating a new population from a previous population increases the amount of generations by one, hopefully yielding a net increase in average fitness.

\subsubsection{Crossovers and mutations}
In natural evolution, a pair of individuals come together to produce one or more new child individuals, with genes from both of the parent individuals. The process of procreation is done by performing a \emph{crossover} of the two parent individuals' chromosomes. Parts of each parent's bit strings are used to create the child individual.

\emph{Mutations} can also occur randomly at any point in time upon creating a child individual. If genes are encoded as bit strings, then a mutation arbitrarily toggles one of the bits. This ensures that the population can evolve if no progress would be made if the chromosomes did not allow it.

\subsubsection{Fitness functions}
A \emph{fitness function} must be defined to calculate the desirability for each individual. This function is used to define the most fit individuals in a population. The higher the fitness level is of a given individual, the higher the chance it has to be chosen to reproduce with another individual. The intuition behind this is that choosing two fit individuals to crossover will create an even better individual with the best traits of each of its parents.

\subsubsection{Selection policies}
\label{sec:selectionpolicies}

Many selection policies use the fitness to decide which individuals to choose for a new generation and procreation. We concisely present a few selection policies, which are relevant for this paper.

\paragraph{Roulette wheel selection}

Also known as Fitness proportionate selection, uses the fitness value of each individual to associate a probabilty of being selected to procreate. The probabilities are calculated to give the most fit individual the best chance to be selected. If we say that $f_i$ is the fitness of individual $i$, the probability is calculated as $p_i = \frac{f_i}{F}$, where $F = \sum_{j=1}^{I} f_j$, and $I$ is the total number of individuals\cite{tang1996genetic} \cite{koza1992genetic}. The most fit individual has the highest probability to be selected, while the least fit individual has the lowest probability.

Imagine a roulette wheel that is spun. The individual that is most fit might cover 38 \% of the entire wheel, while the rest of the individuals combined have 62 \% chance. It is obvious that the fittest individual will be selected most often.

\paragraph{Stochastic universal sampling}

This policy is similar to Roulette wheel selection, with one exception. When individuals are selected for procreation, pointers are used to choose the individuals, instead of randomly choosing an individual. The number of pointers, $P$, is equal to the number of individuals for the next generation, and the pointers are equally spaced. The first pointer is placed at a random position in the range $[0, \frac{1}{P}]$, and the space between each pointer is equal to $\frac{1}{P}$. Each pointer then points to an individual, and these individuals are selected for procreation\cite{baker1987reducing}.

\paragraph{Reward-based selection}

Individuals have an associated reward, which is computed as the sum of the individual's reward and the reward of its parents. If the individual is selected for the next generation, then the individual and its parents recieve a reward. The probability for an individual to be selected is proportional to the cumulative reward. There are different functions to calculate a reward\cite{loshchilov2011not}.

\paragraph{Tournament selection}

As the name suggest, this policy works as any other tournament. It involves running several tournaments, and the winner of each tournament is chosen to procreate. The individuals compete to solve the given problem optimally, and the winner is selected\cite{miller1996genetic}.


