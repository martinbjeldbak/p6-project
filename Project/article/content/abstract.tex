\begin{abstract}
We propose a behaviour-driven diversity measurement for genetic algorithms with artificial neural networks. We call it \di{} (\dia{}), and it is an extension of Simpsons Diversity Index known from ecology. 
%In this paper, we propose a phenotypic diversity measurement in genetic algorithms with neural networks as individuals, called \di{} (\dia{}). It is an extension of Simpsons Diversity Index known from ecology.

Avoiding low diversity in a genetic algorithm's population is crucial for finding a global optimum. When diversity is overlooked, premature convergence is the consequence, which possibly leads to only a local optimum.

Experiments are conducted to compare \dia{} to the genotypic diversity measure of Hamming distance, and a phenotypic fitness-based diversity measure. We argue why both of these measures have weaknesses that \dia{} overcomes.

%Results: What's the answer? 
Our experiments show that \dia{} consistently mirrors the intuition of how diversity in populations develop for various replacement rules, better than that of the Hamming distance and fitness-based diversity measurement methods, which seem less predictable. Interestingly, the experiments also show that there might be a connection between the genetic construction of an individual and its phenotypic behaviour.

%Old as of: 220514 14.34 Our results show that the fitness-based measure is unpredictable because the data shows irregular spikes in our experiments. \dia{} and the Hamming distance measure showed a stabile progression in measuring the diversity. The data shows some form of correspondance between these two measures.
\end{abstract}
