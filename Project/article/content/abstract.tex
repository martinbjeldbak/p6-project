\begin{abstract}
We propose a behaviour-based diversity measurement for genetic algorithms with artificial neural networks. We call it \di{} (\dia{}), and it is an extension of Simpsons Diversity Index known from ecology. Avoiding low diversity in a genetic algorithm's population is crucial for finding a global optimum. When diversity is overlooked, premature convergence is the consequence, which possibly leads to only a local optimum.

Experiments are conducted to compare \dia{} to the genotypic diversity measure of Hamming distance, and a phenotypic fitness-based diversity measure. We argue why both of these measures have weaknesses that \dia{} overcomes. Our experiments show that \dia{} consistently mirrors the intuition of how diversity in populations develop for various replacement rules, better than that of the Hamming distance and fitness-based diversity measurement methods, which seem less predictable. Interestingly, the experiments also show that there might be a connection between the genetic and behavioural diversity among individuals.
\end{abstract}
