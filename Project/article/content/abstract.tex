\begin{abstract}
This paper presents \dia, a phenotypic diversity measurement for neural network combined with genetic programs. It is an extension of the Simpsons Diversity Index known from ecology.

%Motivation:
%Why do we care about the problem and the results?
Avoiding low diversity in a genetic algorithm's population is crucial for finding a global optimum. When diversity is overlooked, premature convergence is the consequence. This most likely leads to local optimum.

%Approach:
%How did you go about solving or making progress on the problem?
Many studies have been conducted in this field of research, and we compare our method against other results, which aimed to solve the same problem.

%Results:
%What's the answer? 
Our results show that\dots (Succesful or failure?)

%Conclusions:
%What are the implications of your answer?
This leads to\dots (What are the implications of our answer?)

%Neural networks are often used as individuals within populations of genetic algorithms. Many operators exist to transition a population from one generation to the next. The drawback with many of these methods is that only the offspring of a population are selected to form the next generation, resulting in a fraction of the population being bred on -- individuals with highest fitness. This causes the individuals to share the same traits and possibly collect around local minima. We develop a new operator for offspring selection that attempts to maximize diversity in a population and hence be more effective at finding the global maximum. It only selects the offspring if it has a higher fitness than either of its parents. If this is the case, both parents are replaced with the individual and a random immigrant individual is introduced. If its fitness is lower, both parents survive.

%To evaluate our offspring selection policy, we also develop a method to evaluate diversity of a population consisting of neural networks with outputs that can be classified into species, or bins. This is done by repeatedly giving all individuals in a population the same, random inputs and classifying them by which of their outputs is largest, then evaluating the diversity using Simpson's Diversity Index known from ecology.%

%Our experiments show how well our selection policy compares to crowding and other methods\ldots (we have only just started running experiments)
\end{abstract}
