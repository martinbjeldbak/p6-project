\begin{abstract}
Controlling diversity in a genetic algorithm's population is crucial for finding the global optimum. When diversity is overlooked, premature convergence is the consequence, which possibly only leads to local optima. We propose a trait-based diversity measurement for genetic algorithms using artificial neural networks, which we call \di{} (\dia{}).

Experiments are conducted to compare \dia{} to the genotypic diversity measure of Hamming distance, and a phenotypic fitness-based diversity measure. We argue that both of these measures have weaknesses that \dia{} overcomes. Our experiments show that \dia{} consistently mirrors the intuition of trait diversity in populations, better than Hamming distance and fitness-based diversity measurement methods, which seem less predictable. Interestingly, the experiments also show that there might be a connection between the diversities returned by Hamming distance and \dia{} among neural networks.
\end{abstract}
