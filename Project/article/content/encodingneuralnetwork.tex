\subsubsection{Neural networks as individuals}
%bitstrings
%weight on connections
%each input and each output
%amount of hidden neurons

Neural networks can be used to solve many types of problems, e.g., classification and decision problems. We assume that every individual in its population will have the same architecture. That is, the number of neurons, the size of each layer, and how neurons are connected is the same.

% Uncommented this paragraph friday 2014-3-21, as it just says what a NN is? -Martin
%Each possible input an individual can get will be received through an input neuron. Likewise, each possible action an individual can perform is formulated through the output neurons. The network is constructed with connections between neurons with associated weights. These weights are used to calculate an action given the actual input.

Thus, neural networks differ only in their weights between neurons and the bias of each neuron. Each individual is therefore represented only in terms of the weights and biases.
%If two chromosomes have different bit strings, we say that they have different genotypes. If the neural network they encode produces a different output for some input, we say they have different phenotypes.
% ALso removed this friday 2014-3-21, no need to reference it twice. Possibly if there was an example on how to encode the network in the example?
%An illustration of neural networks is presented in \cref{fig:ann}, and should give a good idea of how the bit string is concatenated.

%\[
%  \underbrace{w_{1,1}}_{n} w_{1,2} \ldots w_{i,j}
%\]

%\[
%  \ldots \underbrace{w_{i,j-1}}_{n} w_{i,j} w_{i,j+1} \ldots
%\]

%where $w_{i,j}$ represents the weight of the connection between the $i'th$ and $j'th$ neuron in bits. Each weight is concatenated with the next weight. The length of any weight is of size $n$.
%
