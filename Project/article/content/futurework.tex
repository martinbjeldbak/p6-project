\section{Future work}\label{sec:futurework}
Much research has already been done on determining how the tweaking of different parameters in a genetic algorithm (GA) affects the fitness values achieved, as well as undesirable convergence. Our diversity measure, \di{} (\dia{}), is a new tool for investigating the impact on behavioural differences when tweaking the parameters of a GA.
From our experiments, we saw that the Explore-Exploit replacement rule produced the most fit individuals for all test cases.
Many other replacement rule exist that switch between exploration and exploitation phases based on changes in diversity.
It is unknown whether these replacement rules will produce more fit individuals by using \dia{} as a diversity measure.

\dia{} is not limited to be used in GAs only. In fact, it can be used anywhere where one wants to measure the behavioural differences between two or more neural networks. The many possible applications of \dia{} still have to be uncovered.
