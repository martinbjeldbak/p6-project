For many problems in computer science, a greedy strategy will not always lead to the best result.
A greedy strategy is one where each step taken is the step that looks most promising at the moment.
If a mountain climber wants to climb to the highest spot in The Himalayas, a greedy strategy is to move in any direction he think is the best, as long as he climbs up, and never backs down.
Using this strategy, he is definitely able to climb up somewhere, but once he reaches a mountain top, he might discover an even greater mountain he could not see before. Unfortunately, his strategy is to never back down, so now he is stuck at what we call a local optimum. What he wanted was the global optimum - the highest mountain of all.
 
Genetic algorithms (GA) can be used to search for a global optimum in many types of problems, e.g. optimization problems.
A genetic algorithm manages a number of individuals, which constitute a population.
Each of these individuals represents a proposed solution to solve the problem.
Individuals are conceptually just a chromosome with different genes on it.
In a GA, individuals are represented internally as a bit string.
There are many ways in which the bit string can be interpreted to form a solution to the problem in question.
Often, a neural network is decoded from the chromosome. A neural network takes a number of inputs, does some internal calculations, and yields a number of outputs, which can be interpreted as a solution to the problem.
The bit string is interpreted determines how these calculations are carried out.
For the particular problem of adding two integers, the neural network can be given the two integers $s$ and $t$ as input, and output a single value, which ideally should be $s+t$.

Two individuals may solve a problem completely different. 
Individuals in a GA procreates to form offspring.
The solution an offspring individual propose will typically combine traits from both its parents.
Hopefully, the combination of these traits will be a better solution to the particular problem in question.
The fitness of an individual indicates how well it solves the problem.

A diverse set of individuals in the population is crucial to identify the combination of traits that work best. 
If diversity is not maintained, only a local optimum of the available traits in the population will be explored, and a global optimum might never be found~\cite{ursem2002diversity}.

We believe it is essential that a diversity measure reflects the difference in traits among individuals. To the best of our knowledge, no current genotypic or phenotypic measures reflect this. Recall, that we are only concerned with neural networks as individuals. Since a \emph{trait} is a rather vague term, we introduce a clear definition of traits among neural networks: ``\emph{two neural networks have different traits if they for some input produce different outputs}''.
We will now argue why we think that neither fitness-based nor genotypic diversity measures catch this trait diversity. 

Consider two individuals who try to find the highest peak on an elevation map, where the fitness of each individual is based on the height of the peak they return.
The two individuals might have completely different strategies causing them to get stuck on two different mountains.
If both mountains happen to have the same height, the two individuals will have the same fitness, and hence a fitness-based diversity measure will return a low diversity, even though the two individuals have different traits.

Two individuals of different genotypes can be equivalent, which means they yield the same output on any input.
An example of such two neural networks is shown in \cref{fig:entire-eqnetwork}.
No matter what input they receive, their output will always be the same.
They are genotypically very diverse (Hamming distance of 6), but not trait diverse at all.
%
\begin{figure*}
  \begin{subfigure}{0.5\textwidth}
    \centering
    \inputresizeto{0.5\linewidth}{drawings/eqnetworks/eqnetworks3}
    \caption{An artificial neural network with connections and weights.}\label{fig:eqnetwork}
  \end{subfigure}
  \begin{subfigure}{0.5\textwidth}
    \centering
    \inputresizeto{0.5\linewidth}{drawings/eqnetworks/eqnetworks4}
    \caption{An artificial neural network equivalent to \cref{fig:eqnetwork}.}\label{fig:eqnetwork2}
  \end{subfigure}
  \caption{Networks with same phenotype, but different genotypes. The binary representation assumes that each weight is represented by four bits.}\label{fig:entire-eqnetwork}
\end{figure*}

%

%Many existing methods which aim to increase diversity, require a lot of computational power \citpls{}. We will develop a method that is computationally inexpensive, yet maintains a high diversity in a population. Common methods for measuring the phenotypic diversity of a population only take into account the fitness of each individual \citpls{}. We develop a method for measuring phenotypic diversity that takes into account the different traits among individuals. 
