Genetic algorithms are used for search optimisation. A genetic algorithm consists of a number of individuals within one population. Each individual tries to solve a given problem optimally. Populations contain a set of individuals, where the individuals procreate by combining traits from the parents to create an offspring. The offspring has a slightly different set of traits than the parents. This slight difference might give it an advantage when trying to solve the given problem. If the offspring solves the given problem better than the parents did, then the fitness of the offspring will be higher than that of the parents. 

Offspring should ideally contain the best traits from both parents, hence a diverse set of individuals is crucial to identify the combination of traits that work best. Consider a function for which the population must find a optimal solution for. If diversity is not maintained, only a local optimal solution of those traits available in the population will be explored, and a global optimal solution might not be found.
\cite{ursem2002diversity}

Many existing methods which aim to increase diversity require a lot of computational power \citpls{}. We will to develop a method that is computationally inexpensive, yet maintains a high diversity in a population. Common methods for measuring the phenotypic diversity of a population only take into account the fitness of each individual \citpls{}. We develop a method for measuring phenotypic diversity that takes into account the different traits among individuals. 
