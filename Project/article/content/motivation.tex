
\subsection{Motivation}
\label{sec:motivation}

%Diversity is important. Avoid retardation.
%Local optimum. The best breed. Low diversity.
%Maintain high diversity. Slow search for optimum.

Genetic algorithms rely on the individuals in a population to be able to evolve. Each individual tries to solve a given problem in its own way. Naturally, this implies that it is important to maintain a divers set of individuals to be able to keep evolving and trying different solutions to the problem. Otherwise, if diversity is not maintained within the population, a global optimal solution might never be found.
\cite{ursem2002diversity}

If diversity is not taken into account, it will of course quickly become low. The population will contain more and more similar individuals, which actually stops the much needed evolution. This approach produces a local optimal solution, by letting the best of the individuals breed and produce offspring, without any kind of intervention to try to maintain diversity. 

The benefit of maintaining a divers population, is a higher likelihood that the outcome will be a global optimal solution. The downside of maintaining diversity, is that the search for a solution becomes much slower. We propose an approach where the diversity is maximised and the time searching for a solution is minimised.
