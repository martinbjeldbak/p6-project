Many problems have non-linear search spaces with multiple local maxima. Genetic algorithms can be used to lead a search for global maxima. A genetic algorithm manages a number of individuals, which consitute a population. Each of these individuals is a possible optimal solution to the problem. Individuals possibly offering a better solution to this problem are formed by two individuals in the population combining their traits by breeding, or \emph{procreating}. The intuition is that by combining two individuals with sought-after traits, their offspring will have a better solution to the problem. How well an individual solves a problem is expressed by a fitness value associated with each individual. 

Ideally, offspring should contain the best traits from both parents, hence a diverse set of individuals is crucial to identify the combination of traits that work best. Consider a function for which the population must find an optimal solution for. If diversity is not maintained, only a local optimal solution of those traits available in the population will be explored, and a global optimal solution might ever not be found~\cite{ursem2002diversity}.

Diversity measures can be either genotypic or phenotypic. The former concerns different ways individuals are represented, while the latter is concerned with behavioural differences. Phenotypic diversity is often measured based on how fit each individual is. Two individuals with different traits do not necessarily have different fitness values. Consider two artificial intelligent (AI) players for the cell phone game ``Snake''. Their fitness can be calculated based on how many pieces of food they collect before they die. One of the AI players may have traits that makes it good at avoiding death by not hitting any walls or its own body. Sometimes, by chance, it hits a piece of food. The other AI player may have traits that makes it good at searching for food. The two AI players can have the same fitness value, that is, they collect the same amount of food before they die, and still have completely different traits. Therefore, fitness-based diversity measures do not always catch the difference in traits among individuals.

We develop a new method for measuring phenotypic diversity, which we claim is more suitable for measuring diversity in genetic algorihtms that use neural networks as individuals, compared to fitness-based diversity measures. With ``more suitable'', we mean a diversity function that better reflects different traits among the individuals of a population. We call this measure \emph{Neural Network Trait Diversity} (NNTD). Futhermore, we develop a policy for how offspring is selected to form the next generation of a population, which forces a higher NNTD\@. We call this selection policy \emph{Ancestor Elitism Selection Policy} (AESP).

%Many existing methods which aim to increase diversity, require a lot of computational power \citpls{}. We will develop a method that is computationally inexpensive, yet maintains a high diversity in a population. Common methods for measuring the phenotypic diversity of a population only take into account the fitness of each individual \citpls{}. We develop a method for measuring phenotypic diversity that takes into account the different traits among individuals. 
