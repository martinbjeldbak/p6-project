Many problems have non-linear search spaces with multiple local maxima. Genetic algorithms can be used to lead a search for global maxima. A genetic algorithm manages a number of individuals, which consitute a population. Each of these individuals is a possible optimal solution to the problem. Individuals possibly offering a better solution to this problem are formed by two individuals in the population combining their traits by breeding, or \emph{procreating}. The intuition is that by combining two individuals with sought-after traits, their offspring will have a better solution to the problem. How well an individual solves a problem is expressed by a fitness value associated with each individual.

Ideally, offspring should contain the best traits from both parents, hence a diverse set of individuals is crucial to identify the combination of traits that work best. Consider a function for which the population must find an optimal solution for. If diversity is not maintained, only a local optimal solution of those traits available in the population will be explored, and a global optimal solution might ever not be found~\cite{ursem2002diversity}.



%Many existing methods which aim to increase diversity, require a lot of computational power \citpls{}. We will develop a method that is computationally inexpensive, yet maintains a high diversity in a population. Common methods for measuring the phenotypic diversity of a population only take into account the fitness of each individual \citpls{}. We develop a method for measuring phenotypic diversity that takes into account the different traits among individuals. 
