\section{Experiments}\label{sec:experiments}
In this section, we evaluate our proposed diversity measurement (\dia) by using it to measure diversity of a population in two different environments.

In the first environment, which we will refer to as the static environment, the GA will not iterate.
For three maximization problems we introduce in \cref{sec:problems}, we create an initial population using a constrain $c$, and a significance $\alpha$. We choose $c$ such that we have an intuition about how the behavioural diversity of individuals will be dependent on $\alpha$. By experiments, we will see if it is likely that \dia{} catches this dependency.

In the second environment, which we will refer to as the dynamic environment, we will let the GA run for a number of iterations on each of the three problems. 
For each problem, we will run the GA using each of the four replacement rules introduced in \cref{sec:replacementrules}.
During all of the tests, we will measure the diversity using each of the diversity measures described in \cref{sec:diversitymeasures}, as well as \dia.

We hope to see that the diversities returned by \dia{} better match the expected behavioural differences than any of the other diversity measures.

%All experiments are performed on three different discrete problems, which we describe in \cref{sec:problems}.
\subsection{Parameter settings}

\begin{table}
  \centering
  \begin{tabular}{l S}
    \toprule
    Parameter & {Specification} \\
    \midrule
    Number of runs & 100 \\
    Generations per run (snake) & 2000 \\
    Generations per run (XOR) & 2000 \\
    Generations per run (leaf) & 500 \\
    Population size & 100 \\
    Selection method & {rank-based} \\
    \bottomrule
  \end{tabular}
  \caption{GA parameters used throughout the experiments.}
  \label{tab:gaparam}
\end{table}

We use a population size of 100. For each iteration, 100 new offspring individuals are created. Half of the offsprings are cloned from a random parent. These are found by using a rank-based selection, and then mutated. The other half is made by performing crossover between two random parents, also using rank-based selection. These settings are presented in \cref{tab:gaparam}.

Three crossover methods are used: one-point crossover, two-point crossover, and uniform crossover. When a new individual is created using crossover, there is an equal chance for any of the crossover methods to be used. Each offspring created by crossover, has a \perc{10} chance to be mutated. When mutating an individual, each bit in its bit string has a \perc{5} chance to be assigned a random boolean value. For specific implementation details, we refer to the source code~\cite{mbm:kmc:ekoGA}.

\subsection{Problems}
\label{sec:problems}
We hereby explain the three discrete problems we perform our experiments on.

\subsubsection{XOR}
We use a neural network to approximate the XOR between two 8-bit strings.
To evaluate fitness of a neural network, we choose 1000 random instances of two 8-bit strings.
For each instance, we determine how many bits the neural network calculates correct.
The fitness is the defined as the average number of bits calculated correct.
Fitness values will thus lie in the range $[0-8]$.
Since the XOR between two random bits will be evenly distributed between 0 and 1,
randomly guessing a solution to the 8-bit XOR problem, is expected to yield a fitness of 4.
The network has 16 input neurons, 16 hidden neurons, 8 output neurons, 2 bits per weight, and 1 bit per bias for each hidden and output neuron.
Any output neuron $i$ represents the XOR between the two input neurons $i$ and $i+8$. 
We have used as few neurons and bits to represent weights and biases as possible, while still being able to verify that the maximum fitness value of 8 is achievable.
The same random seed is always used for generating the 1000 problem instances.

\subsubsection{Leaf classification}
We use a neural network to classify the Leaf data set~\cite{Bache+Lichman:2013, leafdataset}.
The neural network is given 16 properties about an unknown leaf and has to decide which of 40 types of leaves it is.
Fitness is evaluated based on how many instances out of the entire data set the neural network correctly classifies.
The implementation consists of 16 input neurons, 10 hidden and 40 output neurons. The output neuron with the highest value decides the classification. Each weight is encoded by 9 bits and neurons have no bias.

\subsubsection{Snake}
Snake is a game found on old Nokia cell phones, where you are moving a snake around a grid to pick up pieces of food.
Every time a piece of food is collected, both the length of the snake and your score increases by 1.
You loose if the snake head hits its body or one of the edges of the grid.
At all times, the grid contains only a single piece of food.
The game becomes harder as the length of the snake increases, and as the snake is constantly moving, it becomes hard to not trap yourself.

We use a neural network to play a game of Snake in a $10\times10$ grid with an initial snake length of 5 units.
We have defined the fitness of a neural network to be $f + \frac{f}{s/1000}$,
where $f$ is the amount of food it collects and $s$ is the total number of steps the snake is alive. The game is constrained such that the snake can only change its direction \SI{90}{\degree} per step. The neural network has 6 input neurons, each receiving a bit of information about the game state:
\begin{enumerate}
  \item \set{-1, 0, 1} Whether food is to the left, vertically aligned, or to the right of the snake head.
  \item \set{-1, 0, 1} Whether food is above, horizontally aligned, or below the snake head.
  \item \set{0, 1} If the snake dies if its next move is up
  \item \set{0, 1} If the snake dies if its next move is down
  \item \set{0, 1} If the snake dies if its next move is right 
  \item \set{0, 1} If the snake dies if its next move is left
\end{enumerate}
4 output neurons are used. The neuron with the highest value determines whether to move the snake right, left, up, or down.
The neural network uses 5 hidden neurons, 9 bits per weight and neurons have no bias.
For every game of Snake, we always use the same random seed to decide the positions where pieces of food will spawn.

\subsubsection{Criteria for selection}
The XOR function is interesting since it is the simplest boolean function that is not linearly separable.
This fact has made it quite popular in NN research communities\cite{masterThesisGANN}.
A classification problem like Leaf is interesting because it is radically different from the XOR problem.
XOR is a simple and well defined function, whereas classifying leaves is more complex and depends on observations in nature, which may contain noise.
The Snake game differs in that it is an agent decision problem, where not just a single, but a sequence of decisions determines the outcome, where each decision changes the intermediate state.

\subsection{Static experiments}
We perform two different static experiments, which differ in the way they constrain the initially generated population. 

\subsubsection{Initial similarity}
In the first test, we introduce the variable \emph{initial similarity}, which is a real value in the range $[0,1]$.
When making an initial population, an initial similarity of $\alpha$ means that $\alpha$ of the individuals in the population will have the exact same genotype, and $(1-\alpha)$ of the individuals are completely random.
An initial similarity of \num{1} means that all genotypes are the same, and hence the behaviour of all individuals are the same as well. In this case, we will expect the lowest diversity possible.
As initial similarity increases, more genotypes will be identical, and hence more individuals will behave the same.
We therefore expect the diversity to decrease as initial similarity is increased.
An initial similarity of \num{0} means that all genotypes are random, and hence we will expect the most diverse behaviour among individuals to be found here.

\subsubsection{Initial mutation}
In second test, we introduce the variable \emph{initial mutation}, which is a real value in the range $[0,1]$.
When making an initial population, an initial mutation of $\alpha$ affects the population in the following way.
A random genotype is created and given to every individual in the population, such that all individuals have an identical, randomly chosen genotype.
Now, the bit string of each individual is mutated. Each bit will with probability $\alpha$ be set to a random boolean value. 
We expect to see an increase in the different of behaviour, as a result of the mutation of bit strings. 

\subsubsection{Results}
Experiments run with initial similarity and initial mutation between 0 and 1 are shown in \cref{fig:initial-similarity} and \cref{fig:initial-mutation}, respectively.
%
\begin{figure*}
  \centering
  \begin{subfigure}[b]{0.33\textwidth}
    \begin{tikzpicture}
      \begin{axis}[
          initial-sim-root,
          legend to name=initSimMeasures,
          title=Leaf,
        ]
        \addplot[mark=*, color=blue]
        table[y index=1, x index=0] {data/initial_similarity_leaf.csv};
        \addplot[mark=square*, color=aqua] % Hamming
        table[y index=2, x index=0] {data/initial_similarity_leaf.csv};
        \addplot[mark=triangle*, color=teal]
        table[y index=3, x index=0] {data/initial_similarity_leaf.csv};
      \end{axis}
    \end{tikzpicture}
  \end{subfigure}%
  ~
  \begin{subfigure}[b]{0.33\textwidth}
    \begin{tikzpicture}
      \begin{axis}[
          initial-sim,
          title  = Snake,
        ]
        \addplot[mark=*, color=blue]
        table[y index=1, x index=0] {data/initial_similarity_snake.csv};
        \addplot[mark=square*, color=aqua] % Hamming
        table[y index=2, x index=0] {data/initial_similarity_snake.csv};
        \addplot[mark=triangle*, color=teal]
        table[y index=3, x index=0] {data/initial_similarity_snake.csv};
      \end{axis}
    \end{tikzpicture}
  \end{subfigure}%
  ~
  \begin{subfigure}[b]{0.33\textwidth}
    \begin{tikzpicture}
      \begin{axis}[
          initial-sim,
          title  = 8-bit XOR,
        ]
        \addplot[mark=*, color=blue]
        table[y index=1, x index=0] {data/initial_similarity_snake.csv};
        \addplot[mark=square*, color=aqua] % Hamming
        table[y index=2, x index=0] {data/initial_similarity_xor.csv};
        \addplot[mark=triangle*, color=teal]
        table[y index=3, x index=0] {data/initial_similarity_xor.csv};
      \end{axis}
    \end{tikzpicture}
  \end{subfigure}
  \ref{initSimMeasures}
  \caption{Average over \num{100} runs for each diversity measure on data sets over intervals of initial similarity. Each point represents the average of \num{100} runs for that initial similarity value.}\label{fig:initial-similarity}
\end{figure*}

%
% Init sim results
The results of various initial similarity values shown in \cref{fig:initial-similarity} play along with our intuition. For each of the data sets, each diversity measure's diversity output gradually falls upon increasing the amount of similar individuals in the population. Interestingly enough, at an initial similarity of 0, we expect all diversity measures to output maximum diversity.Here, \dia{} outputs the maximum possible diversity of 1, whereas Hamming distance outputs a diversity that is only half of its maximum. It seems that \dia{} captures the chaotic nature of a completely random population better than the other two measures, with a larger fall in diversity over time. One could naively assume that you could just double Hamming distance's diversity, but it is actually possible for the Hamming distance measure to output 1 if each bit in two individuals is different.

\begin{figure*}
  \centering
  \begin{subfigure}[b]{0.33\textwidth}
    \begin{tikzpicture}
      \begin{axis}[
          initial-mut-root,
          initial-mut,
          legend to name=initMutMeasures,
          title=Leaf,
        ]
        \addplot[mark=*, color=blue] % Fitness-based
        table[y index=1, x index=0] {data/initial_mutation_leaf.csv};
        \addplot[mark=square*, color=red] % Hamming
        table[y index=2, x index=0] {data/initial_mutation_leaf.csv};
        \addplot[mark=triangle*, color=green] % NNTD
        table[y index=3, x index=0] {data/initial_mutation_leaf.csv};
      \end{axis}
    \end{tikzpicture}
  \end{subfigure}%
  ~
  \begin{subfigure}[b]{0.33\textwidth}
    \begin{tikzpicture}
      \begin{axis}[
          initial-mut, 
          title  = Snake,
        ]
        \addplot[mark=*, color=blue]
        table[y index=1, x index=0] {data/initial_mutation_snake.csv};
        \addplot[mark=square*, color=red]
        table[y index=2, x index=0] {data/initial_mutation_snake.csv};
        \addplot[mark=triangle*, color=green]
        table[y index=3, x index=0] {data/initial_mutation_snake.csv};
      \end{axis}
    \end{tikzpicture}
  \end{subfigure}%
  ~
  \begin{subfigure}[b]{0.33\textwidth}
    \begin{tikzpicture}
      \begin{axis}[
          initial-mut, 
          title  = 8-bit XOR,
        ]
        \addplot[mark=*, color=blue]
        table[y index=1, x index=0] {data/initial_mutation_xor.csv};
        \addplot[mark=square*, color=red]
        table[y index=2, x index=0] {data/initial_mutation_xor.csv};
        \addplot[mark=triangle*, color=green]
        table[y index=3, x index=0] {data/initial_mutation_xor.csv};
      \end{axis}
    \end{tikzpicture}
  \end{subfigure}
  \ref{initMutMeasures}
  \caption{Average over \num{100} runs for each diversity measure on data sets over intervals of initial mutation. Each point represents the average of \num{100} runs for that initial mutation value.}\label{fig:initial-mutation}
\end{figure*}


% Init mut results
By changing the initial mutation in a population with the results shown in \cref{fig:initial-mutation}, we notice a larger difference between diversity measures. At an initial mutation rate of a mere \perc{1}, \dia{} takes a great leap compared to the other two measures. For Hamming distance, it is obvious that changing only a few bits in the genotypes will only cause a small change in diversity. This is because Hamming distance measures diversity based on genotypes.

For the fitness-based diversity measure, it must be noted that the fitness values are dependent on the particular problem in question and how one chooses to define the fitness function. Consider for instance the problem of making an AI for the game Snake. We have experienced that if we define the fitness only in terms of how many pieces of food a snake collects, then about \perc{98} of all random individuals get a fitness of 0. This is the reason why we chose a fitness function that also takes into account the number of steps a snake is alive. Despite yielding more diverse fitness values, it also increased the fitness values obtained after just 100 iterations notably. This does not mean that a fitness function cannot reflect behavioural differences, but it shows that how one defines the fitness function is crucial if one wishes to catch the behavioural differences of individuals.

\subsection{Dynamic experiments}\label{sec:continuousdiversity}
Our main goal with the experiments we perform is to see how diversity progresses throughout the many iterations of a GA. To measure diversity, we use a fitness-based measurement, Hamming distance, and our own measure, \dia. For each experiment we use the four replacement rules: Greedy Replacement, Ancestor Elitism, Single Parent Elitism, and MEEE, described in \cref{sec:replacementrules}.

For the snake and XOR problems, we measured the average diversity of \num{100} runs over \num{2000} iterations, meaning each run initializes an entirely new population, and each population goes through \num{2000} iterations of procreation in the search for optimal solutions. After each iteration, diversity of the population is measured with each diversity measure. We then take the average diversity for each generation of the \num{100} runs. For the leaf data set, we ran the same experiment only through \num{500} generations. This is due to the fact that we saw no improvement in the population beyond this number of iterations. 

\subsubsection{Results}
The results are presented in figures~\ref{fig:dynamic-leaf} through~\ref{fig:dynamic-snake}. The experiments performed on the leaf data set are presented in \cref{fig:dynamic-leaf}, the XOR experiments are presented in \cref{fig:dynamic-xor}, and the Snake game results are presented in \cref{fig:dynamic-snake}. Each of these three figures contain four plots. The first three of these plots illustrate each diversity measure, with the fourth plot illustrating the average fitness for the four replacement rules throughout the iterations. 

\paragraph{Results of Leaf} An interesting observation in \cref{fig:dynamic-leaf} is the irregular spike in diversity for the fitness-based measure around generation \num{50}. Why this happens is hard to say, due to the stochastic nature of GAs. \dia{} and Hamming distance measures look very similar for this data set. %We can argue that \dia{} better illustrates how diverse the traits of the individuals are, because with the Hamming distance, we do not see diversity ever rise over \num{0.5}, even for the Ancestor Elitism replacement rule, which consistently introduces many randomly initialized individuals. 

\begin{figure*}
  \centering
  \begin{subfigure}[b]{0.25\textwidth}
    \batchmode
    \begin{tikzpicture}
      \begin{axis}[
          dynamic,
          dynamic-root,
          legend to name=leafDynamic,
          title={\dia{}},
        ]
        \addplot[mark=*, smooth, color=black]
        table[ x index=0, y index=3] {data/replacementrule_game/greedy_leaf.csv};
        \addplot[mark=square*, smooth, color=blue]
        table[x index=0, y index=3] {data/replacementrule_game/aerr_leaf.csv};
        \addplot[mark=triangle*, smooth, color=red]
        table[x index=0, y index=3] {data/replacementrule_game/sperr_leaf.csv};
        \addplot[mark=diamond*, smooth, color=green]
        table[x index=0, y index=3] {data/replacementrule_game/eerr_leaf.csv};
      \end{axis}
    \end{tikzpicture}
    \scrollmode
  \end{subfigure}%
  ~
  \begin{subfigure}[b]{0.25\textwidth}
    \batchmode
    \begin{tikzpicture}
      \begin{axis}[
          dynamic,
          title=Fitness-based,
        ]
        \addplot[mark=*, smooth, color=black]
        table[x index=0, y index=1] {data/replacementrule_game/greedy_leaf.csv};
        \addplot[mark=square*, smooth, color=blue]
        table[x index=0, y index=1] {data/replacementrule_game/aerr_leaf.csv};
        \addplot[mark=triangle*, smooth, color=red]
        table[x index=0, y index=1] {data/replacementrule_game/sperr_leaf.csv};
        \addplot[mark=diamond*, smooth, color=green]
        table[x index=0, y index=1] {data/replacementrule_game/eerr_leaf.csv};
      \end{axis}
    \end{tikzpicture}
    \scrollmode
  \end{subfigure}%
  ~
  \begin{subfigure}[b]{0.25\textwidth}
    \batchmode
    \begin{tikzpicture}
      \begin{axis}[
          dynamic,
          title  = Hamming distance,
        ]
        \addplot[mark=*, smooth, color=black]
        table[x index=0, y index=2] {data/replacementrule_game/greedy_leaf.csv};
        \addplot[mark=square*, smooth, color=blue]
        table[x index=0, y index=2] {data/replacementrule_game/aerr_leaf.csv};
        \addplot[mark=triangle*, smooth, color=red]
        table[x index=0, y index=2] {data/replacementrule_game/sperr_leaf.csv};
        \addplot[mark=diamond*, smooth, color=green]
        table[x index=0, y index=2] {data/replacementrule_game/eerr_leaf.csv};
      \end{axis}
    \end{tikzpicture}
    \scrollmode
  \end{subfigure}%
  ~
  \begin{subfigure}[b]{0.25\textwidth}
    \batchmode
    \begin{tikzpicture}
      \begin{axis}[
          fitness
        ]
        \addplot[mark=*, smooth, color=black]
        table[x index=0, y index=4] {data/replacementrule_game/greedy_leaf.csv};
        \addplot[mark=square*, smooth, color=blue]
        table[x index=0, y index=4] {data/replacementrule_game/aerr_leaf.csv};
        \addplot[mark=triangle*, smooth, color=red]
        table[x index=0, y index=4] {data/replacementrule_game/sperr_leaf.csv};
        \addplot[mark=diamond*, smooth, color=green]
        table[x index=0, y index=4] {data/replacementrule_game/eerr_leaf.csv};
      \end{axis}
    \end{tikzpicture}
    \scrollmode
  \end{subfigure}
  \ref{leafDynamic}
  \caption{Average diversity over \num{100} runs for each replacement rule over \num{500} generations of the leaf data set.}\label{fig:dynamic-leaf}
\end{figure*}


\paragraph{Results of XOR} The results of each diversity measure presented in \cref{fig:dynamic-xor} are very similar to those of the Leaf data set. Once again, we see the irregular spike in the plot presenting the fitness-based measure. Once again, \dia{} and Hamming distance measures look very similar for this problem.%Furthermore, it is interesting how \dia{} and Hamming distance are a constant factor apart for each replacement rule.%  Also, for this data set, the fitness-based measurement has an overall higher diversity compared to the other two data sets.%This means that for this experiment, the fitness-based measurement is closer to \dia{} than Hamming distance, as it is for the other results.

\begin{figure*}
  \centering
  \begin{subfigure}[b]{0.25\textwidth}
    \begin{tikzpicture}
      \begin{axis}[
          dynamic,
          dynamic-root,
          title={\dia{}},
          legend to name=xorDynamic,
        ]
        \addplot[mark=*, smooth, color=black]
        table[x index=0, y index=3] {data/replacementrule_game/greedy_xor.csv};
        \addplot[mark=square*, smooth, color=blue]
        table[x index=0, y index=3] {data/replacementrule_game/aerr_xor.csv};
        \addplot[mark=triangle*, smooth, color=red]
        table[x index=0, y index=3] {data/replacementrule_game/sperr_xor.csv};
        \addplot[mark=diamond*, smooth, color=green]
        table[x index=0, y index=3] {data/replacementrule_game/eerr_xor.csv};
      \end{axis}
    \end{tikzpicture}
  \end{subfigure}%
  ~
  \begin{subfigure}[b]{0.25\textwidth}
    \begin{tikzpicture}
      \begin{axis}[
          dynamic,
          title=Fitness-based,
        ]
        \addplot[mark=*, smooth, color=black]
        table[x index=0, y index=1] {data/replacementrule_game/greedy_xor.csv};
        \addplot[mark=square*, smooth, color=blue]
        table[x index=0, y index=1] {data/replacementrule_game/aerr_xor.csv};
        \addplot[mark=triangle*, smooth, color=red]
        table[x index=0, y index=1] {data/replacementrule_game/sperr_xor.csv};
        \addplot[mark=diamond*, smooth, color=green]
        table[x index=0, y index=1] {data/replacementrule_game/eerr_xor.csv};
      \end{axis}
    \end{tikzpicture}
  \end{subfigure}%
  ~
  \begin{subfigure}[b]{0.25\textwidth}
    \begin{tikzpicture}
      \begin{axis}[
          dynamic,
          title  = Hamming distance,
        ]
        \addplot[mark=*, smooth, color=black]
        table[x index=0, y index=2] {data/replacementrule_game/greedy_xor.csv};
        \addplot[mark=square*, smooth, color=blue]
        table[x index=0, y index=2] {data/replacementrule_game/aerr_xor.csv};
        \addplot[mark=triangle*, smooth, color=red]
        table[x index=0, y index=2] {data/replacementrule_game/sperr_xor.csv};
        \addplot[mark=diamond*, smooth, color=green]
        table[x index=0, y index=2] {data/replacementrule_game/eerr_xor.csv};
      \end{axis}
    \end{tikzpicture}
  \end{subfigure}%
  ~
  \begin{subfigure}[b]{0.25\textwidth}
    \begin{tikzpicture}
      \begin{axis}[
          fitness,
        ]
        \addplot[mark=*, smooth, color=black]
        table[x index=0, y index=4] {data/replacementrule_game/greedy_xor.csv};
        \addplot[mark=square*, smooth, color=blue]
        table[x index=0, y index=4] {data/replacementrule_game/aerr_xor.csv};
        \addplot[mark=triangle*, smooth, color=red]
        table[x index=0, y index=4] {data/replacementrule_game/sperr_xor.csv};
        \addplot[mark=diamond*, smooth, color=green]
        table[x index=0, y index=4] {data/replacementrule_game/eerr_xor.csv};
      \end{axis}
    \end{tikzpicture}
  \end{subfigure}
  \ref{xorDynamic}
  \caption{Average over \num{100} runs for each diversity measure on approximating an 8-bit XOR gate.}\label{fig:dynamic-xor}
\end{figure*}


\paragraph{Results of Snake} The results shown in \cref{fig:dynamic-snake}, once again show similar slopes between the \dia{} and Hamming distance measures. The fitness-based measure is again very irregular.

\begin{figure*}
  \centering
  \begin{subfigure}[b]{0.33\textwidth}
      \begin{tikzpicture}
        \begin{axis}[
            dynamic-root,
            dynamic,
            title={\dia{}},
            legend to name=snakeDynamic,
          ]
          \addplot[smooth, color=navy]
          table[x index=0, y index=3] {data/replacementrule_game/greedy_snake.csv};
          \addplot[smooth, color=blue]
          table[x index=0, y index=3] {data/replacementrule_game/aerr_snake.csv};
          \addplot[smooth, color=aqua]
          table[x index=0, y index=3] {data/replacementrule_game/sperr_snake.csv};
          \addplot[smooth, color=teal]
          table[x index=0, y index=3] {data/replacementrule_game/eerr_snake.csv};
        \end{axis}
      \end{tikzpicture}
  \end{subfigure}%
  ~
  \begin{subfigure}[b]{0.33\textwidth}
      \begin{tikzpicture}
        \begin{axis}[
            dynamic,
            title=Fitness-based,
          ]
          \addplot[smooth, color=navy]
          table[x index=0, y index=1] {data/replacementrule_game/greedy_snake.csv};
          \addplot[smooth, color=blue]
          table[x index=0, y index=1] {data/replacementrule_game/aerr_snake.csv};
          \addplot[smooth, color=aqua]
          table[x index=0, y index=1] {data/replacementrule_game/sperr_snake.csv};
          \addplot[smooth, color=teal]
          table[x index=0, y index=1] {data/replacementrule_game/eerr_snake.csv};
        \end{axis}
      \end{tikzpicture}
  \end{subfigure}%
  ~
  \begin{subfigure}[b]{0.33\textwidth}
      \begin{tikzpicture}
        \begin{axis}[
            dynamic,
            title  = Hamming distance,
          ]
          \addplot[smooth, color=navy]
          table[x index=0, y index=2] {data/replacementrule_game/greedy_snake.csv};
          \addplot[smooth, color=blue]
          table[x index=0, y index=2] {data/replacementrule_game/aerr_snake.csv};
          \addplot[smooth, color=aqua]
          table[x index=0, y index=2] {data/replacementrule_game/sperr_snake.csv};
          \addplot[smooth, color=teal]
          table[x index=0, y index=2] {data/replacementrule_game/eerr_snake.csv};
        \end{axis}
      \end{tikzpicture}
  \end{subfigure}
  \ref{snakeDynamic}
  \caption{Average over \num{100} runs for each diversity measure on the snake data set.}\label{fig:dynamic-snake}
\end{figure*}


\paragraph{The big picture} If we compare experiments run on each data set illustrated in figures \ref{fig:dynamic-leaf} through \ref{fig:dynamic-snake}, we see that the fitness-based diversity measure stands out from the other two in each of the experiments. For example, the fitness-based plots have some irregular slopes, which are not easily explained. Because of this, combined with the fact that multiple individuals can have the same fitness yet different traits, we believe that fitness-based diversity does not accurately represent the actual diversity. Fitness-based diversity seems much more dependent on the problem, as its results are not regular. In each figure presenting the dynamic experiments, the plot presenting the fitness-based measurement is notably different from the two other plots, presenting \dia{} and Hamming distance.

%Another observation from the three experiments is that \dia{} and Hamming distance do not vary much between the different problems. 

The experiments also show that there might be a correspondence between the phenotypic \dia{} diversity measure and the genotypic Hamming distance diversity measure. If we take a closer look at the plots illustrating \dia{} and Hamming distance, we can see that their slopes are similar. %The Hamming distance is always below the \dia{} measurement.% So, the \dia{} shows the differences in the behaviour, and the Hamming distance shows the difference in the genetic make-up.  

When it comes to the fitness of the individuals, presented in the fourth plot of figures \ref{fig:dynamic-leaf} through \ref{fig:dynamic-snake}, it is clear to see that fitness is at its best when using the MEEE replacement rule. This rule does not have the most diverse population, but as we see, this is not necessarily a bad thing. This indicates that it is not the best solution to aim for a highly diverse population. One must find a balanced solution and take advantage of the fact that it is possible to switch between maintaining a high diversity and not doing so, which is in correspondance with the results presented in \cite{Darwen00doesextra}.


\subsection{Overall results}
\todo{Skriv fælles træk ved dem alle}
Regardless of problem type, each diversity measure performs similarly.

It is clear from our experiments, that replacement rules play a very large part in how quickly large fitness values are found.
