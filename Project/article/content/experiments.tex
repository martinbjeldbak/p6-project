\section{Experiments}\label{sec:experiments}
In this section, we evaluate our proposed diversity measurement (\dia) by using it to measure diversity of a population in two different environments.

In the first environment, which we refer to as the static environment, the GA does not iterate.
For the three maximization problems introduced in \cref{sec:problems}, we create an initial population using a constrain $c$, and a significance $\alpha$. We choose $c$ such that we have an intuition about how the behavioural diversity of individuals are dependent on $\alpha$. By experiments, we see if it is likely that \dia{} catches this dependency.

In the second environment, which we refer to as the dynamic environment, the GA runs for a number of iterations on each of the three problems. 
For each problem, we perform a test using each of the four replacement rules introduced in \cref{sec:replacementrules}.
During all of the tests, we measure the diversity using \dia{} as well as each of the diversity measures described in \cref{sec:diversitymeasures}.

By these experiments we can conclude whether the diversities returned by \dia{} better match the expected behavioural differences of the test cases, compared to the other diversity measures.

%All experiments are performed on three different discrete problems, which we describe in \cref{sec:problems}.
\subsection{Parameter settings}

\begin{table}
  \centering
  \begin{tabular}{l l S}
    \toprule
    Parameter & & {Specification} \\
    \midrule
    Number of runs & & 100 \\
    Generations per run & snake & 2000 \\
    & XOR & 2000 \\
    & leaf & 500 \\
    Population size & & 100 \\
    Selection method & & {rank-based} \\
    \bottomrule
  \end{tabular}
  \caption{GA parameters used throughout experimenting.}
  \label{tab:gaparam}
\end{table}

We use a population size of \num{100} individuals. For each iteration, \num{100} new offspring individuals are created. Half of the offspring is cloned from a random parent. These are found by using a rank-based selection, and then mutated. The other half is made by performing crossover between two random parents, also using rank-based selection. These settings are presented in \cref{tab:gaparam}.

Three crossover methods are used: one-point crossover, two-point crossover, and uniform crossover. When a new individual is chosen for creation using crossover, there is an equal chance for any of the crossover methods to be used. Each offspring created by crossover has a \perc{10} chance to be mutated. When mutating an individual, each bit in its bit string has a \perc{5} chance to be assigned a random boolean value. For specific implementation details, we refer to the source code~\cite{mbm:kmc:ekoGA}.

\subsection{Problems}
% Discrete: Leaf
The discrete problem is the small leaf data set~\cite{Bache+Lichman:2013, leafdataset}. Fitness is evaluated based on how many instances out of the entire data set the neural network correctly classifies. It consists of 16 inputs and 1 output.

% Discrete: Rosenbrock 
%Approximating the 2D Rosenbrock function is the second discrete problem. Here fitness is computed as the average distance to the actual output of the Rosenbrock function, over \num{10000} random inputs. The individuals have two input neurons and one output.

% Continuous: Snake
The continuous problem used is the Snake game known from Nokia cell phones. This game is very similar to the artificial ant problem~\cite[p.\ 147--155]{koza1992genetic}. A snake has to traverse a square $10\times10$ grid and search for a randomly positioned piece of food using a sensing function that's able to see the squares and their contents directly next to its head and above and below its head. Once the snake traverses into the square with food, the food disappears and the snake's length grows by one square. A new piece of food then spawns at a new random position. The snake has four actions to move around the grid, each of which takes one time unit: move up, down, left, and right. The game ends if the amount of time units exeeds \num{1000}, or if the snake collides into a wall. Its fitness function is based upon the amount of food eaten.

\subsection{Static experiments}
We perform two different static experiments, which differ in the way they constrain the initially generated population. 

\subsubsection{Initial similarity}
Here, we introduce the variable \emph{initial similarity}, which is a real value in the range $[0,1]$.
When making an initial population, an initial similarity of $\alpha$ means that $\alpha$ of the individuals in the population will have the exact same genotype, and $(1 - \alpha)$ of the individuals are completely random.
An initial similarity of \num{1} means that all genotypes are the same, and hence the traits of all individuals are the same as well. In this case, we expect the lowest diversity possible.
As initial similarity increases, more genotypes will be identical, and hence more individuals will have similar traits.
We therefore expect the diversity to decrease as initial similarity is increased.
An initial similarity of \num{0} means that all genotypes are random, and hence we expect the most diverse traits among individuals to be found here.

\subsubsection{Initial mutation}
Here, we introduce the variable \emph{initial mutation}, which is also a real value in the range $[0,1]$.
When making an initial population, an initial mutation of $\alpha$ affects the population in the following way.
A random genotype is created and given to every individual in the population, such that all individuals have an identical genotype.
Now, the bit string of each individual is mutated. Each bit will with probability $\alpha$ be set to a random boolean value. 
We expect to see increasingly different traits, as a result of mutating bit strings. 

\subsubsection{Results of initial similarity} The results of various initial similarity values shown in \cref{fig:initial-similarity} play along with our intuition. For each of the problems, each diversity measure's output gradually falls upon increasing the amount of similar individuals in the population. Interestingly enough, at an initial similarity of 0, we expect all diversity measures to output maximum diversity. Here, \dia{} outputs the maximum possible diversity of 1, whereas Hamming distance outputs a diversity that is only half of its maximum. One could assume that you could just double Hamming distance's diversity, but it is actually possible for the Hamming distance between two individuals to output 1, if every pair of bits are different. It seems that \dia{} captures the chaotic nature of a completely random population better than the other two measures, with a larger fall in diversity over time.
%
\begin{figure*}
  \centering
  \begin{subfigure}[b]{0.33\textwidth}
    \correctlyresize{\linewidth}{%
      \begin{tikzpicture}
        \begin{axis}[
            initial-sim-root,
            legend to name=initSimMeasures,
            title=Leaf,
          ]
          \addplot[mark=*, color=maroon]
          table[y index=1, x index=0] {data/initial_similarity_leaf.csv};
          \addplot[mark=square*, color=navy]
          table[y index=2, x index=0] {data/initial_similarity_leaf.csv};
          \addplot[mark=triangle*, color=blue]
          table[y index=3, x index=0] {data/initial_similarity_leaf.csv};
        \end{axis}
      \end{tikzpicture}
    }
  \end{subfigure}%
  ~
  \begin{subfigure}[b]{0.33\textwidth}
    \correctlyresize{\linewidth}{%
      \begin{tikzpicture}
        \begin{axis}[
            initial-sim,
            title  = Snake,
          ]
          \addplot[mark=*, color=maroon]
          table[y index=1, x index=0] {data/initial_similarity_snake.csv};
          \addplot[mark=square*, color=navy]
          table[y index=2, x index=0] {data/initial_similarity_snake.csv};
          \addplot[mark=triangle*, color=blue]
          table[y index=3, x index=0] {data/initial_similarity_snake.csv};
        \end{axis}
      \end{tikzpicture}
    }
  \end{subfigure}%
  ~
  \begin{subfigure}[b]{0.33\textwidth}
    \correctlyresize{\linewidth}{%
      \begin{tikzpicture}
        \begin{axis}[
            initial-sim,
            title  = XOR,
          ]
          \addplot[mark=*, color=maroon]
          table[y index=1, x index=0] {data/initial_similarity_snake.csv};
          \addplot[mark=square*, color=navy]
          table[y index=2, x index=0] {data/initial_similarity_xor.csv};
          \addplot[mark=triangle*, color=blue]
          table[y index=3, x index=0] {data/initial_similarity_xor.csv};
        \end{axis}
      \end{tikzpicture}
    }
  \end{subfigure}
  \ref{initSimMeasures}
  \caption{Average over \num{100} runs for each diversity measure on data sets over intervals of initial similarity.}\label{fig:initial-similarity}
\end{figure*}

%
\subsubsection{Results of initial mutation} By changing the initial mutation in a population, with the results shown in \cref{fig:initial-mutation}, we notice a larger difference between diversity measures. At an initial mutation rate of a mere \perc{1}, \dia{} takes a great leap compared to the other two measures. For Hamming distance, it is obvious that changing only a few bits in the genotypes will only cause a small change in diversity. This is because Hamming distance measures diversity based on genotypes. Consider for instance changing just a single bit of a chromosome. This bit could be a sign bit that causes a significant change to the neural network's traits, or it could cause no change at all. Hamming distance neglects this fact and produces the same diversity measure regardless of whether the bit caused a change of traits for the individual or not.

For the fitness-based diversity measure, it must be noted that the fitness values are dependent on the particular problem in question and how one chooses to define the fitness function. Consider for instance the problem of making an AI for the game Snake. We have experienced that if we define the fitness only in terms of how many pieces of food a snake collects, then about \perc{98} of randomly initialized individuals get a fitness value of 0. This is the reason why we chose a fitness function that also takes into account the number of steps a snake has been alive. Despite yielding more diverse fitness values, it also notably increased the fitness values obtained after just 100 iterations. This does not mean that a fitness function cannot reflect differences in traits, but it shows that how one defines the fitness function is crucial, if one wishes to catch the different traits among individuals.
%
\begin{figure*}
  \centering
  \begin{subfigure}[b]{0.33\textwidth}
    \begin{tikzpicture}
      \begin{axis}[
          initial-mut-root,
          legend to name=initMutMeasures,
          title=Leaf,
        ]
        \addplot[mark=*, color=blue] % Fitness-based
        table[y index=1, x index=0] {data/initial_mutation_leaf.csv};
        \addplot[mark=square*, color=aqua] % Hamming
        table[y index=2, x index=0] {data/initial_mutation_leaf.csv};
        \addplot[mark=triangle*, color=teal] % NNTD
        table[y index=3, x index=0] {data/initial_mutation_leaf.csv};
      \end{axis}
    \end{tikzpicture}
  \end{subfigure}%
  ~
  \begin{subfigure}[b]{0.33\textwidth}
    \begin{tikzpicture}
      \begin{axis}[
          initial-mut, 
          title  = Snake,
        ]
        \addplot[mark=*, color=blue]
        table[y index=1, x index=0] {data/initial_mutation_snake.csv};
        \addplot[mark=square*, color=aqua]
        table[y index=2, x index=0] {data/initial_mutation_snake.csv};
        \addplot[mark=triangle*, color=teal]
        table[y index=3, x index=0] {data/initial_mutation_snake.csv};
      \end{axis}
    \end{tikzpicture}
  \end{subfigure}%
  ~
  \begin{subfigure}[b]{0.33\textwidth}
    \begin{tikzpicture}
      \begin{axis}[
          initial-mut, 
          title  = 8-bit XOR,
        ]
        \addplot[mark=*, color=blue]
        table[y index=1, x index=0] {data/initial_mutation_xor.csv};
        \addplot[mark=square*, color=aqua]
        table[y index=2, x index=0] {data/initial_mutation_xor.csv};
        \addplot[mark=triangle*, color=teal]
        table[y index=3, x index=0] {data/initial_mutation_xor.csv};
      \end{axis}
    \end{tikzpicture}
  \end{subfigure}
  \ref{initMutMeasures}
  \caption{Average over \num{100} runs for each diversity measure on data sets over intervals of initial mutation. Each point represents the average of \num{100} runs for that initial mutation value.}\label{fig:initial-mutation}
\end{figure*}


\subsection{Dynamic experiments}\label{sec:continuousdiversity}
Our main goal with the experiments we perform is to see how diversity progresses throughout the many iterations of a GA. Furthermore, we must see if there are any significant differences between different diversity measures and replacement rules. To measure diversity, we use a fitness-based measurement, Hamming distance, and our own measure, \dia. The four replacement rules are: Greedy Replacement, Ancestor Elitism, Single Parent Elitism, and MEEE, described in \cref{sec:replacementrules}. We run the experiments four times, one with each of the replacement rules.

For the snake and XOR problems, we measured the average diversity of \num{100} runs over \num{2000} iterations, meaning each run initializes an entirely new population, and each population goes through \num{2000} iterations of procreation in the search for optimal solutions. After each iteration, diversity of the population is measured for each diversity measure. We then take the average diversity for each generation of the \num{100} runs. For the leaf data set, we ran the same experiment only through \num{500} generations, because it is a smaller data set consisting of only \num{340} instances. This is due to the fact that we saw no improvement in the population beyond this number of iterations. 

\subsubsection{Results}
The results are presented in figures~\ref{fig:dynamic-leaf} through~\ref{fig:dynamic-snake}. The experiments performed on the leaf data set are presented in \cref{fig:dynamic-leaf}, the XOR experiments are presented in \cref{fig:dynamic-xor}, and the Snake game results are presented in \cref{fig:dynamic-snake}. Each of these three figures contain four plots. The first three of these plots illustrate each diversity measure, with the fourth plot illustrating the average fitness for the four replacement rules throughout the iterations. 

\paragraph{Results of Leaf} By comparing the first three plots in \cref{fig:dynamic-leaf}, we see that each replacement rule has somewhat the same curve fit for each diversity measure. Each diversity measure captures the intuition of the replacement rules. An interesting observation is the irregular spike in diversity for the fitness-based measure around generation \num{50}. Why this happens is hard to say, due to the stochastic nature of GAs.

\dia{} and Hamming distance measures look very similar for this data set. We can argue that \dia{} better illustrates how diverse the traits of the individuals are, because with the Hamming distance, we do not see diversity ever rise over \num{0.5}, even for the Ancestor Elitism replacement rule, which consistently introduces many randomly initialized individuals. 

\begin{figure*}
  \centering
  \begin{subfigure}[b]{0.25\textwidth}
    \begin{tikzpicture}
      \begin{axis}[
          dynamic,
          dynamic-root,
          legend to name=leafDynamic,
          title={\dia{}},
        ]
        \addplot[mark=*, smooth, color=black]
        table[ x index=0, y index=3] {data/replacementrule_game/greedy_leaf.csv};
        \addplot[mark=square*, smooth, color=blue]
        table[x index=0, y index=3] {data/replacementrule_game/aerr_leaf.csv};
        \addplot[mark=triangle*, smooth, color=red]
        table[x index=0, y index=3] {data/replacementrule_game/sperr_leaf.csv};
        \addplot[mark=diamond*, smooth, color=green]
        table[x index=0, y index=3] {data/replacementrule_game/eerr_leaf.csv};
      \end{axis}
    \end{tikzpicture}
  \end{subfigure}%
  ~
  \begin{subfigure}[b]{0.25\textwidth}
    \begin{tikzpicture}
      \begin{axis}[
          dynamic,
          title=Fitness-based,
        ]
        \addplot[mark=*, smooth, color=black]
        table[x index=0, y index=1] {data/replacementrule_game/greedy_leaf.csv};
        \addplot[mark=square*, smooth, color=blue]
        table[x index=0, y index=1] {data/replacementrule_game/aerr_leaf.csv};
        \addplot[mark=triangle*, smooth, color=red]
        table[x index=0, y index=1] {data/replacementrule_game/sperr_leaf.csv};
        \addplot[mark=diamond*, smooth, color=green]
        table[x index=0, y index=1] {data/replacementrule_game/eerr_leaf.csv};
      \end{axis}
    \end{tikzpicture}
  \end{subfigure}%
  ~
  \begin{subfigure}[b]{0.25\textwidth}
    \begin{tikzpicture}
      \begin{axis}[
          dynamic,
          title  = Hamming distance,
        ]
        \addplot[mark=*, smooth, color=black]
        table[x index=0, y index=2] {data/replacementrule_game/greedy_leaf.csv};
        \addplot[mark=square*, smooth, color=blue]
        table[x index=0, y index=2] {data/replacementrule_game/aerr_leaf.csv};
        \addplot[mark=triangle*, smooth, color=red]
        table[x index=0, y index=2] {data/replacementrule_game/sperr_leaf.csv};
        \addplot[mark=diamond*, smooth, color=green]
        table[x index=0, y index=2] {data/replacementrule_game/eerr_leaf.csv};
      \end{axis}
    \end{tikzpicture}
  \end{subfigure}%
  ~
  \begin{subfigure}[b]{0.25\textwidth}
    \begin{tikzpicture}
      \begin{axis}[
          fitness
        ]
        \addplot[mark=*, smooth, color=black]
        table[x index=0, y index=4] {data/replacementrule_game/greedy_leaf.csv};
        \addplot[mark=square*, smooth, color=blue]
        table[x index=0, y index=4] {data/replacementrule_game/aerr_leaf.csv};
        \addplot[mark=triangle*, smooth, color=red]
        table[x index=0, y index=4] {data/replacementrule_game/sperr_leaf.csv};
        \addplot[mark=diamond*, smooth, color=green]
        table[x index=0, y index=4] {data/replacementrule_game/eerr_leaf.csv};
      \end{axis}
    \end{tikzpicture}
  \end{subfigure}
  \ref{leafDynamic}
  \caption{Average over \num{100} runs for each diversity measure on the leaf data set.}\label{fig:dynamic-leaf}
\end{figure*}


\paragraph{Results of XOR} The results of each diversity measure presented in \cref{fig:dynamic-xor} are very similar to the Leaf data set above. Once again, we see the irregular spike in the plot presenting the fitness-based measure. %Furthermore, it is interesting how \dia{} and Hamming distance are a constant factor apart for each replacement rule.%  Also, for this data set, the fitness-based measurement has an overall higher diversity compared to the other two data sets.%This means that for this experiment, the fitness-based measurement is closer to \dia{} than Hamming distance, as it is for the other results.

\begin{figure*}
  \centering
  \begin{subfigure}[b]{0.25\textwidth}
    \batchmode
    \begin{tikzpicture}
      \begin{axis}[
          dynamic,
          dynamic-root,
          title={\dia{}},
          legend to name=xorDynamic,
        ]
        \addplot[mark=*, smooth, color=black]
        table[x index=0, y index=3] {data/replacementrule_game/greedy_xor.csv};
        \addplot[mark=square*, smooth, color=blue]
        table[x index=0, y index=3] {data/replacementrule_game/aerr_xor.csv};
        \addplot[mark=triangle*, smooth, color=red]
        table[x index=0, y index=3] {data/replacementrule_game/sperr_xor.csv};
        \addplot[mark=diamond*, smooth, color=green]
        table[x index=0, y index=3] {data/replacementrule_game/eerr_xor.csv};
      \end{axis}
    \end{tikzpicture}
    \scrollmode
  \end{subfigure}%
  ~
  \begin{subfigure}[b]{0.25\textwidth}
    \batchmode
    \begin{tikzpicture}
      \begin{axis}[
          dynamic,
          title=Fitness-based,
        ]
        \addplot[mark=*, smooth, color=black]
        table[x index=0, y index=1] {data/replacementrule_game/greedy_xor.csv};
        \addplot[mark=square*, smooth, color=blue]
        table[x index=0, y index=1] {data/replacementrule_game/aerr_xor.csv};
        \addplot[mark=triangle*, smooth, color=red]
        table[x index=0, y index=1] {data/replacementrule_game/sperr_xor.csv};
        \addplot[mark=diamond*, smooth, color=green]
        table[x index=0, y index=1] {data/replacementrule_game/eerr_xor.csv};
      \end{axis}
    \end{tikzpicture}
    \scrollmode
  \end{subfigure}%
  ~
  \begin{subfigure}[b]{0.25\textwidth}
    \batchmode
    \begin{tikzpicture}
      \begin{axis}[
          dynamic,
          title  = Hamming distance,
        ]
        \addplot[mark=*, smooth, color=black]
        table[x index=0, y index=2] {data/replacementrule_game/greedy_xor.csv};
        \addplot[mark=square*, smooth, color=blue]
        table[x index=0, y index=2] {data/replacementrule_game/aerr_xor.csv};
        \addplot[mark=triangle*, smooth, color=red]
        table[x index=0, y index=2] {data/replacementrule_game/sperr_xor.csv};
        \addplot[mark=diamond*, smooth, color=green]
        table[x index=0, y index=2] {data/replacementrule_game/eerr_xor.csv};
      \end{axis}
    \end{tikzpicture}
    \scrollmode
  \end{subfigure}%
  ~
  \begin{subfigure}[b]{0.25\textwidth}
    \batchmode
    \begin{tikzpicture}
      \begin{axis}[
          fitness,
        ]
        \addplot[mark=*, smooth, color=black]
        table[x index=0, y index=4] {data/replacementrule_game/greedy_xor.csv};
        \addplot[mark=square*, smooth, color=blue]
        table[x index=0, y index=4] {data/replacementrule_game/aerr_xor.csv};
        \addplot[mark=triangle*, smooth, color=red]
        table[x index=0, y index=4] {data/replacementrule_game/sperr_xor.csv};
        \addplot[mark=diamond*, smooth, color=green]
        table[x index=0, y index=4] {data/replacementrule_game/eerr_xor.csv};
      \end{axis}
    \end{tikzpicture}
    \scrollmode
  \end{subfigure}
  \ref{xorDynamic}
  \caption{Average diversity over \num{100} runs for each replacement rule over \num{2000} generations for the XOR logical operator.}\label{fig:dynamic-xor}
\end{figure*}


\paragraph{Results of Snake} The results shown in \cref{fig:dynamic-snake}, once again show a constant difference between the \dia{} and Hamming distance measures. The fitness-based measure is again very irregular, this time for the Ancestor Elitism replacement rule. Here, the plot falls and rises before it stabilises.

\begin{figure*}
  \centering
  \begin{subfigure}[b]{0.25\textwidth}
    \batchmode
    \begin{tikzpicture}
      \begin{axis}[
          dynamic-root,
          dynamic,
          title={\dia{}},
          legend to name=snakeDynamic,
        ]
        \addplot[mark=*, smooth, color=black]
        table[x index=0, y index=3] {data/replacementrule_game/greedy_snake.csv};
        \addplot[mark=square*, smooth, color=blue]
        table[x index=0, y index=3] {data/replacementrule_game/aerr_snake.csv};
        \addplot[mark=triangle*, smooth, color=red]
        table[x index=0, y index=3] {data/replacementrule_game/sperr_snake.csv};
        \addplot[mark=diamond*, smooth, color=green]
        table[x index=0, y index=3] {data/replacementrule_game/eerr_snake.csv};
      \end{axis}
    \end{tikzpicture}
    \scrollmode
  \end{subfigure}%
  ~
  \begin{subfigure}[b]{0.25\textwidth}
    \batchmode
    \begin{tikzpicture}
      \begin{axis}[
          dynamic,
          title=Fitness-based,
        ]
        \addplot[mark=*, smooth, color=black]
        table[x index=0, y index=1] {data/replacementrule_game/greedy_snake.csv};
        \addplot[mark=square*, smooth, color=blue]
        table[x index=0, y index=1] {data/replacementrule_game/aerr_snake.csv};
        \addplot[mark=triangle*, smooth, color=red]
        table[x index=0, y index=1] {data/replacementrule_game/sperr_snake.csv};
        \addplot[mark=diamond*, smooth, color=green]
        table[x index=0, y index=1] {data/replacementrule_game/eerr_snake.csv};
      \end{axis}
    \end{tikzpicture}
    \scrollmode
  \end{subfigure}%
  ~
  \begin{subfigure}[b]{0.25\textwidth}
    \batchmode
    \begin{tikzpicture}
      \begin{axis}[
          dynamic,
          title  = Hamming distance,
        ]
        \addplot[mark=*, smooth, color=black]
        table[x index=0, y index=2] {data/replacementrule_game/greedy_snake.csv};
        \addplot[mark=square*, smooth, color=blue]
        table[x index=0, y index=2] {data/replacementrule_game/aerr_snake.csv};
        \addplot[mark=triangle*, smooth, color=red]
        table[x index=0, y index=2] {data/replacementrule_game/sperr_snake.csv};
        \addplot[mark=diamond*, smooth, color=green]
        table[x index=0, y index=2] {data/replacementrule_game/eerr_snake.csv};
      \end{axis}
    \end{tikzpicture}
    \scrollmode
  \end{subfigure}%
  ~
  \begin{subfigure}[b]{0.25\textwidth}
    \batchmode
    \begin{tikzpicture}
      \begin{axis}[
          fitness,
        ]
        \addplot[mark=*, smooth, color=black]
        table[x index=0, y index=4] {data/replacementrule_game/greedy_snake.csv};
        \addplot[mark=square*, smooth, color=blue]
        table[x index=0, y index=4] {data/replacementrule_game/aerr_snake.csv};
        \addplot[mark=triangle*, smooth, color=red]
        table[x index=0, y index=4] {data/replacementrule_game/sperr_snake.csv};
        \addplot[mark=diamond*, smooth, color=green]
        table[x index=0, y index=4] {data/replacementrule_game/eerr_snake.csv};
      \end{axis}
    \end{tikzpicture}
    \scrollmode
  \end{subfigure}
  \ref{snakeDynamic}
  \caption{Average over \num{100} runs for each diversity measure on the snake data set.}\label{fig:dynamic-snake}
\end{figure*}


\paragraph{The big picture} If we compare experiments run on each data set illustrated in figures \ref{fig:dynamic-leaf} through \ref{fig:dynamic-snake}, we see that the fitness-based diversity measure stands out from the other two in each of the experiments. For example, the fitness-based plots have some irregular curves, which are not easily explained. Because of this, combined with the fact that multiple individuals can have the same fitness yet behave very differently, we argue that fitness-based diversity does not accurately represent the actual diversity. This measure is simple and does not need any extra computations, but it is not suitable for every problem. 

Another observation from the three experiments is that \dia{} and Hamming distance do not vary much between the different problems. Each replacement rule is always around the same interval for these two measures. Fitness-based diversity is much more dependent on the problem, as its results are not regular. In each figure presenting the dynamic experiments, the plot presenting the fitness-based measurement is notably different from the two other plots, presenting \dia{} and Hamming distance.

The experiments show that there might be a correspondence between the phenotypic \dia{} diversity measure and the genotypic Hamming distance diversity measure. If we take a closer look at the plots illustrating \dia{} and Hamming distance, we can see that they are scaled by a constant, compared to each other. The Hamming distance is always below the \dia{} measurement.% So, the \dia{} shows the differences in the behaviour, and the Hamming distance shows the difference in the genetic make-up.  

When it comes to the fitness of the three experiments, it is clear to see that fitness is at its best when using the MEEE replacement rule. This rule does not have the most diverse population, but as we see, this is not necessarily a bad thing. This indicates that it is not the best solution to aim for a highly diverse population, as stated in~\cite{Darwen00doesextra}. One must find a balanced solution and take advantage of the fact that it is possible to switch between maintaining a high diversity and not doing so, which is in correspondance with the results presented in \cite{Darwen00doesextra}.

