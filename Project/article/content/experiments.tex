\section{Experiments}
%In this section, we evaluate our proposed method of a better diversity measurement (NNTD) by using it to measure diversity of the replacement rules described in \cref{sec:replacementrules}. This is to compare each replacement rules' influence on diversity using NNTD.
Three problem instances are used to evaluate our diversity measure. Two problems discrete, and one continuous. A standard implementation of GAs, ANNs, replacement policies, and diversity measures is used (see~\cite{mbm:kmc:ekoGA} for the authors' implementation details). 

\subsection{Parameter settings}

\begin{table}
  \centering
  \begin{tabular}{l S}
    \toprule
    Parameter & {Specification} \\
    \midrule
    Number of runs & 100 \\
    Generations per run (continuous) & 2000 \\
    Generations per run (discrete) & 500 \\
    Population size & 100 \\
    Selection method & {rank-based} \\
    \bottomrule
  \end{tabular}
  \caption{GA parameters used throughout experimenting.}
  \label{tab:gaparam}
\end{table}

We use a population size of 100. For each iteration, 100 new offspring individuals are created. Half of the offspring is cloned from a random parent, found using rank-based selection, and then mutated. The other half is made by performing crossover between two random parents, also using rank-based selection.
Three crossover methods are used: one-point crossover, two-point crossover, and uniform crossover. When a new individual is created using crossover, there is an equal chance for any of the crossover methods to be used. 
Each offspring individual created by crossover, has a \perc{10} chance to be mutated.
When mutating an individual, each bit in its bit string has a \perc{5} chance to be assigned a random boolean value.

\subsection{Problems}
\label{sec:problems}
We hereby explain the three discrete problems we perform our experiments on.

\subsubsection{XOR}
We use a neural network to approximate the XOR between two 8-bit strings.
To evaluate fitness of a neural network, we choose 1000 random instances of two 8-bit strings.
For each instance, we determine how many bits the neural network calculates correct.
The fitness is the defined as the average number of bits calculated correct.
Fitness values will thus lie in the range $[0-8]$.
Since the XOR between two random bits will be evenly distributed between 0 and 1,
randomly guessing a solution to the 8-bit XOR problem, is expected to yield a fitness of 4.
The network has 16 input neurons, 16 hidden neurons, 8 output neurons, 2 bits per weight, and 1 bit per bias for each hidden and output neuron.
Any output neuron $i$ represents the XOR between the two input neurons $i$ and $i+8$. 
We have used as few neurons and bits to represent weights and biases as possible, while still being able to verify that the maximum fitness value of 8 is achievable.
The same random seed is always used for generating the 1000 problem instances.

\subsubsection{Leaf classification}
We use a neural network to classify the Leaf data set~\cite{Bache+Lichman:2013, leafdataset}.
The neural network is given 16 properties about an unknown leaf and has to decide which of 40 types of leaves it is.
Fitness is evaluated based on how many instances out of the entire data set the neural network correctly classifies.
The implementation consists of 16 input neurons, 10 hidden and 40 output neurons. The output neuron with the highest value decides the classification. Each weight is encoded by 9 bits and neurons have no bias.

\subsubsection{Snake}
Snake is a game found on old Nokia cell phones, where you are moving a snake around a grid to pick up pieces of food.
Every time a piece of food is collected, both the length of the snake and your score increases by 1.
You loose if the snake head hits its body or one of the edges of the grid.
At all times, the grid contains only a single piece of food.
The game becomes harder as the length of the snake increases, and as the snake is constantly moving, it becomes hard to not trap yourself.

We use a neural network to play a game of Snake in a $10\times10$ grid with an initial snake length of 5 units.
We have defined the fitness of a neural network to be $f + \frac{f}{s/1000}$,
where $f$ is the amount of food it collects and $s$ is the total number of steps the snake is alive. The game is constrained such that the snake can only change its direction \SI{90}{\degree} per step. The neural network has 6 input neurons, each receiving a bit of information about the game state:
\begin{enumerate}
  \item \set{-1, 0, 1} Whether food is to the left, vertically aligned, or to the right of the snake head.
  \item \set{-1, 0, 1} Whether food is above, horizontally aligned, or below the snake head.
  \item \set{0, 1} If the snake dies if its next move is up
  \item \set{0, 1} If the snake dies if its next move is down
  \item \set{0, 1} If the snake dies if its next move is right 
  \item \set{0, 1} If the snake dies if its next move is left
\end{enumerate}
4 output neurons are used. The neuron with the highest value determines whether to move the snake right, left, up, or down.
The neural network uses 5 hidden neurons, 9 bits per weight and neurons have no bias.
For every game of Snake, we always use the same random seed to decide the positions where pieces of food will spawn.

\subsubsection{Criteria for selection}
The XOR function is interesting since it is the simplest boolean function that is not linearly separable.
This fact has made it quite popular in NN research communities\cite{masterThesisGANN}.
A classification problem like Leaf is interesting because it is radically different from the XOR problem.
XOR is a simple and well defined function, whereas classifying leaves is more complex and depends on observations in nature, which may contain noise.
The Snake game differs in that it is an agent decision problem, where not just a single, but a sequence of decisions determines the outcome, where each decision changes the intermediate state.

\subsection{Diversity measurements}
To test our NNTD method, we introduce two variables, both of which only have an effect during the very first generation of a population: initial similarity and initial mutation of a population. 

Initial similarity describes how equal the individuals initially are. It can take on a range between \num{0} and \num{1}, where \num{0} means that all individuals are random, and \num{1} denotes that \perc{100} all individuals have the exact same genotype. With a value of \num{0.5}, one individual is cloned so that half of the population has the same genetic makeup and the other half is initialized to be completely random. A population with the value of \num{1} for initial similarity should have the lowest diversity, and a completely randomly initialize population (initial similarity of \num{0}) should have the highest diversity, since every individual is randomly chosen.

Initial mutation requires an initial similarity set to \num{1}, e.g.\ the initial population consists of a single, cloned individual. This is due to the fact that mutating individuals already randomly initialized will not yield a different diversity of individuals. The initial mutation rate simply signifies the mutation percentage of all bits in every individual of the population. A population consisting of the same cloned individual with an initial mutation rate of \num{1} should have the highest diversity, since every bit in every, initially similar individual, will have a \perc{100} chance to be selected for mutation, resulting in a randomly initialized population.

Experiments on NNTD using both of these two variables are shown in \cref{fig:initial-mutation-similarity}. Each line represents the average diversity measure of \num{100} runs at a variable intervals of \num{0.05}. As we can see by the chart, diversity increases as expected for both variables. Diversity is maximum at a value of \num{7.2} for a completely random population. To reach this maximum, it requires an initial mutation rate of a mere \perc{2} and an initial similarity of \perc{0}.

\begin{figure}[htpb]
  \centering
  \inputresize{drawings/initial-mutation-similarity/graph}
  \caption{Diversity measurements with NNTD, given increased ranges of initial similarity and initial mutation rates.}\label{fig:initial-mutation-similarity}
\end{figure}

\begin{algorithm}
  \caption{Procedure AESP}\label{alg:aesp}
  \begin{algorithmic}[1]
  \Procedure{aesp}{$G_k, O_k$}
    \State $G_{k+1} \gets G_k$
    \ForAll{$o \in O_k$}
      \If {$o.\mathtt{num\_parents} = 1$}
        \If {$\beta(o) > \beta(o.p_1)$}
          \State $G_{k+1} \gets \left(G_{k+1} \setminus \set{o.p_1}\right) \cup \set{o}$
        \EndIf
      \Else
        \If {$\beta(o) > \max\left(\beta(o.p_1), \beta(o.p_2)\right)$}
        \State  $G_{k+1} \gets$ \begin{varwidth}[t]{\linewidth}$(G_{k+1} \setminus \{o.p_1, o.p_2\})$\par
          \hskip\algorithmicindent $\cup \set{o} \cup \set{\rho}$
        \end{varwidth}
        \EndIf
      \EndIf
    \EndFor
  \EndProcedure
  \end{algorithmic}
\end{algorithm}


\todo{Make sure to mention we have used \dia{} as the diversity measure for Explore-exploit}
