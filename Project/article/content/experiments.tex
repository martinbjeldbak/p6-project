\section{Experiments}
%In this section, we evaluate our proposed method of a better diversity measurement (NNTD) by using it to measure diversity of the replacement rules described in \cref{sec:replacementrules}. This is to compare each replacement rules' influence on diversity using NNTD.
Three problems instances are used to evaluate our diversity measure. Two problems discrete, and one continuous. A standard implementation of GAs, ANNs, replacement policies, and diversity measures is used (see~\cite{mbm:kmc:ekoGA} for the authors' implementation details). 

\subsection{Parameter settings}

\begin{table}
  \centering
  \begin{tabular}{l S}
    \toprule
    Parameter & {Specification} \\
    \midrule
    Number of runs & 100 \\
    Generations per run (continuous) & 2000 \\
    Generations per run (discrete) & 500 \\
    Population size & 100 \\
    Selection method & {rank-based} \\
    \bottomrule
  \end{tabular}
  \caption{GA parameters used throughout experimenting.}
  \label{tab:gaparam}
\end{table}

Three crossover methods are used: one-point crossover, two-point crossover, and uniform crossover. Upon creating a new population, there is an equal chance for each to be chosen.

Half of the offspring is created from crossover between two selected parents in population $p$, the other half created solely by mutation of the individuals also in $p$. Of the individuals in $p$'s offspring belonging to the half created by crossover, there is a \perc{10} chance that the individual also will be selected for mutation. For each bit belonging to the individuals selected for mutation, there is a \perc{5} chance that that will be mutated. Of the bits to be mutated, each bit has a \perc{50} chance it will remain the same, and a \perc{50} chance to be inverted.

\subsection{Problems}
% Discrete: Leaf
The discrete problem is the small leaf data set~\cite{Bache+Lichman:2013, leafdataset}. Fitness is evaluated based on how many instances out of the entire data set the neural network correctly classifies. It consists of 16 inputs and 1 output.

% Discrete: Rosenbrock 
%Approximating the 2D Rosenbrock function is the second discrete problem. Here fitness is computed as the average distance to the actual output of the Rosenbrock function, over \num{10000} random inputs. The individuals have two input neurons and one output.

% Continuous: Snake
The continuous problem used is the Snake game known from Nokia cell phones. This game is very similar to the artificial ant problem~\cite[p.\ 147--155]{koza1992genetic}. A snake has to traverse a square $10\times10$ grid and search for a randomly positioned piece of food using a sensing function that's able to see the squares and their contents directly next to its head and above and below its head. Once the snake traverses into the square with food, the food disappears and the snake's length grows by one square. A new piece of food then spawns at a new random position. The snake has four actions to move around the grid, each of which takes one time unit: move up, down, left, and right. The game ends if the amount of time units exeeds \num{1000}, or if the snake collides into a wall. Its fitness function is based upon the amount of food eaten.

\subsection{Diversity measurements}
To test our NNTD method, we introduce two variables, both of which only have an effect during the very first generation of a population: initial similarity and initial mutation of a population. 

Initial similarity describes how equal the individuals initially are. It can take on a range between \num{0} and \num{1}, where \num{0} means that all individuals are random, and \num{1} denotes that \perc{100} all individuals have the exact same genotype. With a value of \num{0.5}, one individual is cloned so that half of the population has the same genetic makeup and the other half is initialized to be completely random. A population with the value of \num{1} for initial similarity should have the lowest diversity, and a completely randomly initialize population (initial similarity of \num{0}) should have the highest diversity, since every individual is randomly chosen.

Initial mutation requires an initial similarity set to \num{1}, e.g.\ the initial population consists of a single, cloned individual. This is due to the fact that mutating individuals already randomly initialized will not yield a different diversity of individuals. The initial mutation rate simply signifies the mutation percentage of all bits in every individual of the population. A population consisting of the same cloned individual with an initial mutation rate of \num{1} should have the highest diversity, since every bit in every, initially similar individual, will have a \perc{100} chance to be selected for mutation, resulting in a randomly initialized population.

Experiments on NNTD using both of these two variables are shown in \cref{fig:initial-mutation-similarity}. Each point represents the average diversity measure of \num{100} runs at a variable intervals of \num{0.05}. As we can see by the chart, diversity increases as expected for both variables. Diversity is maximum at a value of \num{7.2} for a completely random population. To reach this maximum, it requires an initial mutation rate of a mere \perc{2} and an initial similarity of \perc{0}.

\begin{figure}[htpb]
  \centering
  \inputresize{drawings/initial-mutation-similarity/graph}
  \caption{Diversity measurements with NNTD, given increased ranges of initial similarity and initial mutation rates. Each point is the average diversity over \num{100} runs.}\label{fig:initial-mutation-similarity}
\end{figure}

\begin{algorithm}
  \caption{Procedure AESP}\label{alg:aesp}
\begin{algorithmic}%[1]
  \Procedure{AESP}{$G_k, O_k$}
  \State $G_{k+1} \gets G_k$
  \ForAll{$o \in O_k$}
    \If {$o.\mathtt{num\_parents} = 1$}
      \If {$\beta(o) > \beta(\pi_1(o))$}
        \State $G_{k+1} \gets (G_{k+1} \setminus \set{\pi_1(o)}) \cup \set{o}$
      \EndIf
    \Else
      \If {$\beta(o) > \mathtt{max}(\beta(\pi_1(o)), \beta(\pi_2(o)))$}
      \State $\mathtt{tmp} \gets (G_{k+1} \setminus \{\pi_1(o), \pi_2(o)\})$ 
        \State $G_{k+1} \gets \mathtt{tmp} \cup \set{o} \cup \set{\rho}$
      \EndIf
    \EndIf
  \EndFor
\EndProcedure
\end{algorithmic}
\end{algorithm}


\todo{Make sure to mention we have used \dia{} as the diversity measure for Explore-exploit}
