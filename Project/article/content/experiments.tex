\section{Experiments}
In this section, we evaluate our proposed method of a better diversity measurement (NNTD), and compare our Ancestor Elitism Replacement Rule to a standard naive replacement rule as well as the probabilistic crowding replacement rule.

Each of the following experiments use the same configuration of certain variables. Populations consist of \num{100} individuals, and upon creating a new population, rank-based selection is utilized on the current population $p$ to select individuals for crossover and mutation to create offspring. Half of the offspring is created from crossover between two selected parents in population $p$, the other half created solely by mutation of the individuals also in $p$. Of the individuals in $p$'s offspring belonging to the half created by crossover, there is a \perc{10} chance that the individual also will be selected for mutation. For each bit belonging to the individuals selected for mutation, there is a \perc{5} chance that that will be mutated. Of the bits to be mutated, each bit has a \perc{50} chance it will remain the same, and a \perc{50} chance to be inverted.

\subsection{Diversity measurements}
To test our NNTD method, we introduce two variables, both of which only have an effect during the very first generation of a population: initial similarity and initial mutation of a population. 

Initial similarity describes how equal the individuals initially are. It can take on a range between \num{0} and \num{1}, where \num{0} means that all individuals are random, and \num{1} denotes that \perc{100} all individuals have the exact same genotype. With a value of \num{0.5}, one individual is cloned so that half of the population has the same genetic makeup and the other half is initialized to be completely random. A population with the value of \num{1} for initial similarity should have the lowest diversity, and a completely randomly initialize population (initial similarity of \num{0}) should have the highest diversity, since every individual is randomly chosen.

Initial mutation requires an initial similarity set to \num{1}, e.g.\ the initial population consists of a single, cloned individual. This is due to the fact that mutating individuals already randomly initialized will not yield a different diversity of individuals. The initial mutation rate simply signifies the mutation percentage of all bits in every individual of the population. A population consisting of the same cloned individual with an initial mutation rate of \num{1} should have the highest diversity, since every bit in every, initially similar individual, will have a \perc{100} chance to be selected for mutation, resulting in a randomly initialized population.

Experiments on NNTD using both of these two variables are shown in \cref{fig:initial-mutation-similarity}. Each line represents the average diversity measure of \num{100} runs at a variable intervals of \num{0.05}. As we can see by the chart, diversity increases as expected for both variables. Diversity is maximum at a value of \num{7.2} for a completely random population. To reach this maximum, it requires an initial mutation rate of a mere \perc{2} and an initial similarity of \perc{0}.

\begin{figure}[htpb]
  \centering
  \inputresize{drawings/initial-mutation-similarity/graph}
  \caption{Diversity measurements with NNTD, given increased ranges of initial similarity and initial mutation rates.}\label{fig:initial-mutation-similarity}
\end{figure}

\begin{algorithm}
  \caption{Procedure AESP}\label{alg:aesp}
  \begin{algorithmic}[1]
  \Procedure{aesp}{$G_k, O_k$}
    \State $G_{k+1} \gets G_k$
    \ForAll{$o \in O_k$}
      \If {$o.\mathtt{num\_parents} = 1$}
        \If {$\beta(o) > \beta(o.p_1)$}
          \State $G_{k+1} \gets \left(G_{k+1} \setminus \set{o.p_1}\right) \cup \set{o}$
        \EndIf
      \Else
        \If {$\beta(o) > \max\left(\beta(o.p_1), \beta(o.p_2)\right)$}
        \State  $G_{k+1} \gets$ \begin{varwidth}[t]{\linewidth}$(G_{k+1} \setminus \{o.p_1, o.p_2\})$\par
          \hskip\algorithmicindent $\cup \set{o} \cup \set{\rho}$
        \end{varwidth}
        \EndIf
      \EndIf
    \EndFor
  \EndProcedure
  \end{algorithmic}
\end{algorithm}

