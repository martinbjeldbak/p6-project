\section{\di{}}
We believe it is essential that a diversity measure reflects the difference in traits among individuals, and that to the best of our knowledge, no current genotypic or phenotypic measures reflect this. Recall that we are only concerned with neural networks as individuals. Since a \emph{trait} is a rather vague term, we introduce a clear definition of traits among neural networks: ``\emph{two neural networks have different traits if they for some input produce different outputs}''. By this definition, we denote the difference in traits among neural networks.

Fitness-based diversity measures do not catch this trait diversity. Consider two artificial intelligent (AI) players for the cell phone game ``Snake''. Their fitness can be calculated based on how much food they collect before they die. One AI player may have traits that makes it good at avoiding death by not hitting any walls or its own body. Sometimes, by chance, it hits a piece of food. The other AI player may have traits that makes it good at searching for food. The two AI players can have the same fitness value, because they collect the same amount of food before they die, yet still have completely different traits. Genotypic diversity measures do not catch the trait diversity either. This is because two individuals of different genotypes can yield the same output on any input. An example of this are the two neural networks shown in \cref{fig:entire-eqnetwork}. No matter what input they receive, their output will always be the same. They are genotypically very diverse (Hamming distance of 6), but are not trait diverse.
%
\begin{figure*}
  \begin{subfigure}{0.5\textwidth}
    \centering
    \inputresizeto{0.5\linewidth}{drawings/eqnetworks/eqnetworks3}
    \caption{An artificial neural network with connections and weights.}\label{fig:eqnetwork}
  \end{subfigure}
  \begin{subfigure}{0.5\textwidth}
    \centering
    \inputresizeto{0.5\linewidth}{drawings/eqnetworks/eqnetworks4}
    \caption{An artificial neural network equivalent to \cref{fig:eqnetwork}.}\label{fig:eqnetwork2}
  \end{subfigure}
  \caption{Networks with same phenotype, but different genotypes. The binary representation assumes that each weight is represented by four bits.}\label{fig:entire-eqnetwork}
\end{figure*}

%
To the best of our knowledge, no diversity measure exists that catches this definition of trait diversity. In the following, we propose a method for measuring trait diversity which we call \emph{\di{}} (\dia). \dia{} aims to reflect the diversity of different traits among individuals. 

\subsection{Algorithm}
\dia{} is based on Simpsons Diversity Index (SDI), which is a diversity measure used in ecology to quantifiy the biodiversity of a habitat~\cite{simpson1949measurement}. We modify SDI to work with neural network. We will thus use the term individual and neural network interchangeably.

Let $F = \set{f_1, f_2, \dots, f_n}$, where each $f_j$ for $0 \leq j \leq n$ denote the set of neural networks contained in a population, which all have the same architecture of $a$ input and $b$ output neurons. \dia{} is calculated with respect to a number of random inputs $\set{R_1, R_2, \dots, R_m}$, where each $R_i$ for $1 \leq i \leq m$ is an $a$-tuple of real values chosen randomly. For each sample $R_i$, a diversity is calculated. The calculation depends on a set of species, so we distribute the neural networks into species based on which of their output neurons yields the highest value on input $R_i$. This means that for any neural network belonging to species $S_k(R_i)$, the value of the $k$th output neuron will be greater than or equal to the value of any other output neuron given input $R_i$.% This definition implies that the number of species equals the number of output neurons $b$, e.g.\ $1 \leq k \leq b$. We distribute neural networks into a set of species as follows
%
\begin{equation*}\label{eq:species}
  S_k\left(R_i\right) = \setof{f_j}{\forall l \in \set{1, 2, \dots, b} \left(\sigma_{kji} \geq \sigma_{lji}\right)}
\end{equation*}
%

where $\sigma_{xyz} \in \set{0, 1}$ is the output of the $x$th neuron in neural network $f_y$ on input $R_z$.

We calculate the diversity $m$ times with \dia{} as
%
\begin{equation*}\label{eq:sdi}
  D_i = 1 - \left(\frac{\sum_{j=1}^{k}\lvert S_j\left(R_i\right)\rvert\left(\lvert S_j\left(R_i\right)\rvert - 1\right)}{\lvert F\rvert \left(\lvert F\rvert - 1\right)}\right) 
\end{equation*}
%
where $\lvert S_j(R_i)\rvert$ is the total number of individuals of the $j$th species, with respect to random input $R_i$, and $k$ is the maximum number of species. $F$ denotes the total number of neural networks in the population. 

The \dia{} $D$ (that is, the actual diversity) is calculated as the average of all $m$ $D_i$ values
%
\[D =\frac{\sum_{i=1}^m{D_i}}{m}\]
%
The species classification implies that a single network can only be part of one species per input. One disadvantage of \dia{} is that it relies on random inputs, which means that fewer random inputs implies less statistical significance.
