\section{\di{}}
We believe it is essential that a diversity measure reflects the difference in traits among individuals. To the best of our knowledge, no current genotypic or phenotypic measures reflect this. Recall, that we are only concerned with neural networks as individuals. Since a \emph{trait} is a rather vague term, we introduce a clear definition of traits among neural networks: ``\emph{two neural networks have different traits if they for some input produce different outputs}''. By this definition, we denote the difference in traits among neural networks.

Fitness-based diversity measures do not catch this trait diversity. Consider two artificial intelligent (AI) players for the cell phone game ``Snake''. Their fitness can be calculated based on how much food they collect before they die. One AI player may have traits that makes it good at avoiding death by not hitting any walls or its own body. Sometimes, by chance, it hits a piece of food. The other AI player may have traits that makes it good at searching for food. The two AI players can have the same fitness value, because they collect the same amount of food before they die, yet still have completely different traits. Genotypic diversity measures do not catch the trait diversity either. This is because two individuals of different genotypes can yield the same output on any input. An example of this are the two neural networks shown in \cref{fig:entire-eqnetwork}. No matter what input they receive, their output will always be the same. They are genotypically very diverse (Hamming distance of 6), but are not trait diverse.
%
\begin{figure*}
  \begin{subfigure}{0.5\textwidth}
    \centering
    \inputresizeto{0.5\linewidth}{drawings/eqnetworks/eqnetworks3}
    \caption{An artificial neural network with connections and weights.}\label{fig:eqnetwork}
  \end{subfigure}
  \begin{subfigure}{0.5\textwidth}
    \centering
    \inputresizeto{0.5\linewidth}{drawings/eqnetworks/eqnetworks4}
    \caption{An artificial neural network equivalent to \cref{fig:eqnetwork}.}\label{fig:eqnetwork2}
  \end{subfigure}
  \caption{Networks with same phenotype, but different genotypes. The binary representation assumes that each weight is represented by four bits.}\label{fig:entire-eqnetwork}
\end{figure*}

%

In the following, we propose a method for measuring trait diversity, which we call \emph{\di{}} (\dia). \dia{} aims to reflect the diversity of different traits among individuals. 

\subsection{Algorithm}
\dia{} is based on Simpsons Diversity Index (SDI), which is a diversity measure used in ecology to quantifiy the biodiversity of a habitat~\cite{simpson1949measurement}. We modify SDI to work with neural network. We will thus use the term individual and neural network interchangeably.

Let $F = \set{f_1, f_2, \dots, f_n}$, where each $f_j$ for $0 \leq j \leq n$ denote the set of neural networks contained in a population, which all have the same architecture of $a$ input and $b$ output neurons. \dia{} is calculated with respect to a number of random inputs $\set{R_1, R_2, \dots, R_m}$, where each $R_i$ for $1 \leq i \leq m$ is an $a$-tuple of real values chosen randomly. For each sample $R_i$, a diversity is calculated. The calculation depends on a set of species, so we distribute the neural networks into species based on which of their output neurons yields the highest value on input $R_i$. 

If $b_nb_{n-1}\dots b_1$ is the binary representation of a number $i$,
we define the species $S_i$ to contain any neural network satisfying
%
\[
  \forall o_j\forall o_k \in O \left(b_j \rightarrow \left(o_j \geq o_k\right)\right)
\]
%
where $O$ is the set of all output neurons.
In a population where each individual has $b$ output neurons, the total number of species is $2^b - 1$.  As an example, assume that we have a neural network with six output neurons. If the third output neuron has the highest output, then that neural network is placed in species $S_4$, because the bits set to 1 correspond to \texttt{000100} (4 in binary). If another individual has multiple output neurons giving the highest output, say the first and third output neuron, then that neural network will be placed in species $S_5$ (\texttt{000101}, 5 in binary). If a third neural network has all of its output neurons give the same result, then that neural network is placed in $S_{63}$. Any species that is empty, does not count as a species in the diversity measurement.

Diversity with respect to random input $i$ is then defined by
%
\begin{equation*}\label{eq:sdi}
  D_i = 1 - \left(\frac{\sum_{j=1}^{k}\lvert S_j\left(R_i\right)\rvert\left(\lvert S_j\left(R_i\right)\rvert - 1\right)}{\lvert F\rvert \left(\lvert F\rvert - 1\right)}\right) 
\end{equation*}
%
where $\lvert S_j(R_i)\rvert$ is the total number of individuals of the $j$th species, and $k$ is the total number of non-empty species, with respect to random input $R_i$. %, and $k$ is the number of non-empty species. 
$F$ denotes the total number of neural networks in the population. 

The \dia{} $D$ (that is, the actual diversity) is then calculated as the average of all $m$ $D_i$ values by
%
\[
D =\frac{\sum_{i=1}^m{D_i}}{m}
\]
%
The definition of species implies that a single network can actually be part of all species for each input. This happens if the value of all output neurons is the same. One disadvantage of \dia{} is that it relies on random inputs, which means that fewer random inputs implies less statistical significance. Due to the nature of \dia, it is not suitable for continuous problems. Consider for instance a neural network used for approximating any function $f$, having only a single output neuron. Since \dia{} defines species based on the output neuron with the highest value, and we in this case only have a single output neuron, all neural networks will belong to the same species, and hence we get a diversity of 0.  
