\section{Neural Network Trait Diversity}
In the following, we will use the term individual and neural network interchangeably, since we represent an individual by a neural network.  Let $F = \set{f_1, f_2, \dots, f_n}$, denote the set of neural networks contained in a population, which all have the same architecture of $a$ input and $b$ output neurons. NNTD is calculated with respect to a number of random inputs $\set{R_1, R_2, \dots, R_m}$, where each $R_i$ for $1 \leq i \leq m$ is an $a$-tuple of real values chosen randomly. For each $R_i$, a Simpson's Diversity Index (SDI) is calculated. SDI is a diversity measure used in ecology to quantify the biodiversity of a habitat. NNTD is calculated as the average SDI for all $R_i$.
To calculate SDI, the total number of neural networks $\lvert F \lvert$ is needed, as well as a distribution of neural networks into a set of species. We classify species of individuals using $S_i(R)$ as the set of neural networks belonging to the $i$th species with respect to input $R$.

We distribute the neural networks into species based on which of their output neurons yield the highest value on input $R$. This means that for any neural network belonging to the species $S_i(R)$, the value of the $i$th output neuron will be greater than or equal to the value of any other output neuron given the input $R$.  This definition implies that the number of species equals the number of output neurons $b$. We distribute neural networks into a set of species as follows
%
\begin{equation*}\label{eq:species}
  S_i(R) = \setof{f_p}{\forall j \in \set{1, 2, \dots, b} \left(\sigma_{Rpi} \geq \sigma_{Rpj}\right)}
\end{equation*}
%
where $\sigma_{xyz} \in \set{0, 1}$ is the output of the $z$th output neuron in neural network $f_y$ on input $x$.

One disadvantage of NNTD is that it relies on random inputs, which means that fewer random inputs implies less statistical significance.
