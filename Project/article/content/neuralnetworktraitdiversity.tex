\section{\di{}}\label{sec:nntd}
In the following, we propose a method for measuring trait diversity, which we call \emph{\di{}} (\dia). \dia{} aims to reflect the diversity of different traits among individuals. 

\subsection{Algorithm}
\dia{} is based on Simpsons Diversity Index (SDI), which is a diversity measure used in ecology to quantify the biodiversity of a habitat~\cite{simpson1949measurement}. SDI is the probability that two randomly chosen samples belong to different species\cite{Venturi2007182}. To calculate SDI, we must define a method for distributing the individuals (neural networks) of a population into a number of species.

Let $\indset = \set{\ind_1, \ind_2, \dots, \ind_a}$ be the set of neural networks contained in a population, all with the same architecture of $n$ input and $m$ output neurons, and let \ran{} be any random input for the neural networks. \ran{} must then be an $n$-tuple of values. If $b_{m}b_{m-1}\dots b_1$ is the binary representation of a number $i$, we define the species \speciesi{i}{\ran} to contain any individual $\ind \in \indset$, that given $\ran$ as input satisfies
%
\begin{equation}\label{eq:nntdbucket}
  \forall j, h \in \set{1, 2, \dots, m} \left(b_j \rightarrow \left(\nnout_j \geq \nnout_h\right)\right)
\end{equation}
%
where $\nnout_k \in \nnoutset$ is the value of the $k$th output neuron of neural network \ind. As an example, assume that we have a neural network with six output neurons. If the third output neuron has the highest output when given $\ran$ as input, the neural network will belong to species $\species_4$, because the \texttt{1}s in \texttt{000100} (the binary representation of 4) matches the indices of the output neurons with the highest value. If multiple output neurons have the highest value, say the first and third output neuron, the neural network will belong to species $\species_5$ since \texttt{000101} in binary is 5 in decimal. If all output neurons have the same value, the neural network belongs to species $\species_{63}$ since \texttt{111111} in binary is 63 in decimal.
Once each individual is assigned a single species, we calculate SDI given $\ran$ by inserting into the formula for SDI\cite{Venturi2007182}
%
\begin{equation*}\label{eq:nntd}
  D_x = 1 - \frac{\sum_{q \in \nemptyspeciesf{x}}\left(\sizeof{q}\left(\sizeof{q} - 1\right)\right)}{\indsetl\left(\indsetl - 1\right)}
\end{equation*}
%
where $Q_x = \set{\speciesi{1}{\ran_x}, \speciesi{2}{\ran_x}, \dots, \speciesi{2^{m}-1}{\ran_i}}$ is the set of species for random input $r_i$. The \dia{} $D$ (the actual diversity) is calculated as the average SDI given a number of random inputs $\ranset = \set{\ran_1, \ran_2, \cdots, \ran_x}$
%
\begin{equation}\label{eq:nntdavg}
  D =\frac{\sum_{x=1}^{\ransetl}{D_x}}{\ransetl}
\end{equation}
%

One disadvantage of \dia{} is that it relies on random inputs, which means that fewer random inputs implies less statistical significance. Also, the choice of random inputs might affect the calculated \dia. We suggest that each random input is chosen according to the probability of a neural network receiving that input in the intended application. 
Due to the nature of \dia, it is not suitable for continuous problems. Consider for instance two neural networks $\ind_1$ and $\ind_2$, used for approximating a function $g$, where $\ind_1$ and $\ind_2$ both have only a single output neuron. To tell how different $\ind_1$ and $\ind_2$ behave, it is necessary to distinguish between different values of each neural network's output neuron. Since \dia{} defines species based on the output neuron with the highest value, and we in this case only have a single output neuron in $\ind_1$ and $\ind_2$, the two neural networks will always belong to the same species $\speciesi{1}{r}$ for all $r \in \ranset$, yielding a diversity of 0.  

\subsection{Complexity}
Computations associated with \dia{} are split into two halves. The first half distributes neural networks into a number of species for each random input corresponding to \cref{eq:nntdbucket}. These species are then used in the second half, where SDI is calculated for each random input, corresponding to \cref{eq:nntdavg}.

The first part of computation is carried out as follows. For each random input $\ran \in \ranset$, every neural network $\ind \in \indset$ must calculate an output based on random input \ran. Each output neuron $\nnout \in \nnoutset$ belonging to \ind{} is now considered once to calculate the particular species \ind{} belongs to. The complexity of the first part is thus $\bigO{\ransetl \cdot \indsetl \cdot \left(\timet + \nnoutsetl\right)}$, where \timet{} is the time required to calculate the output of a neural network given any input. When calculating the output of a neural network, every neuron in the network must be considered. Therefore, \timet{} is an upper bound of \nnoutsetl. This allows us to reduce the complexity to $\bigO{\ransetl \cdot \indsetl \cdot \timet}$ for the first half of \dia.

In the second part, we note that SDI considers every species once for every random input, and hence this part has complexity $\bigO{\ransetl\cdot\nemptyspeciessetl}$, where \nemptyspeciesset{} is the largest set of non-empty species for all random inputs \ranset. Since each neural network is put in only one species, \indsetl{} is an upper bound of \nemptyspeciessetl, yielding the complexity $\bigO{\ransetl\cdot\indsetl}$ for the second half.

Combining both halves results in a total complexity of $\bigO{\ransetl \cdot \indsetl \cdot \timet + \ransetl \cdot \indsetl}$, which is reduced to $\bigO{\ransetl\cdot\indsetl\cdot\timet}$.

\subsection{Complexity comparison}
Out of all diversity measures, the linear complexity of fitness-based measures is difficult to compete with.

Comparing the complexities of Hamming distance and \dia{} allows some interesting reductions and assumptions. Upon comparison, we can remove the common factor \indset{} in both of two complexities $\bigO{\ransetl\cdot\indsetl\cdot\timet}$ of \dia{} and \bigO{\indsetl^2 \cdot \bitstringl} of Hamming distance resulting in $\bigO{\ransetl \cdot t}$ and $\bigO{\indsetl \cdot l}$, respectively. 

Additionally, we can make an assumption about how the length of a bit string \bitstringl{} is related to the value \timet, which is the number of steps required to calculate the output of a neural network given an arbitrary input. Let $E$ denote the set of edges in a neural network. Each weight must be considered only once when the output of a neural network is calculated. Applying the activation function one time can be considered a constant amount of work. In the extreme case, every neuron will only have a single input connection, hence the activation function will be calculated at most $|E|$ times. Therefore, we have that $\timet = \bigT{\card{E}}$. If we assume that $\card{E} = \bigT{\bitstringl}$, we also get that $\timet = \bigT{\bitstringl}$. This assumption is reasonable, since it is fulfilled if the weight of each edge is separately encoded in the bit string. With this assumption, we can simplify the complexity comparison of \dia{} and Hamming distance to a comparison of the complexities \bigO{\ransetl} and \bigO{\indsetl}, respectively.

In this manner, we can conclude that the complexities of \dia{} and Hamming distance are asymptotically the same, whenever $\ransetl = c \cdot \indsetl$ for some constant $c$. That is, whenever the number of random inputs chosen for \dia{} is a constant times the population size.
