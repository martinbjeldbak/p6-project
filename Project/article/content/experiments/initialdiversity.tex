\subsection{Initial diversity measurements}
Two controlled experiments are run on the genetics of individuals in the very first generation of a population to evaluate how each measure grasps the diversity in a population. This generation is vital, because on initializing, it is possible to exactly define the distribution of individuals. These two experiments involve defining how similar the initial population is, and how much mutation to apply. We introduce two variables to control these factors.

Initial similarity describes how equal the individuals initially are. It can range between \num{0} and \num{1}, where \num{0} means that all individuals are random, and \num{1} denotes that \perc{100} of all individuals have the exact same genotype. With a value of \num{0.5}, one individual is cloned so that half of the population has the same genetic makeup, and the other half is initialized to be completely random. By intuition, a population initialized with the value of \num{1} for initial similarity should have the absolute lowest diversity, and a completely randomly initialize population (initial similarity of \num{0}) should have the highest diversity, since every individual is created from a random bit string.

Initial mutation requires an initial similarity set to \num{1}, e.g.\ the initial population consists of a single, cloned individual. This is due to the fact that mutating individuals already randomly initialized will not yield a different diversity of individuals -- they will be just as random. The initial mutation rate defines the mutation percentage of all bits in every individual of the population. A population with an initial mutation rate of \num{1} should have the highest diversity, since every bit in every, initially similar individual, will have a \perc{100} chance to be selected for mutation, resulting in a randomly initialized population. When a bit is selected for mutation, there is a \perc{50} chance it will be flipped.

Results of different intervals between 

Experiments on \dia{} using both of these two variables are shown in \cref{fig:initial-mutation-similarity}. Each point represents the average diversity measure of \num{100} runs at a variable intervals of \num{0.05}. As we can see by the chart, diversity increases as expected for both variables. Diversity is maximum at a value of \num{7.2} for a completely random population. To reach this maximum, it requires an initial mutation rate of a mere \perc{2} and an initial similarity of \perc{0}.

\begin{figure*}
  \centering
  \begin{subfigure}[b]{0.33\textwidth}
    \begin{tikzpicture}
      \begin{axis}[
          initial-sim-root,
          legend to name=initSimMeasures,
          title=Leaf,
        ]
        \addplot[mark=*, color=blue]
        table[y index=1, x index=0] {data/initial_similarity_leaf.csv};
        \addplot[mark=square*, color=aqua] % Hamming
        table[y index=2, x index=0] {data/initial_similarity_leaf.csv};
        \addplot[mark=triangle*, color=teal]
        table[y index=3, x index=0] {data/initial_similarity_leaf.csv};
      \end{axis}
    \end{tikzpicture}
  \end{subfigure}%
  ~
  \begin{subfigure}[b]{0.33\textwidth}
    \begin{tikzpicture}
      \begin{axis}[
          initial-sim,
          title  = Snake,
        ]
        \addplot[mark=*, color=blue]
        table[y index=1, x index=0] {data/initial_similarity_snake.csv};
        \addplot[mark=square*, color=aqua] % Hamming
        table[y index=2, x index=0] {data/initial_similarity_snake.csv};
        \addplot[mark=triangle*, color=teal]
        table[y index=3, x index=0] {data/initial_similarity_snake.csv};
      \end{axis}
    \end{tikzpicture}
  \end{subfigure}%
  ~
  \begin{subfigure}[b]{0.33\textwidth}
    \begin{tikzpicture}
      \begin{axis}[
          initial-sim,
          title  = 8-bit XOR,
        ]
        \addplot[mark=*, color=blue]
        table[y index=1, x index=0] {data/initial_similarity_snake.csv};
        \addplot[mark=square*, color=aqua] % Hamming
        table[y index=2, x index=0] {data/initial_similarity_xor.csv};
        \addplot[mark=triangle*, color=teal]
        table[y index=3, x index=0] {data/initial_similarity_xor.csv};
      \end{axis}
    \end{tikzpicture}
  \end{subfigure}
  \ref{initSimMeasures}
  \caption{Average over \num{100} runs for each diversity measure on data sets over intervals of initial similarity. Each point represents the average of \num{100} runs for that initial similarity value.}\label{fig:initial-similarity}
\end{figure*}

\begin{figure*}
  \centering
  \begin{subfigure}[b]{0.33\textwidth}
    \begin{tikzpicture}
      \begin{axis}[
          initial-mut-root,
          initial-mut,
          legend to name=initMutMeasures,
          title=Leaf,
        ]
        \addplot[mark=*, color=blue] % Fitness-based
        table[y index=1, x index=0] {data/initial_mutation_leaf.csv};
        \addplot[mark=square*, color=red] % Hamming
        table[y index=2, x index=0] {data/initial_mutation_leaf.csv};
        \addplot[mark=triangle*, color=green] % NNTD
        table[y index=3, x index=0] {data/initial_mutation_leaf.csv};
      \end{axis}
    \end{tikzpicture}
  \end{subfigure}%
  ~
  \begin{subfigure}[b]{0.33\textwidth}
    \begin{tikzpicture}
      \begin{axis}[
          initial-mut, 
          title  = Snake,
        ]
        \addplot[mark=*, color=blue]
        table[y index=1, x index=0] {data/initial_mutation_snake.csv};
        \addplot[mark=square*, color=red]
        table[y index=2, x index=0] {data/initial_mutation_snake.csv};
        \addplot[mark=triangle*, color=green]
        table[y index=3, x index=0] {data/initial_mutation_snake.csv};
      \end{axis}
    \end{tikzpicture}
  \end{subfigure}%
  ~
  \begin{subfigure}[b]{0.33\textwidth}
    \begin{tikzpicture}
      \begin{axis}[
          initial-mut, 
          title  = 8-bit XOR,
        ]
        \addplot[mark=*, color=blue]
        table[y index=1, x index=0] {data/initial_mutation_xor.csv};
        \addplot[mark=square*, color=red]
        table[y index=2, x index=0] {data/initial_mutation_xor.csv};
        \addplot[mark=triangle*, color=green]
        table[y index=3, x index=0] {data/initial_mutation_xor.csv};
      \end{axis}
    \end{tikzpicture}
  \end{subfigure}
  \ref{initMutMeasures}
  \caption{Average over \num{100} runs for each diversity measure on data sets over intervals of initial mutation. Each point represents the average of \num{100} runs for that initial mutation value.}\label{fig:initial-mutation}
\end{figure*}

