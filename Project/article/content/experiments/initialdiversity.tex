\subsection{Static experiments}
We perform two different static experiments, which differ in the way they constrain the initially generated population. 

\subsubsection{Initial similarity}
In this test, we introduce the variable \emph{initial similarity}, which is a real value in the range $[0,1]$.
When making an initial population, an initial similarity of $\alpha$ means that $\alpha$ of the individuals in the population will have the exact same genotype, and $(1-\alpha)$ of the individuals are completely random.
An initial similarity of \num{1} means that all genotypes are the same, and hence the behaviour of all individuals are the same as well. In this case, we will expect the lowest diversity possible.
As initial similarity increases, more genotypes will be identical, and hence more individuals will behave the same.
We therefore expect the diversity to decrease as initial similarity is increased.
An initial similarity of \num{0} means that all genotypes are random, and hence we will expect the most diverse behaviour among individuals to be found here.

\subsubsection{Initial mutation}
In this test, we introduce the variable \emph{initial mutation}, which is a real value in the range $[0,1]$.
When making an initial population, an initial mutation of $\alpha$ affects the population in the following way.
A random genotype is created and given to every individual in the population, such that all individuals have an identical, randomly chosen genotype.
Now, the bit string of each individual is mutated. Each bit will with probability $\alpha$ be set to a random boolean value. 
We expect to see an increase in the different of behaviour, as a result of the mutation of bit strings. 

\subsubsection{Results}
Experiments run with initial similarity and initial mutation between 0 and 1 are shown in \cref{fig:initial-similarity} and \cref{fig:initial-mutation}, respectively.


\begin{figure*}
  \centering
  \begin{subfigure}[b]{0.33\textwidth}
    \begin{tikzpicture}
      \begin{axis}[
          initial-sim-root,
          legend to name=initSimMeasures,
          title=Leaf,
        ]
        \addplot[mark=*, color=blue]
        table[y index=1, x index=0] {data/initial_similarity_leaf.csv};
        \addplot[mark=square*, color=aqua] % Hamming
        table[y index=2, x index=0] {data/initial_similarity_leaf.csv};
        \addplot[mark=triangle*, color=teal]
        table[y index=3, x index=0] {data/initial_similarity_leaf.csv};
      \end{axis}
    \end{tikzpicture}
  \end{subfigure}%
  ~
  \begin{subfigure}[b]{0.33\textwidth}
    \begin{tikzpicture}
      \begin{axis}[
          initial-sim,
          title  = Snake,
        ]
        \addplot[mark=*, color=blue]
        table[y index=1, x index=0] {data/initial_similarity_snake.csv};
        \addplot[mark=square*, color=aqua] % Hamming
        table[y index=2, x index=0] {data/initial_similarity_snake.csv};
        \addplot[mark=triangle*, color=teal]
        table[y index=3, x index=0] {data/initial_similarity_snake.csv};
      \end{axis}
    \end{tikzpicture}
  \end{subfigure}%
  ~
  \begin{subfigure}[b]{0.33\textwidth}
    \begin{tikzpicture}
      \begin{axis}[
          initial-sim,
          title  = 8-bit XOR,
        ]
        \addplot[mark=*, color=blue]
        table[y index=1, x index=0] {data/initial_similarity_snake.csv};
        \addplot[mark=square*, color=aqua] % Hamming
        table[y index=2, x index=0] {data/initial_similarity_xor.csv};
        \addplot[mark=triangle*, color=teal]
        table[y index=3, x index=0] {data/initial_similarity_xor.csv};
      \end{axis}
    \end{tikzpicture}
  \end{subfigure}
  \ref{initSimMeasures}
  \caption{Average over \num{100} runs for each diversity measure on data sets over intervals of initial similarity. Each point represents the average of \num{100} runs for that initial similarity value.}\label{fig:initial-similarity}
\end{figure*}


% Init sim results
The results of various initial similarity values shown in \cref{fig:initial-similarity} play along with our intuition.
For each of the data sets, each diversity measure's diversity output gradually falls upon increasing the amount of similar individuals in the population. Interestingly enough, at an initial similarity of 0, \dia{} outputs the maximum possible diversity, whereas Hamming distance outputs a diversity that is only half of its maximum. It seems that \dia{} captures the chaotic nature of a completely random population better than the other two measures, with a larger fall in diversity over time.

\begin{figure*}
  \centering
  \begin{subfigure}[b]{0.33\textwidth}
    \begin{tikzpicture}
      \begin{axis}[
          initial-mut-root,
          initial-mut,
          legend to name=initMutMeasures,
          title=Leaf,
        ]
        \addplot[mark=*, color=blue] % Fitness-based
        table[y index=1, x index=0] {data/initial_mutation_leaf.csv};
        \addplot[mark=square*, color=red] % Hamming
        table[y index=2, x index=0] {data/initial_mutation_leaf.csv};
        \addplot[mark=triangle*, color=green] % NNTD
        table[y index=3, x index=0] {data/initial_mutation_leaf.csv};
      \end{axis}
    \end{tikzpicture}
  \end{subfigure}%
  ~
  \begin{subfigure}[b]{0.33\textwidth}
    \begin{tikzpicture}
      \begin{axis}[
          initial-mut, 
          title  = Snake,
        ]
        \addplot[mark=*, color=blue]
        table[y index=1, x index=0] {data/initial_mutation_snake.csv};
        \addplot[mark=square*, color=red]
        table[y index=2, x index=0] {data/initial_mutation_snake.csv};
        \addplot[mark=triangle*, color=green]
        table[y index=3, x index=0] {data/initial_mutation_snake.csv};
      \end{axis}
    \end{tikzpicture}
  \end{subfigure}%
  ~
  \begin{subfigure}[b]{0.33\textwidth}
    \begin{tikzpicture}
      \begin{axis}[
          initial-mut, 
          title  = 8-bit XOR,
        ]
        \addplot[mark=*, color=blue]
        table[y index=1, x index=0] {data/initial_mutation_xor.csv};
        \addplot[mark=square*, color=red]
        table[y index=2, x index=0] {data/initial_mutation_xor.csv};
        \addplot[mark=triangle*, color=green]
        table[y index=3, x index=0] {data/initial_mutation_xor.csv};
      \end{axis}
    \end{tikzpicture}
  \end{subfigure}
  \ref{initMutMeasures}
  \caption{Average over \num{100} runs for each diversity measure on data sets over intervals of initial mutation. Each point represents the average of \num{100} runs for that initial mutation value.}\label{fig:initial-mutation}
\end{figure*}


% Init mut results
By changing the initial mutation in a population with the results shown in \cref{fig:inital-mutation}, we notice a larger difference between diversity measures. At an initial mutation rate of a mere \perc{1}, \dia{} takes a great leap compared to the other two measures.
For Hamming distance, it is obvious that changing only a few bits in the genotypes will only cause a small change in diversity.
This is because Hamming distance measures diversity based on genotypes.

For the fitness-based diversity measure, it must be noted that the fitness values are dependent on the particular problem in question and how one chooses to define the fitness function.
Consider for instance the problem of making an AI for the game Snake.
We have experienced that if we define the fitness only in terms of how many pieces of food a snake collects,
then about $98 \%$ of all random individuals gets a fitness of $0$.
This is the reason why we chose a fitness function that also takes into account the number of steps a snake is alive.
Despite yielding more diverse fitness values, it also increased the fitness values obtained after just 100 iterations notably.
This does not mean that a fitness function cannot reflect behavioural differences, but it shows that how one defines the fitness function is crucial if one wishes to catch the behavioural differences of individuals.
