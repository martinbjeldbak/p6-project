\subsection{Initial diversity measurements}
Two controlled experiments are run on the genetics of individuals in the very first generation of a population to evaluate how each measure grasps the diversity in a population. This generation is vital, because on initializing, it is possible to exactly define the distribution of individuals. These two experiments involve defining how similar the initial population is, and how much mutation to apply. We introduce two variables to control these factors.

\subsubsection{First experiment}
Initial similarity describes how equal the individuals initially are. It can range between \num{0} and \num{1}, where \num{0} means that all individuals are random, and \num{1} denotes that \perc{100} of all individuals have the exact same genotype. With a value of \num{0.5}, one individual is cloned so that half of the population has the same genetic makeup, and the other half is initialized to be completely random. By intuition, a population initialized with the value of \num{1} for initial similarity should have the absolute lowest diversity, and a completely randomly initialize population (initial similarity of \num{0}) should have the highest diversity, since every individual is created from a random bit string.

\subsubsection{Second experiment}
Initial mutation requires an initial similarity set to \num{1}, e.g.\ the initial population consists of a single, cloned individual. This is due to the fact that mutating individuals already randomly initialized will not yield a different diversity of individuals -- they will be just as random. The initial mutation rate defines the mutation percentage of all bits in every individual of the population. A population with an initial mutation rate of \num{1} should have the highest diversity, since every bit in every, initially similar individual, will have a \perc{100} chance to be selected for mutation, resulting in a randomly initialized population. When a bit is selected for mutation, there is a \perc{50} chance it will be flipped.

\subsubsection{Results}
Experiments run with initial similarity and initial mutation between 0 and 1 are shown in \cref{fig:initial-similarity} and \cref{fig:initial-mutation}, respectively.

By intuition, we expect increasing initial similarity will result in lower diversity, and increasing initial mutation will result in greater diversity. Maximum diversity should be seen at the lowest value of initial similarity, and the highest value of initial mutation. This is due to the fact that many alike individuals will behave similarly and thus have a negative impact on high diversity.


\begin{figure*}
  \centering
  \begin{subfigure}[b]{0.33\textwidth}
    \correctlyresize{\linewidth}{%
      \begin{tikzpicture}
        \begin{axis}[
            initial-sim-root,
            legend to name=initSimMeasures,
            title=Leaf,
          ]
          \addplot[mark=*, color=maroon]
          table[y index=1, x index=0] {data/initial_similarity_leaf.csv};
          \addplot[mark=square*, color=navy]
          table[y index=2, x index=0] {data/initial_similarity_leaf.csv};
          \addplot[mark=triangle*, color=blue]
          table[y index=3, x index=0] {data/initial_similarity_leaf.csv};
        \end{axis}
      \end{tikzpicture}
    }
  \end{subfigure}%
  ~
  \begin{subfigure}[b]{0.33\textwidth}
    \correctlyresize{\linewidth}{%
      \begin{tikzpicture}
        \begin{axis}[
            initial-sim,
            title  = Snake,
          ]
          \addplot[mark=*, color=maroon]
          table[y index=1, x index=0] {data/initial_similarity_snake.csv};
          \addplot[mark=square*, color=navy]
          table[y index=2, x index=0] {data/initial_similarity_snake.csv};
          \addplot[mark=triangle*, color=blue]
          table[y index=3, x index=0] {data/initial_similarity_snake.csv};
        \end{axis}
      \end{tikzpicture}
    }
  \end{subfigure}%
  ~
  \begin{subfigure}[b]{0.33\textwidth}
    \correctlyresize{\linewidth}{%
      \begin{tikzpicture}
        \begin{axis}[
            initial-sim,
            title  = XOR,
          ]
          \addplot[mark=*, color=maroon]
          table[y index=1, x index=0] {data/initial_similarity_snake.csv};
          \addplot[mark=square*, color=navy]
          table[y index=2, x index=0] {data/initial_similarity_xor.csv};
          \addplot[mark=triangle*, color=blue]
          table[y index=3, x index=0] {data/initial_similarity_xor.csv};
        \end{axis}
      \end{tikzpicture}
    }
  \end{subfigure}
  \ref{initSimMeasures}
  \caption{Average over \num{100} runs for each diversity measure on data sets over intervals of initial similarity.}\label{fig:initial-similarity}
\end{figure*}


% Init sim results
The results of various initial similarity values shown in \cref{fig:initial-similarity} play along with our intuition. For each of the data sets, each diversity measure's diversity output gradually falls upon increasing the amount of similar individuals in the population. Interestingly enough, at an initial similarity of 0, \dia{} outputs the maxiumum possible diversity, whereas Hamming distance outputs a diversity that is only half of its maximum. It seems that \dia{} captures the chaotic nature of a completely random population better than the other two measures, with a larger fall in diversity over time.

\begin{figure*}
  \centering
  \begin{subfigure}[b]{0.33\textwidth}
    \begin{tikzpicture}
      \begin{axis}[
          initial-mut-root,
          legend to name=initMutMeasures,
          title=Leaf,
        ]
        \addplot[mark=*, color=blue] % Fitness-based
        table[y index=1, x index=0] {data/initial_mutation_leaf.csv};
        \addplot[mark=square*, color=aqua] % Hamming
        table[y index=2, x index=0] {data/initial_mutation_leaf.csv};
        \addplot[mark=triangle*, color=teal] % NNTD
        table[y index=3, x index=0] {data/initial_mutation_leaf.csv};
      \end{axis}
    \end{tikzpicture}
  \end{subfigure}%
  ~
  \begin{subfigure}[b]{0.33\textwidth}
    \begin{tikzpicture}
      \begin{axis}[
          initial-mut, 
          title  = Snake,
        ]
        \addplot[mark=*, color=blue]
        table[y index=1, x index=0] {data/initial_mutation_snake.csv};
        \addplot[mark=square*, color=aqua]
        table[y index=2, x index=0] {data/initial_mutation_snake.csv};
        \addplot[mark=triangle*, color=teal]
        table[y index=3, x index=0] {data/initial_mutation_snake.csv};
      \end{axis}
    \end{tikzpicture}
  \end{subfigure}%
  ~
  \begin{subfigure}[b]{0.33\textwidth}
    \begin{tikzpicture}
      \begin{axis}[
          initial-mut, 
          title  = 8-bit XOR,
        ]
        \addplot[mark=*, color=blue]
        table[y index=1, x index=0] {data/initial_mutation_xor.csv};
        \addplot[mark=square*, color=aqua]
        table[y index=2, x index=0] {data/initial_mutation_xor.csv};
        \addplot[mark=triangle*, color=teal]
        table[y index=3, x index=0] {data/initial_mutation_xor.csv};
      \end{axis}
    \end{tikzpicture}
  \end{subfigure}
  \ref{initMutMeasures}
  \caption{Average over \num{100} runs for each diversity measure on data sets over intervals of initial mutation. Each point represents the average of \num{100} runs for that initial mutation value.}\label{fig:initial-mutation}
\end{figure*}


% Init mut results
By changing the initial mutation in a population with the results shown in \cref{fig:inital-mutation}, we notice a larger difference between diversity measures. At an initial mutation rate of a mere \perc{1}, \dia{} reaches maximum diversity compared to the other two measures. This is due to the fact that when only a few bits are flipped in the individuals, Hamming distance and fitness-based diversity measure won't capture the fact that this bit could be a sign bit, a high-order bit, or another behaviourally vital bit. Hamming distance does not capture this due to being genotypic, and fitness-based also lacks an ability to capture this change, because an increasingly random population does not necessarily mean a change in fitness.
