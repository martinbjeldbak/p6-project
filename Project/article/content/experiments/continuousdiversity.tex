\subsection{Continuous Diversity}
\label{sec:continuousdiversity}

Our main goal with the experiments we perform, is to see how the diversity progresses throughout the many iterations a GA goes through. Furthermore, we must see if there are any significant differences between different diversity measures and replacement rules. To measure the diversity we use fitness-based measurement, Hamming distance, and our own \dia{} after each iteration. The replacement rules are Greedy Replacement, Ancestor Elitism, Single Parent Elitism, and Explore-Exploit. We run the experiments four times, one with each of the replacement rules.

We measured the average diversity of \num{100} runs of \num{2000} iterations. Each run initializes an entirely new population, and each population goes through \num{2000} iterations of procreation in the search for optimal solutions. After each iteration the diversity of the population is measured. We then take the average diversity of the \num{100} runs. %These numbers show us how the diversity is progressing with the different replacement rules for the different problems.

%how much did we do it
We use the average of the \num{100} runs, because we have randomness in our system. Each new population is initialized randomly. By taking the average we belive that we get a better picture of how the diversity is progressing.

\subsubsection{Results}
The results are presented through figures \emph{s} to \emph{s+2}. The experiments performed on the data we call Leaf is presented in \emph{s}, the XOR experiments are presented in \emph{s+1}, and the data we call Snake is presented in \emph{s+2}.

\paragraph{Results of Leaf} dsdsds

\paragraph{Results of XOR} dsds

\paragraph{Results of Snake} dsds

%what were the results
In figure \emph{something} we see that bla bla\dots

This also means that bla bla\dots
%anything else?

