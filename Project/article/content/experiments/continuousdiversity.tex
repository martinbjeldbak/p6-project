\subsection{Continuous Diversity}
\label{sec:continuousdiversity}

Our main goal with the experiments we perform, is to see how the diversity progresses throughout the many iterations a GA goes through. Furthermore, we must see if there are any significant differences between different diversity measures and replacement rules. To measure the diversity we use fitness-based measurement, Hamming distance, and our own \dia{} after each iteration. The replacement rules are Greedy Replacement, Ancestor Elitism, Single Parent Elitism, and Explore-Exploit. We run the experiments four times, one with each of the replacement rules.

We measured the average diversity of \num{100} runs of \num{2000} iterations. Each run initializes an entirely new population, and each population goes through \num{2000} iterations of procreation in the search for optimal solutions. After each iteration the diversity of the population is measured. We then take the average diversity of the \num{100} runs. %These numbers show us how the diversity is progressing with the different replacement rules for the different problems.

%how much did we do it
We use the average of the \num{100} runs, because we have randomness in our system. Each new population is initialized randomly. By taking the average we belive that we get a better picture of how the diversity is progressing.

\subsubsection{Results}
The results are presented through figures \emph{s} to \emph{s+2}. The experiments performed on the data we call Leaf is presented in \emph{s}, the XOR experiments are presented in \emph{s+1}, and the data we call Snake is presented in \emph{s+2}.

\paragraph{Results of Leaf} dsdsds

\paragraph{Results of XOR} dsds

\paragraph{Results of Snake} dsds

%what were the results
In figure \emph{something} we see that bla bla\dots

This also means that bla bla\dots
%anything else?

\begin{figure*}
  \centering
  \begin{subfigure}[b]{0.25\textwidth}
    \batchmode
    \begin{tikzpicture}
      \begin{axis}[
          dynamic,
          dynamic-root,
          legend to name=leafDynamic,
          title={\dia{}},
        ]
        \addplot[mark=*, smooth, color=black]
        table[ x index=0, y index=3] {data/replacementrule_game/greedy_leaf.csv};
        \addplot[mark=square*, smooth, color=blue]
        table[x index=0, y index=3] {data/replacementrule_game/aerr_leaf.csv};
        \addplot[mark=triangle*, smooth, color=red]
        table[x index=0, y index=3] {data/replacementrule_game/sperr_leaf.csv};
        \addplot[mark=diamond*, smooth, color=green]
        table[x index=0, y index=3] {data/replacementrule_game/eerr_leaf.csv};
      \end{axis}
    \end{tikzpicture}
    \scrollmode
  \end{subfigure}%
  ~
  \begin{subfigure}[b]{0.25\textwidth}
    \batchmode
    \begin{tikzpicture}
      \begin{axis}[
          dynamic,
          title=Fitness-based,
        ]
        \addplot[mark=*, smooth, color=black]
        table[x index=0, y index=1] {data/replacementrule_game/greedy_leaf.csv};
        \addplot[mark=square*, smooth, color=blue]
        table[x index=0, y index=1] {data/replacementrule_game/aerr_leaf.csv};
        \addplot[mark=triangle*, smooth, color=red]
        table[x index=0, y index=1] {data/replacementrule_game/sperr_leaf.csv};
        \addplot[mark=diamond*, smooth, color=green]
        table[x index=0, y index=1] {data/replacementrule_game/eerr_leaf.csv};
      \end{axis}
    \end{tikzpicture}
    \scrollmode
  \end{subfigure}%
  ~
  \begin{subfigure}[b]{0.25\textwidth}
    \batchmode
    \begin{tikzpicture}
      \begin{axis}[
          dynamic,
          title  = Hamming distance,
        ]
        \addplot[mark=*, smooth, color=black]
        table[x index=0, y index=2] {data/replacementrule_game/greedy_leaf.csv};
        \addplot[mark=square*, smooth, color=blue]
        table[x index=0, y index=2] {data/replacementrule_game/aerr_leaf.csv};
        \addplot[mark=triangle*, smooth, color=red]
        table[x index=0, y index=2] {data/replacementrule_game/sperr_leaf.csv};
        \addplot[mark=diamond*, smooth, color=green]
        table[x index=0, y index=2] {data/replacementrule_game/eerr_leaf.csv};
      \end{axis}
    \end{tikzpicture}
    \scrollmode
  \end{subfigure}%
  ~
  \begin{subfigure}[b]{0.25\textwidth}
    \batchmode
    \begin{tikzpicture}
      \begin{axis}[
          fitness
        ]
        \addplot[mark=*, smooth, color=black]
        table[x index=0, y index=4] {data/replacementrule_game/greedy_leaf.csv};
        \addplot[mark=square*, smooth, color=blue]
        table[x index=0, y index=4] {data/replacementrule_game/aerr_leaf.csv};
        \addplot[mark=triangle*, smooth, color=red]
        table[x index=0, y index=4] {data/replacementrule_game/sperr_leaf.csv};
        \addplot[mark=diamond*, smooth, color=green]
        table[x index=0, y index=4] {data/replacementrule_game/eerr_leaf.csv};
      \end{axis}
    \end{tikzpicture}
    \scrollmode
  \end{subfigure}
  \ref{leafDynamic}
  \caption{Average diversity over \num{100} runs for each replacement rule over \num{500} generations of the leaf data set.}\label{fig:dynamic-leaf}
\end{figure*}

\begin{figure*}
  \centering
  \begin{subfigure}[b]{0.25\textwidth}
    \begin{tikzpicture}
      \begin{axis}[
          dynamic,
          dynamic-root,
          title={\dia{}},
          legend to name=xorDynamic,
        ]
        \addplot[mark=*, smooth, color=black]
        table[x index=0, y index=3] {data/replacementrule_game/greedy_xor.csv};
        \addplot[mark=square*, smooth, color=blue]
        table[x index=0, y index=3] {data/replacementrule_game/aerr_xor.csv};
        \addplot[mark=triangle*, smooth, color=red]
        table[x index=0, y index=3] {data/replacementrule_game/sperr_xor.csv};
        \addplot[mark=diamond*, smooth, color=green]
        table[x index=0, y index=3] {data/replacementrule_game/eerr_xor.csv};
      \end{axis}
    \end{tikzpicture}
  \end{subfigure}%
  ~
  \begin{subfigure}[b]{0.25\textwidth}
    \begin{tikzpicture}
      \begin{axis}[
          dynamic,
          title=Fitness-based,
        ]
        \addplot[mark=*, smooth, color=black]
        table[x index=0, y index=1] {data/replacementrule_game/greedy_xor.csv};
        \addplot[mark=square*, smooth, color=blue]
        table[x index=0, y index=1] {data/replacementrule_game/aerr_xor.csv};
        \addplot[mark=triangle*, smooth, color=red]
        table[x index=0, y index=1] {data/replacementrule_game/sperr_xor.csv};
        \addplot[mark=diamond*, smooth, color=green]
        table[x index=0, y index=1] {data/replacementrule_game/eerr_xor.csv};
      \end{axis}
    \end{tikzpicture}
  \end{subfigure}%
  ~
  \begin{subfigure}[b]{0.25\textwidth}
    \begin{tikzpicture}
      \begin{axis}[
          dynamic,
          title  = Hamming distance,
        ]
        \addplot[mark=*, smooth, color=black]
        table[x index=0, y index=2] {data/replacementrule_game/greedy_xor.csv};
        \addplot[mark=square*, smooth, color=blue]
        table[x index=0, y index=2] {data/replacementrule_game/aerr_xor.csv};
        \addplot[mark=triangle*, smooth, color=red]
        table[x index=0, y index=2] {data/replacementrule_game/sperr_xor.csv};
        \addplot[mark=diamond*, smooth, color=green]
        table[x index=0, y index=2] {data/replacementrule_game/eerr_xor.csv};
      \end{axis}
    \end{tikzpicture}
  \end{subfigure}%
  ~
  \begin{subfigure}[b]{0.25\textwidth}
    \begin{tikzpicture}
      \begin{axis}[
          fitness,
        ]
        \addplot[mark=*, smooth, color=black]
        table[x index=0, y index=4] {data/replacementrule_game/greedy_xor.csv};
        \addplot[mark=square*, smooth, color=blue]
        table[x index=0, y index=4] {data/replacementrule_game/aerr_xor.csv};
        \addplot[mark=triangle*, smooth, color=red]
        table[x index=0, y index=4] {data/replacementrule_game/sperr_xor.csv};
        \addplot[mark=diamond*, smooth, color=green]
        table[x index=0, y index=4] {data/replacementrule_game/eerr_xor.csv};
      \end{axis}
    \end{tikzpicture}
  \end{subfigure}
  \ref{xorDynamic}
  \caption{Average over \num{100} runs for each diversity measure on approximating an 8-bit XOR gate.}\label{fig:dynamic-xor}
\end{figure*}

\begin{figure*}
  \centering
  \begin{subfigure}[b]{0.33\textwidth}
      \begin{tikzpicture}
        \begin{axis}[
            dynamic-root,
            dynamic,
            title={\dia{}},
            legend to name=snakeDynamic,
          ]
          \addplot[smooth, color=navy]
          table[x index=0, y index=3] {data/replacementrule_game/greedy_snake.csv};
          \addplot[smooth, color=blue]
          table[x index=0, y index=3] {data/replacementrule_game/aerr_snake.csv};
          \addplot[smooth, color=aqua]
          table[x index=0, y index=3] {data/replacementrule_game/sperr_snake.csv};
          \addplot[smooth, color=teal]
          table[x index=0, y index=3] {data/replacementrule_game/eerr_snake.csv};
        \end{axis}
      \end{tikzpicture}
  \end{subfigure}%
  ~
  \begin{subfigure}[b]{0.33\textwidth}
      \begin{tikzpicture}
        \begin{axis}[
            dynamic,
            title=Fitness-based,
          ]
          \addplot[smooth, color=navy]
          table[x index=0, y index=1] {data/replacementrule_game/greedy_snake.csv};
          \addplot[smooth, color=blue]
          table[x index=0, y index=1] {data/replacementrule_game/aerr_snake.csv};
          \addplot[smooth, color=aqua]
          table[x index=0, y index=1] {data/replacementrule_game/sperr_snake.csv};
          \addplot[smooth, color=teal]
          table[x index=0, y index=1] {data/replacementrule_game/eerr_snake.csv};
        \end{axis}
      \end{tikzpicture}
  \end{subfigure}%
  ~
  \begin{subfigure}[b]{0.33\textwidth}
      \begin{tikzpicture}
        \begin{axis}[
            dynamic,
            title  = Hamming distance,
          ]
          \addplot[smooth, color=navy]
          table[x index=0, y index=2] {data/replacementrule_game/greedy_snake.csv};
          \addplot[smooth, color=blue]
          table[x index=0, y index=2] {data/replacementrule_game/aerr_snake.csv};
          \addplot[smooth, color=aqua]
          table[x index=0, y index=2] {data/replacementrule_game/sperr_snake.csv};
          \addplot[smooth, color=teal]
          table[x index=0, y index=2] {data/replacementrule_game/eerr_snake.csv};
        \end{axis}
      \end{tikzpicture}
  \end{subfigure}
  \ref{snakeDynamic}
  \caption{Average over \num{100} runs for each diversity measure on the snake data set.}\label{fig:dynamic-snake}
\end{figure*}

