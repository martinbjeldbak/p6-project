\subsection{Problems under study}
\label{sec:problems}
We hereby explain the three discrete problems we perform our experiments on. The first problem of approximating the XOR function is interesting, since it is the simplest boolean function that is not linearly separable. This fact has made it quite popular in neural network research communities~\cite{masterThesisGANN}. Next is the classification of leaves from a dataset in the UCI Machine Learning Repository is interesting, because it is radically different from the XOR problem. XOR is a simple and well defined function, whereas classifying leaves is more complex and depends on observations in nature, which may contain noise. The Snake game differs in that it is an agent decision problem, where not just a single, but a sequence of decisions determines the outcome, where each decision changes the intermediate state.

\subsubsection{XOR}
We use a neural network to approximate the XOR between two 8-bit strings. To evaluate the fitness of an individual solving this problem, we calculate the XOR of \num{1000} random two 8-bit strings. For each instance, we determine how many bits the individual calculates correctly. The fitness is then defined as the average number of bits correctly calculated. For the 8-bit XOR problem, any fitness value will thus be a real number in the range $[0, 8]$. The XOR between two bits of random values has equal probability of yielding the value 0 or 1. Therefore, randomly guessing a solution to the XOR between two bit strings of length 8 is expected to yield a fitness of 4.

The network has 16 input neurons, 16 hidden neurons, 8 output neurons, 2 bits per weight, and 1 bit per bias for each hidden and output neuron. We represent the two 8-bit strings as being side by side. So, any output neuron $i$ represents the XOR between the two input neurons $i$ and $i + 8$. 

We have used as few neurons and bits to represent weights and biases as possible, while still being able to verify that the maximum fitness value of 8 is achievable.
The same random seed is always used for generating the 1000 problem instances.

\subsubsection{Leaf classification}
We use a neural network to classify the Leaf data set~\cite{Bache+Lichman:2013, leafdataset}. The neural network is given 16 properties about an unknown leaf and has to decide which of 40 types of leaves it is. Fitness is evaluated based on how many instances out of the entire data set the neural network correctly classifies.
The implementation consists of 16 input neurons, 10 hidden and 40 output neurons. The output neuron with the highest value decides the classification. Each weight is encoded by 9 bits and neurons have no bias.

\subsubsection{Snake}
Snake is a game found on old Nokia cell phones, where the player controls a snake around a grid to pick up pieces of food.
Every time a piece of food is collected, both the length of the snake and your score increases by 1.
You lose if the snake head hits its body or one of the edges of the grid.
At all times, the grid contains only a single piece of food.
The game becomes harder as the length of the snake increases, and as the snake is constantly moving, the challenge in not trapping oneself gets tougher.

We use a neural network to play a game of Snake in a $10 \times 10$ grid with an initial snake length of 5 units.
We have defined the fitness of a neural network to be 
\[
  \rho + \frac{\rho}{s/1000}
\]
where $\rho$ is the amount of collected food, and $s$ is the total number of steps the snake is alive. The game is constrained such that the snake can only change its direction \SI{90}{\degree} per step. The neural network has 6 input neurons, each receiving a bit of information about the game state. The first two inputs are relative to the snake's head. So, if the food is vertically and horizontally aligned with the head, then these two input values are 0
\begin{enumerate}
  \item \set{-1, 0, 1} food is left of, verti.\ aligned, or right of
  \item \set{-1, 0, 1} food is above, hori.\ aligned, or below
  \item \set{0, 1} death upon moving up 
  \item \set{0, 1} death upon moving down 
  \item \set{0, 1} death upon moving right
  \item \set{0, 1} death upon moving left 
\end{enumerate}
The neural network has 4 output neurons, one for each direction to choose. The neuron with the highest value determines which direction the snake moves.
The neural network uses 5 hidden neurons, 9 bits per weight and neurons have no bias.
For every game of Snake, we always use the same random seed to decide the positions where pieces of food will spawn.
