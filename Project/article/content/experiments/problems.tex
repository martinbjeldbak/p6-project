\subsection{Problems}
% Discrete: Leaf
The discrete problem is the small leaf data set~\cite{Bache+Lichman:2013, leafdataset}. Fitness is evaluated based on how many instances out of the entire data set the neural network correctly classifies. It consists of 16 inputs and 1 output.

% Discrete: Rosenbrock 
%Approximating the 2D Rosenbrock function is the second discrete problem. Here fitness is computed as the average distance to the actual output of the Rosenbrock function, over \num{10000} random inputs. The individuals have two input neurons and one output.

% Continuous: Snake
The continuous problem used is the Snake game known from Nokia cell phones. This game is very similar to the artificial ant problem~\cite[p.\ 147--155]{koza1992genetic}. A snake has to traverse a square $10\times10$ grid and search for a randomly positioned piece of food using a sensing function that's able to see the squares and their contents directly next to its head and above and below its head. Once the snake traverses into the square with food, the food disappears and the snake's length grows by one square. A new piece of food then spawns at a new random position. The snake has four actions to move around the grid, each of which takes one time unit: move up, down, left, and right. The game ends if the amount of time units exeeds \num{1000}, or if the snake collides into a wall. Its fitness function is based upon the amount of food eaten.
