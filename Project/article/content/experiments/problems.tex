\subsection{Problems}
We hereby explain the three discrete problems we perform our experiments on.

\subsubsection{XOR}
We use a neural network to approximate the XOR between two 8-bit strings.
To evaluate fitness of a neural network, we choose 1000 random instances of two 8-bit strings.
For each instance, we determine how many bits the neural network calculates correct.
The fitness is the defined as the average number of bits calculated correct.
Fitness values will thus lie in the range $[0-8]$.
Since the XOR between two random bits will be evenly distributed between 0 and 1,
randomly guessing will ideally yield a fitness of 4.
The network has 16 input neurons, 16 hidden neurons, 8 output neurons, 9 bits per weight, and 2 bits to encode a threshold for each hidden and output neuron. Any output neuron $i$ represents the XOR between the input neuron $i$ and $i+8$. 
We always use the same random seed for generating the 1000 problem instances.

\subsubsection{Leaf classification}
We use a neural network to classify the Leaf data set~\cite{Bache+Lichman:2013, leafdataset}.
The neural network is given 16 properties about an unknown leaf and has to decide which of 40 types of leaves it is.
Fitness is evaluated based on how many instances out of the entire data set the neural network correctly classifies.
The implementation consists of 16 input neurons, 10 hidden and 40 output neurons. The output neuron with the highest value decides the classification. Each weight is encoded by 9 bits and neurons have no threshold.

\subsubsection{Snake}
Snake is a game found on old Nokia cell phones, where you are moving a snake around a grid to pick up pieces of food.
Every time a piece of food is collected, both the length of the snake and your score increases by 1.
You loose if the snake head hits its body or one of the edges of the grid.
At all times, the grid contains only a single piece of food.
The game becomes harder as the length of the snake increases, and as the snake is constantly moving, it becomes hard to not trap yourself.

We use a neural network to play a game of Snake in a $10\times10$ grid with an initial snake length of 5 units.
We have defined the fitness of a neural network to be $f + \frac{f}{s/1000}$,
where $f$ is the amount of food it collects and $s$ is the total number of steps the snake is alive.
The game is constrained such that the snake can only change its direction $90\deg$ / step.
The neural network has 6 input neurons, receiving the following information:
\begin{enumerate}
\item $\{-1, 0, 1\}$ Whether the food is to the left, vertically aligned, or to the right of the snake head.
\item $\{-1, 0, 1\}$ Whether the food is above, horizontally aligned, or below the snake head.
\item $\{0, 1\}$ Whether the snake will die if its next move is up
\item $\{0, 1\}$ Whether the snake will die if its next move is down
\item $\{0, 1\}$ Whether the snake will die if its next move is right
\item $\{0, 1\}$ Whether the snake will die if its next move is left
\end{enumerate}
4 output neurons are used. The neuron with the highest value determines whether to move the snake right, left, up, or down.
The neural network uses 5 hidden neurons, 9 bits per weight and no threshold on neurons.
For every game of Snake, we always use the same random seed to decide the positions where pieces of food will spawn.

\subsubsection{Criteria for selection}
The XOR function is interesting since it is the simplest boolean function that is not linearly separable.
This fact has made it quite popular in NN research communities\cite{masterThesisGANN}.
A classification problem like Leaf is interesting because it is radically different from the XOR problem.
XOR is a simple and well defined function, whereas classifying leaves is more complex and depends on observations in nature, which may contain noise.
The Snake game differs in that it is an agent decision problem, where not just a single, but a sequence of decisions determines the outcome, where each decision changes the intermediate state.