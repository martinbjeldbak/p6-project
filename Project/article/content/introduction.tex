\section{Introduction}
Neural networks are often chosen as individuals in a population within genetic algorithms. These individuals procreate to form a new generation by creating offspring. Offspring consists of some combination of parent individuals and usually replace the individuals that are worst off in the preceding generation, simulating natural selection.

Neural networks are often chosen as individuals within genetic algorithms.

% Artificial Neural networks
%   - Neurons
%   - Transfer functions
% Genetic algorithms
%  Are inspired by nature
%   - Individual encoding (bit strings), genotype, phenotype
%   - Populations
%   - Fitness function define how fit an individual is in a population
%   - Generations
%     - Selection methods
%       We propose a new method of selection
%       - Mass extinction, random individuals
%     - Crossovers and mutations, probability of mutation and crossover
%       - Methods

\subsection{Artificial Neural Network}

An artificial neural network can from the outside be seen as a black box that 
given the values $x_1, x_2, ..., x_n$ outputs the values $y_1, y_2, ..., y_m$.
With the right internal structure, a neural network can be used for a variety of purposes, 
e.g. face recognition, where the intensities of different pixels in an image are used as input,
and a single output value $y_1$ is produced, where $y_1 = 1$ if the image was of a face and $0$ if not. 
We will now dig a bit into the internals of a neural network, to see how these output values are calculated based on the
input values.

\paragraph{Neurons}
A neural network is a collection of neurons. A neuron takes a number of values as input, applies
a weight to each value, sum them, and finally apply a function to produce a single output value. 
The function applied is called the transfer function.
In a feed forward network, neurons are placed in different layers, where each neuron uses the output of all
neurons from the previous layer as input. Each neuron in the first layer is called an input neuron and receives one
of the values input to the neural network. Any neuron that does not belong to the first layer is called a hidden neuron.
The neurons in the last layer are called output neurons. The value output by these neurons becomes
the output of the neural network.


\paragraph{Training a neural network}
Any application of a neural network requires that a correct number of layers, neurons, and weights between
neurons are found to adequately solve the problem. In many applications, a single layer of $k$ hidden neurons works well, where $k = \frac{1}{2}(|\mathtt{input neurons}| |\mathtt{output neurons}|$.
The weights between neurons are calculated by a process called training. Training algorithms, such as the
back-propagation algorithm, typically requires that for each input to the neural network, the output is already known.
For the application of face recognition, this means that the pictures used for training,
each has a predicate indicating whether it is a picture of a face or not.  

For some purposes, the desired output of a neural network is not known given a number of input values.
Consider for instance a computer game, where a player is controlled by a neural network. 
The neural network takes as input values indicating the state of the game, such as the position of the players
and enemies around him. For each input, the neural network returns a single value indicating which action the
player should take, e.g. move left, move right, jump. Given a state of the game, we might not be able to say
whether an action output by the neural network is right or wrong. It might be wrong to move closer to
an enemy if it kills you, but it might be right if the next action is to kill the enemy.

From this, it is clear that a back-propagation algorithm is not appropriate for training a neural network
to control the actions of a artificial intelligent (AI) player in a computer game. For this purpose
we propose another approach. We can tell how good a neural network performs, or how fit it is,
by simulating a game being played using the neural network to control the player. 
The fitness of the neural network can be determined by the score achieved in the game or
any other function indicating how well the AI player performed using that particular neural network.
By now being able to measure the fitness of a particular neural network, we can use a genetic algorithm
to search for a neural network with the highest possible fitness.

\subsection{Genetic algorithms}
A \emph{genetic algorithm} belongs to the class of evolutionary algorithms. This class of algorithms is inspired by natural evolution as seen in Biology. Genetic algorithms are search algorithms which imitate the process of natural selection.

\paragraph{Populations}

A genetic algorithm has a \emph{population} of \emph{individuals}. These individuals can be initialized randomly with a fixed population size.
The goal for these individuals is to solve a problem optimally. Each individual in the population has a fitness level that defines the individual's ability to solve a given problem.

\paragraph{Genes}

Each individual has specific \emph{genes}. Genes constitute the DNA of the individual. These genes are an encoding of some attribute or skill the individual has. The encoding of genes is up to the creator, but it is most often encoded as bit strings.

\paragraph{Crossovers and mutations}

In natural evolution, a pair of individuals come together to produce a new individual, a child individual, with genes from both of the parent individuals. The process of breeding is done by performing a \emph{crossover} of the two parent individuals' genes. A crossover point is defined and one part of each of the parents is copied and combined to form a new set of genes for the child individual.

%illustration of crossover and mutation

\emph{Mutations} can occur randomly at any point in time. While the genes are encoded as bit strings, then a mutation arbitrarily changes one of the bits. This ensures that the population can evolve if no progress would be made if the genes did not allow it.

\paragraph{Fitness function}

A \emph{fitness function} must be defined to calculate the fitness level for each individual. This function is used to define the most fit individuals. The higher the fitness level is of a given individual, the higher the chance it has to be chosen to reproduce with another individual.

\subsection{Neural network}

The artificial \emph{neural network} is a computational model inspired by the brain. Neural networks are constructed of connected computational units called \emph{neurons}.

\paragraph{Transfer function}

Each computational unit uses a transfer function to produce an output. In this article we use the sigmoid function:

\[
    S(t) = \frac{1}{1+e^{-t}}
\]

The sigmoid function produces a value between $0$ and $1$. The output, $y_l$, of the $k$th neuron with $m$ inputs to the given neuron, is calculated as:

\[
    y_l = \varphi\left( \sum_{k=1}^m w_{lk} x_k \right)
\]

where $\varphi$ is the transfer function, $x_k$ is the value of input $k$, and $w_{l p}$ is the weight of the connection between neuron $l$ and $k$.
