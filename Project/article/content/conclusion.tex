\section{Conclusion}\label{sec:conclusion}
We have shown that fitness-based diversity measurements do not always catch the difference in traits, since two different individuals can have the same fitness, yet have different traits. This is also the case for genotypic diversity measures, where we have shown how two individuals of different genotypes can have the exact same traits.

We have performed and compared a number of experiments to test our own diversity measure, which we call \di{} (\dia{}). For the diversity measures Hamming distance, fitness-based, and \dia{}, we performed experiments using each of the four replacement rules: the Greedy, Ancestor Elitism, Single Parent Elitism, and Mass Extinction Explore Exploit (MEEE) replacement rule. These experiments indicate that fitness-based diversity measures are unpredictable.

When it comes to \dia{}, which is a phenotypic measure, and Hamming distance, which is a genotypic measure, we observe something interesting. For all the dynamic experiments, we see that the slopes of each replacement rule seem similar. In real world applications, like those investigated in the dynamic experiments, there might be some connection between Hamming distance and \dia{}.

From experiments performed in a static environment, we saw that even a slight mutation among chromosomes of identical individuals caused a major peak in \dia{}, but only a small change in Hamming distance. We have shown that even a small change in an individual's genotype can cause a significant change in its traits. It is also important to be aware of the fact that the Hamming distance measure never exceeded a diversity of \num{0.5} during our experiments, possibly due to the fact that two random bits have the probability \num{0.5} of being identical. A Hamming distance above \num{0.5} is indeed possible, but neither during static nor dynamic experiments, did this happen. 

Based on our experiments and observations, we can conclude that it is reasonable to believe that we have succeeded in creating a diversity measure that better reflects different traits among individuals of a population.

%It is interesting, that for each of the three problems under study, the MEEE replacement rule produced the best fitness values. Furthermore, it was the only replacement rule for which it seems that performing more iterations would yield even greater fitness values. %The MEEE replacement rule actively uses a diversity measure. Therefore, it is important to have a diversity measure that can capture the concept of a population being diverse.
One disadvantage of the species classification approach of \dia{} is that it defines species based on the output neurons yielding the highest output value. So, neural networks with only a single output neuron will always belong to the same species.

The complexity of the fitness-based diversity measure is indeed less than that of \dia{} and Hamming distance, yet also yielding the most stochastic results. Comparing the complexity of \dia{} and Hamming distance depends on the population size and how many random inputs one chooses to use for \dia{}. If the number of random inputs is just a constant times the population size, the two diversity measures asymptotically have the same complexity.
