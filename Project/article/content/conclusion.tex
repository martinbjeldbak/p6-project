\section{Conclusion}\label{sec:conclusion}

We have performed experiments to test our own diversity measure, which we call \di{} (\dia{}). We experimented and compared the results of \dia{} against two other widely used diversity measures, the Hamming distance and the fitness-based diversity measure. These experiments are discussed in \cref{sec:experiments}. We did the testing with four different replacement rules, the Greedy replacement rule, Ancestor Elitism, Single Parent Elitism, and Explore-Exploit. These replacement rules are explained in \cref{sec:replacementrules}.

We argue that fitness-based diversity measurements are much less useful for catching behavioural differences, since two different individuals can have the same fitness, and yet very different behaviour. Our experiments confirm that fitness-based diversity measures are unpredictable.

When it comes to \dia{}, which is a phenotypic measure, and the Hamming distance, which is a genotypic measure, we see something interesting. It seems like these two measures follow each other in a manner. This only holds for tests performed in a dynamic environment though. 
In real world applications, like those investigated in the dynamic tests, there might be some constant factor between Hamming distance and \dia{}.

From tests performed in a static environment, we saw that even a slight mutation among the chromosomes of identical individuals caused a major peak in \dia{}, but only a small change in Hamming distance.
We think it makes sense that even a small change in an individual's genotype can cause a significant change in its behaviour. Consider for instance a single bit that changes. This bit could be a sign bit that causes a significant change to the neural network's behaviour, or it could cause no change at all. The Hamming distance neglect this fact and focuses only on how the genotype is represented by a bit string.

It is also important to be aware of the fact that the Hamming distance measure never exceeded a diversity of \num{0.5} during our tests, which is obviously due to the fact that two random bits have the probability \num{0.5} of being identical.
A hamming distance above \num{0.5} is indeed possible, but neither during static nor dynamic tests, did this happen. 

Based on our experiments and observations, we can conclude that it is reasonable to believe that we have succeeded in creating a diversity measure that better reflects different traits among individuals of a population.

Furthermore, it is clear to see that the replacement rule with the best outcome of fitness values, is the Explore-Exploit, which actively uses a diversity measure. Fore those type of replacement rules, it is important to have a diversity measure that can capture the concept of a population being diverse. We have found that \dia{} seems to better catch different behaviour of the individuals.