\begin{figure*}
  \centering
  \begin{subfigure}[b]{0.33\textwidth}
    \correctlyresize{\linewidth}{%
      \begin{tikzpicture}
        \begin{axis}[
            initial-mut-root,
            legend to name=initMutMeasures,
            title=Leaf,
          ]
          \addplot[mark=*, color=maroon] % Fitness-based
          table[y index=1, x index=0] {data/initial_mutation_leaf.csv};
          \addplot[mark=square*, color=navy] % Hamming
          table[y index=2, x index=0] {data/initial_mutation_leaf.csv};
          \addplot[mark=triangle*, color=blue] % NNTD
          table[y index=3, x index=0] {data/initial_mutation_leaf.csv};
        \end{axis}
      \end{tikzpicture}
    }
  \end{subfigure}%
  ~
  \begin{subfigure}[b]{0.33\textwidth}
    \correctlyresize{\linewidth}{%
      \begin{tikzpicture}
        \begin{axis}[
            initial-mut, 
            title  = Snake,
          ]
          \addplot[mark=*, color=maroon]
          table[y index=1, x index=0] {data/initial_mutation_snake.csv};
          \addplot[mark=square*, color=navy]
          table[y index=2, x index=0] {data/initial_mutation_snake.csv};
          \addplot[mark=triangle*, color=blue]
          table[y index=3, x index=0] {data/initial_mutation_snake.csv};
        \end{axis}
      \end{tikzpicture}
    }
  \end{subfigure}%
  ~
  \begin{subfigure}[b]{0.33\textwidth}
    \correctlyresize{\linewidth}{%
      \begin{tikzpicture}
        \begin{axis}[
            initial-mut, 
            title  = XOR,
          ]
          \addplot[mark=*, color=maroon]
          table[y index=1, x index=0] {data/initial_mutation_xor.csv};
          \addplot[mark=square*, color=navy]
          table[y index=2, x index=0] {data/initial_mutation_xor.csv};
          \addplot[mark=triangle*, color=blue]
          table[y index=3, x index=0] {data/initial_mutation_xor.csv};
        \end{axis}
      \end{tikzpicture}
    }
  \end{subfigure}
  \ref{initMutMeasures}
  \caption{Average over \num{100} runs for each diversity measure on data sets over intervals of initial mutation.}\label{fig:initial-mutation}
\end{figure*}
