\newcommand{\perc}[1]{\SI{#1}{\percent}}
\newcommand{\di}{Neural Network Trait Diversity}
\newcommand{\dia}{NNTD}
\newcommand{\fit}{\phi}

\newcommand{\mail}[1]{\href{mailto:#1}{#1}}
\newcommand{\var}[1]{\ensuremath{\mathit{#1}}}
\newcommand{\citpls}[1]{\footnote{\textcolor{red}{Citation needed here.}}}
\newcommand{\set}[1]{\ensuremath{\left\{ #1 \right\}}}
\newcommand{\setof}[2]{\ensuremath{\left\{ #1 \mid #2 \right\}}}
\newcommand{\bigO}[1]{\ensuremath{\operatorname{O}\left(#1\right)}}

\newcommand{\correctlyresize}[2]{\resizebox{#1}{!}{#2}}
\newcommand{\inputresize}[1]{\correctlyresize{\linewidth}{\input{#1}}}
\newcommand{\inputresizeto}[2]{\correctlyresize{#1}{\input{#2}}}

\newcommand{\sizeof}[1]{\lvert#1\rvert}

\newcommand{\ind}{\ensuremath{f}} % an individual
\newcommand{\indset}{\ensuremath{F}} % set of neural networks (population)
\newcommand{\indsetl}{\ensuremath{\sizeof{\indset}}}
\newcommand{\ran}{\ensuremath{r}} % a random input
\newcommand{\ranset}{\ensuremath{R}} % set of random inputs 
\newcommand{\ransetl}{\ensuremath{\sizeof{\ranset}}}
\newcommand{\nnoutset}{\ensuremath{O}}
\newcommand{\nnoutsetl}{\ensuremath{\sizeof{\nnoutset}}}
\newcommand{\nnout}{\ensuremath{o}}
\newcommand{\nnin}{\ensuremath{i}}
\newcommand{\nninset}{\ensuremath{I}}
\newcommand{\nninsetl}{\ensuremath{\sizeof{\nninset}}}
\newcommand{\timet}{\ensuremath{t}}

\definecolor{maroon}{HTML}{85144B}
\definecolor{navy}{HTML}{001F3F}
\definecolor{blue}{HTML}{0074D9}
\definecolor{aqua}{HTML}{7FDBFF}
\definecolor{teal}{HTML}{39CCCC} 
\definecolor{red}{HTML}{FF4136}
\definecolor{blue}{HTML}{0074D9}
\definecolor{black}{HTML}{111111}
\definecolor{purple}{HTML}{B10DC9}
\definecolor{green}{HTML}{2ECC40}

\sisetup{%
  locale=UK,
  zero-decimal-to-integer % do not show .0 on whole numbers
}

\pgfplotsset{%
  compat=newest,
    %grid=major,
    %grid style={dashed,gray!30},
    %scale only axis,
  table/col sep=comma,
  width=\linewidth,
  mark end/.style={%
    % From: https://tex.stackexchange.com/questions/116690/pgfplots-marks-mandatory-for-1st-and-last-point
    scatter,
    scatter src=x,
    scatter/@pre marker code/.code={%
      \pgfmathtruncatemacro\usemark{%
        (\coordindex==(1))
      }
      \ifnum\usemark=0
      \pgfplotsset{mark=none}
      \fi
    },
    scatter/@post marker code/.code={}
  },
  x-tick-siunitx/.style={%
    xticklabel = \pgfmathparse{\tick*1}\num{\pgfmathresult},
  },
  y-tick-siunitx/.style={%
    yticklabel = \pgfmathparse{\tick*1}\num{\pgfmathresult},
  },
  y-div/.style={%
    x-tick-siunitx,
    y-tick-siunitx,
    ylabel = Diversity,
  },
  initial-mut/.style={%
    y-div,
    xlabel = {Initial Mutation},
  },
  initial-sim/.style={%
    y-div,
    xlabel = {Initial Similarity},
  },
  initial-mut-sim-root/.style={%
    legend columns=-1,
    legend entries={Fitness-based, Hammming distance, \dia{}},
  },
  initial-sim-root/.style={%
    initial-mut-sim-root,
  },
  initial-mut-root/.style={%
    initial-mut-sim-root,
  },
  dynamic-root/.style={%
    legend columns=-1,
    legend entries={Greedy, Ancestor Elitism, Single Parent Elitism, Explore-Exploit},
  },
  dynamic/.style={%
    mark end,
    y-div,
    xlabel = Generation,
    each nth point=5,
  },
  fitness/.style={%
    mark end,
    x-tick-siunitx,
    y-tick-siunitx,
    ylabel = Fitness,
    xlabel = Generation,
    title  = Fitness,
    each nth point=5,
  },
}
