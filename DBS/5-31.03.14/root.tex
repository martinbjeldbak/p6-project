\section{Self Study 5: Mini-project part 4}
This self study was done on \formatdatewday{31}{3}{2014}. We first converted the IMDB MySQL dump to PosgresSQL and then mapped all the tables to our schema, before finally designing the queries described in the exercise document. The PostgreSQL dump of our converted schema can be found at this link: \url{https://bitbucket.org/Obeyed/p6-project/src/72274ff3487ef59dfd2782cb3a277775b179654c/DBS/5-31.03.14/ourimdb.sql?at=master} with filename \texttt{ourimdb.sql}.

\subsection{Filling the database}
We use a Ruby gem (\url{https://github.com/maxlapshin/mysql2postgres}) to convert the MySQL dump of the IMDB database found on Moodle to a PostgreSQL database with the same tables and data. To ease transfer, we added the tables defined in \cref{fig:ss2-schema} of self study 2 in the same database and executed the \texttt{INSERT INTO} queries below to map the data from IMDB to our tables.

\begin{lstlisting}[language=SQL]
INSERT INTO our_movie
SELECT title, language, year
FROM movie
\end{lstlisting}

\begin{lstlisting}[language=SQL]
INSERT INTO our_genre
SELECT DISTINCT genre
FROM genre
\end{lstlisting}

\begin{lstlisting}[language=SQL]
INSERT INTO "our_genreMovie"
SELECT genre.genre, movie.title, movie.year
FROM genre, movie
WHERE genre."movieId" = movie.id
\end{lstlisting}

\begin{lstlisting}[language=SQL]
INSERT INTO our_refers
SELECT m1.title, m2.title, m1.year, m2.year
FROM movieref, movie m1, movie m2
WHERE movieref."fromId" = m1.id AND movieref."toId" = m2.id
\end{lstlisting}

\begin{lstlisting}[language=SQL]
INSERT INTO our_person
SELECT id, birthdate, name, '', deathdate, gender
FROM person
\end{lstlisting}

\begin{lstlisting}[language=SQL]
INSERT INTO our_role
SELECT DISTINCT role
FROM involved
\end{lstlisting}

\begin{lstlisting}[language=SQL]
INSERT INTO our_participates
SELECT movie.title, involved.role, involved."personId", movie.year
FROM involved, movie
WHERE involved."movieId" = movie.id
\end{lstlisting}

\begin{lstlisting}[language=SQL]
INSERT INTO our_participates
SELECT movie.title, involved.role, involved."personId", movie.year
FROM involved, movie
WHERE involved."movieId" = movie.id
\end{lstlisting}

\begin{lstlisting}[language=SQL]
INSERT INTO our_user
SELECT DISTINCT "user", null, null
FROM ratings
\end{lstlisting}

\begin{lstlisting}[language=SQL]
INSERT INTO our_review
SELECT rating, "user", movie.title, null, movie.year
FROM ratings, movie
WHERE movie.id = ratings."movieId"
\end{lstlisting}

No data was present in the dump to identify awards given to movies.

Sometimes we have decided to drop the “date” attribute, rename it to ``year'' and change its type to int, causing the order of attributes to be rearranged.

There was no information about gender in the database, making it impossible to do any queries based on gender.

Upon mapping of the ‘ratings’ table in the IMDB database, we noticed out that our structure (as shown in figure 3) can uniquely identifiy ratings (reviews) merely by the user’s primary key: username, and the movie’s primary keys: title and date. This restricts the user in only being able to have one review per movie, which is what we wish and allows us to remove the ``id'' column.

\subsection{Querying}

\subsubsection{1.\ How many Danish language movies are in the database?}
Query can be seen below and result is illustrated in \cref{tab:5q1}.
\begin{lstlisting}[language=SQL]
SELECT COUNT(*) FROM movie WHERE language = 'Danish'
\end{lstlisting}

\begin{table}
  \centering
\begin{tabular}[htpb]{S}
  \toprule
  {Count} \\
  \midrule
  419 \\
  \bottomrule
\end{tabular}
\caption{Results of query 1}\label{tab:5q1}
\end{table}

\subsubsection{2.\ For each year, what is the total number of reviews to movies from that year?}
Query can be seen below and result is illustrated in \cref{tab:5q2}.

\begin{lstlisting}[language=SQL]
SELECT movie.year, COUNT(movie.year) FROM review, movie
WHERE review.title = movie.title AND review.year = movie.year
GROUP BY movie.year
\end{lstlisting}

\begin{table}
  \centering
  \begin{tabular}[htpb]{S S}
    \toprule
    {movie.year} & {COUNT(movie.year)} \\
    \midrule
    2010 & 23 \\
    2000 & 24 \\
    2006 & 27 \\
    1962 & 1 \\
    2003 & 27 \\
    2012 & 17 \\
    2002 & 19 \\
    1993 & 4 \\
    2001 & 28 \\
    1999 & 48 \\
    \bottomrule
\end{tabular}
\caption{Results of query 2}\label{tab:5q2}
\end{table}

\subsubsection{3.\ Which movies have John Travolta and Uma Thurman starred in together?}
Query can be seen below and result is illustrated in \cref{tab:5q3}.
\begin{lstlisting}[language=SQL]
SELECT m.title, m.year
FROM movie m, person pe1, person pe2, participates pa1,
     participates pa2
WHERE pa1.title = m.title AND pa1.year = m.year AND
      pa2.title = m.title AND pa2.year = m.year AND
      pe1.id = pa1.id AND pe2.id = pa2.id AND pe1.id != pe2.id AND
      pe1.name = 'John Travolta' AND
      pe2.name = 'Uma Thurman' AND pa1.rolename = 'actor'
      AND pa2.rolename = 'actor'
\end{lstlisting}

\begin{table}
  \centering
  \begin{tabular}[htpb]{l S}
    \toprule
    m.title & {m.year} \\
    \midrule
    Entertainment Tonight & 1981 \\
    Late Show with David Letterman & 1993 \\
    The Tonight Show with Jay Leno & 1992 \\
    E! True Hollywood Story & 1996 \\
    Live with Regis and Kathie Lee & 1988 \\
    The Charlie Rose Show & 1991 \\
    The Rosie O'Donnell Show &1996 \\
    Ellen: The Ellen DeGeneres Show & 2003 \\
    Be Cool & 2005 \\
    Pulp Fiction & 1994 \\
    \bottomrule
\end{tabular}
\caption{Results of query 3}\label{tab:5q3}
\end{table}

\subsubsection{4.\ How many actors and directors have a first name starting with ``Q``?}
Query can be seen below and result is illustrated in \cref{tab:5q4}.

\begin{lstlisting}[language=SQL]
SELECT pa.rolename, COUNT(*) FROM person pe, participates pa 
WHERE pe.id = pa.id AND pa.rolename IN ('actor', 'director') AND
      pe.name LIKE 'Q%'
GROUP BY pa.rolename
\end{lstlisting}

\begin{table}
  \centering
  \begin{tabular}[htpb]{l S}
    \toprule
    pa.rolename & {COUNT(*)} \\
    \midrule
    "actor & 2100 \\
    "director & 45 \\
    \bottomrule
\end{tabular}
\caption{Results of query 4}\label{tab:5q4}
\end{table}

\subsubsection{5.\ How many users rated at least 3 movies?}
Query can be seen below and result is illustrated in \cref{tab:5q5}.

\begin{lstlisting}[language=SQL]
SELECT COUNT(*) FROM 
(SELECT r1.username FROM review r1, review r2, review r3
WHERE r1.username = r2.username AND r1.username = r3.username
    AND r1.title != r2.title AND r2.title != r3.title
    AND r1.title != r3.title
GROUP BY r1.username) AS "total"
\end{lstlisting}

\begin{table}
  \centering
  \begin{tabular}[htpb]{S}
    \toprule
    {COUNT(*)} \\
    \midrule
    34 \\
    \bottomrule
  \end{tabular}
  \caption{Results of query 5}\label{tab:5q5}
\end{table}

\subsubsection{6.\ What is the name and birth year of all actors in “Pulp Fiction”? List the actors in increasing order of birth year.}
Query can be seen below and result is illustrated in \cref{tab:5q6}.

\begin{lstlisting}[language=SQL]
SELECT p.name, p.birthdate FROM participates pp,
       person p WHERE pp.title = 'Pulp Fiction' AND pp.id = p.id
       AND pp.rolename = 'actor' ORDER BY p.birthdate ASC
\end{lstlisting}

\begin{table}
  \centering
  \begin{tabular}[htpb]{l l}
    \toprule
    p.name & p.birthdate \\
    \midrule
    Emil Sitka & 1914-12-22 \\
    Harvey Keitel & 1939-05-13 \\
    Rene Beard & 1941-06-03 \\
    Christopher Walken & 1943-03-31 \\
    Joseph Pilato & 1949-03-16 \\
    Brenda Hillhouse & 1953-12-11 \\
    John Travolta & 1954-02-18 \\
    Bruce Willis & 1955-03-19 \\
    Amanda Plummer & 1957-03-23 \\
    Lawrence Bender & 1957-10-17 \\
    \bottomrule
  \end{tabular}
  \caption{Results of query 6}\label{tab:5q6}
\end{table}

\subsubsection{7.\ What are the titles and years of all movies from the 1980s that John Travolta starred in?}
Query can be seen below and result is illustrated in \cref{tab:5q7}.

\begin{lstlisting}[language=SQL]
SELECT pa.title, pa.year FROM participates pa, person p 
WHERE pa.year >= 1980 AND pa.year < 1990
      AND p.name = 'John Travolta' AND pa.id = p.id
      AND pa.rolename = 'actor'
\end{lstlisting}

\begin{table}
  \centering
  \begin{tabular}[htpb]{l S}
    \toprule
    pa.title & {pa.year} \\
    \midrule
    Entertainment Tonight & 1981 \\
    Live with Regis and Kathie Lee & 1988 \\
    Biography & 1987 \\
    Look Who's Talking & 1989 \\
    Wetten, dass\dots? & 1981 \\
    Larry King Live & 1985 \\
    Two of a Kind & 1983 \\
    Staying Alive & 1983 \\
    That's Dancing! & 1985 \\
    Urban Cowboy & 1980 \\
    \bottomrule
  \end{tabular}
  \caption{Results of query 7}\label{tab:5q7}
\end{table}
\subsubsection{8.\ What are the top-2 highest rated movies (average) from the 1990s according to the users?}
Query can be seen below and result is illustrated in \cref{tab:5q8}.

\begin{lstlisting}[language=SQL]
SELECT title, year, AVG(rating) AS avgrate
FROM review WHERE year >= 1990 AND year < 2000 GROUP BY title, year
ORDER BY avgrate DESC LIMIT 2
\end{lstlisting}

\begin{table}
  \centering
  \begin{tabular}[htpb]{l S S}
    \toprule
    title & {year} & {avgrate} \\
    \midrule
    The Usual Suspects & 1995 & 9.3333333333333333 \\
    The Shawshank Redemption & 1994 & 9.1250000000000000 \\
    \bottomrule
  \end{tabular}
  \caption{Results of query 8}\label{tab:5q8}
\end{table}

\subsubsection{9.\ What are the top-2 highest rated movies (average) from the 1990s according to at least 2 users?}
Query can be seen below and result is illustrated in \cref{tab:5q9}.

\begin{lstlisting}[language=SQL]
SELECT r1.title, r1.year, AVG(r1.rating) AS avgrate
FROM review r1, review r2 WHERE r1.year >= 1990 AND
     r1.year < 2000 AND r1.username != r2.username
GROUP BY r1.title, r1.year
ORDER BY avgrate DESC LIMIT 2
\end{lstlisting}

\begin{table}
  \centering
  \begin{tabular}[htpb]{l S S}
    \toprule
    title & {year} & {avgrate} \\
    \midrule
    The Usual Suspects & 1995 & 9.3489392831016825 \\
    The Shawshank Redemption & 1994 & 9.1187933796049119 \\
    \bottomrule
  \end{tabular}
  \caption{Results of query 9}\label{tab:5q9}
\end{table}

\subsubsection{10.\ In 1994, what was the average rating of a movie for each language?}
Query can be seen below and result is illustrated in \cref{tab:5q10}.

\begin{lstlisting}[language=SQL]
SELECT AVG(r.rating), m.language 
FROM movie m, review r 
WHERE m.title = r.title AND m.year = r.year AND m.year = 1994
GROUP BY m.language
\end{lstlisting}

\begin{table}
  \centering
  \begin{tabular}[htpb]{l S}
    \toprule
    {AVG(r.rating)} & m.language \\
    \midrule
    7.0000000000000000 & "" \\
    9.0000000000000000 & French \\
    8.3043478260869565 & English \\
    \bottomrule
  \end{tabular}
  \caption{Results of query 10}\label{tab:5q10}
\end{table}

\subsubsection{11.\ Which actors in Pulp Fiction have never, before or after, starred in the same movie as one of the other actors in “Pulp Fiction”?}
We did not complete this query, because it required a massive, advanced query.

\subsubsection{12.\ Which movie starring John Travolta has the highest user ratings?}
Query can be seen below and result is illustrated in \cref{tab:5q12}.

\begin{lstlisting}[language=SQL]
SELECT r.title, r.year FROM review r, person p, participates pa 
WHERE p.name = 'John Travolta' AND p.id = pa.id AND
      r.title = pa.title AND r.year = pa.year
GROUP BY r.title, r.year
ORDER BY AVG (r.rating) DESC
LIMIT 1
\end{lstlisting}

\begin{table}
  \centering
  \begin{tabular}[htpb]{l S}
    \toprule
    r.title & {r.year} \\
    \midrule
    Pulp Fiction & 1994 \\
    \bottomrule
  \end{tabular}
  \caption{Results of query 12}\label{tab:5q12}
\end{table}

\subsubsection{13.\ How many actresses have not been alive at the same time as Charles Chaplin?}
Query can be seen below and result is illustrated in \cref{tab:5q13}.

\begin{lstlisting}[language=SQL]
SELECT COUNT(*)
FROM person p1 JOIN person p2
ON (p1.birthdate > p2.dateofdeath OR
    p1.dateofdeath < p2.birthdate) AND
    p2.name = 'Charles Chaplin'
\end{lstlisting}
%
\begin{table}
  \centering
  \begin{tabular}[htpb]{S}
    \toprule
    {COUNT(*)} \\
    \midrule
    11707 \\
    \bottomrule
  \end{tabular}
  \caption{Results of query 13}\label{tab:5q13}
\end{table}

\subsubsection{14.\ What is the average rating of movies from each genre?}
Query can be seen below and result is illustrated in \cref{tab:5q14}.

\begin{lstlisting}[language=SQL]
SELECT gm.name, AVG(r.rating)
FROM review r, genremovie gm 
WHERE gm.title = r.title AND gm.year = r.year
GROUP BY gm.name
\end{lstlisting}
%
\begin{table}
  \centering
  \begin{tabular}[htpb]{l S}
    \toprule
    gm.name & {AVG(r.rating)} \\
    \midrule
    Drama & 7.8423236514522822 \\
    Comedy & 7.2380952380952381 \\
    Fantasy & 7.2685185185185185 \\
    Biography & 8.1333333333333333 \\
    Thriller & 7.7543859649122807 \\
    Crime & 8.3949579831932773 \\
    War & 8.2692307692307692 \\
    Musical & 7.0000000000000000 \\
    History & 8.1250000000000000 \\
    Adventure & 7.4479166666666667 \\
    \bottomrule
  \end{tabular}
  \caption{Results of query 14}\label{tab:5q14}
\end{table}

\subsubsection{15.\ What is the average rating of movies from each genre? List only genres with at least 2 ratings.}
Query can be seen below and result is illustrated in \cref{tab:5q15}.
%
\begin{lstlisting}[language=SQL]
SELECT name, AVG(review.rating)
FROM genremovie, review
WHERE genremovie.title = review.title AND genremovie.year = review.year
GROUP BY name HAVING COUNT(*) >= 2
\end{lstlisting}
%
\begin{table}
  \centering
  \begin{tabular}[htpb]{l S}
    \toprule
    name & {AVG(review.rating)} \\
    \midrule
    Drama & 7.8423236514522822 \\
    Comedy & 7.2380952380952381 \\
    Fantasy & 7.2685185185185185 \\
    Biography & 8.1333333333333333 \\
    Thriller & 7.7543859649122807 \\
    Crime & 8.3949579831932773 \\
    War & 8.2692307692307692 \\
    History & 8.1250000000000000 \\
    Adventure & 7.4479166666666667 \\
    Sci-Fi & 7.4710144927536232 \\
    \bottomrule
  \end{tabular}
  \caption{Results of query 15}\label{tab:5q15}
\end{table}

\subsubsection{16.\ Which movie has the largest number of 2-link references?}
If A refers to B, and B refers to C, then we say that A has a 2-link reference, through B, to C. If there are several paths leasing from A to C, we count all of them. Query can be seen below and result is illustrated in \cref{tab:5q16}.
%
\begin{lstlisting}[language=SQL]
SELECT r1.title, COUNT(*) AS ref
FROM refers r1, refers r2
WHERE r1.referredtitle = r2.title AND r1.referredyear = r2.year
GROUP BY r1.title ORDER BY ref DESC LIMIT 1
\end{lstlisting}
%
\begin{table}
  \centering
  \begin{tabular}[htpb]{l S}
    \toprule
    r1.title & {ref} \\
    \midrule
    Saturday Night Live & 38834 \\
    \bottomrule
  \end{tabular}
  \caption{Results of query 16}\label{tab:5q16}
\end{table}

\subsubsection{17.\ How many actors have also been active as director of at least one movie?}
Query can be seen below and result is illustrated in \cref{tab:5q17}.
%
\begin{lstlisting}[language=SQL]
SELECT COUNT(*)
FROM (SELECT DISTINCT p1.id
      FROM participates p1, participates p2
      WHERE p1.id = p2.id AND p1.rolename = ‘actor’ AND
            p2.rolename = ‘director’) AS director_actors
\end{lstlisting}
%
\begin{table}
  \centering
  \begin{tabular}[htpb]{S}
    \toprule
    {COUNT(*)} \\
    \midrule
    5930 \\
    \bottomrule
  \end{tabular}
  \caption{Results of query 17}\label{tab:5q17}
\end{table}

\subsubsection{18.\ Which two genres are most often linked to the same movie?}
Query can be seen below and result is illustrated in \cref{tab:5q18}.
%
\begin{lstlisting}[language=SQL]
SELECT g1.name, g2.name, COUNT(*) as c
FROM genremovie g1, genremovie g2
WHERE g1.name != g2.name AND g1.title = g2.title AND
      g1.year = g2.year GROUP BY g1.name, g2.name
ORDER BY c DESC
LIMIT 1
\end{lstlisting}
%
\begin{table}
  \centering
  \begin{tabular}{l l S}
    \toprule
    g1.name & g2.name & {COUNT(*)} \\
    \midrule
    Romance & Drama & 5837 \\
    \bottomrule
  \end{tabular}
  \caption{Results of query 18.}\label{tab:5q18}
\end{table}
