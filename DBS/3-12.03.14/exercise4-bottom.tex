Relational Algebra:

\begin{align*}
  \pi_{\text{ c1.sid, c2.sid }}((\rho_{\text{ c1 }}(\mathtt{Catalog}) \times \rho_{\text{ c2 }}(\mathtt{Catalog})) &\bowtie_{\text{ c1.pid } = \text{ p1.pid }} \wedge \\ 
  \text{ c2.pid } = \text{ p2.pid }(\rho_{\text{ p1 }}(\mathtt{Parts}) &\bowtie_{\text{ p1.cost } > \text{ p2.cost }} \rho_{\text{ p2 }}(\mathtt{Parts})))
\end{align*}



Domain Relational Calculus:

\begin{align*}
  \{\langle \text{ sid, sid' } \rangle 
    \mid \exists \text{ pid, cost} (\langle \text{ sid, pid, cost } \rangle 
      &\in \mathtt{Catalog} \; \wedge \\
    \exists \text{ pid, cost } (\lag \text{ sid', pid, cost' } \rag
      &\in \mathtt{Catalog} \; \wedge \\
  (\text{ cost } > \text{ cost' }) &\wedge (\text{ sid } \neq \text{ sid' })))\}
\end{align*}



\subsubsection{5}

Find the pids of parts supplied by at least two different suppliers:

Relational Algebra:

\begin{align*}
  \pi_{\var{c1.pid}}(\sigma_{\var{c1.sid}=\var{c2.sid}}(\rho_{\var{c2}}(\mathtt{Catalog}) \bowtie_{\var{c1.pid} = \var{c2.pid}} \rho_{\var{c1}}(\mathtt{Catalog})))
\end{align*}

Tuple Relational Calculus:

\begin{align*}
  \setof{\var{c1.pid}}{\var{c1} \in \mathtt{Catalog} &\wedge \var{c2} \in \mathtt{Catalog} (\\
(\var{c1.pid} = \var{c2.pid}) &\wedge (\var{c1.sid} \neq \var{c2.sid}))}
\end{align*}

Domain Relational Calculus:

\begin{align*}
  \{\langle \text{ pid } \rangle 
    \mid \exists \text{ sid}, \text{ cost } 
    (\langle \text{ sid}, \text{ pid}, \text{ cost} \rangle 
      &\in \mathtt{Catalog} \; \wedge \\
       \exists \text{ sid'}, \text{ cost' } (
        \langle \text{ sid'}, \text{ pid}, \text{ cost' } \rangle 
    &\in \mathtt{Catalog} \; \wedge \text{sid } \neq \text{ sid'}))\}
\end{align*}


