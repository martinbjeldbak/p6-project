\subsection{Exercise 3: Relational Algebra}

In the first relational algebra expression we begin by selecting all entries in the \texttt{zoos} table, where $\mathtt{country} = \mathtt{'Germany'}$.
We then project only the zooId from that result.
Before we do the division, we project the species and zooId of the \texttt{animals} table.
Finally, we divide the two tables and end up with:

\begin{center}
  \begin{tabular}{ c }
    \hline
    species \\  
    \hline
    giraffe \\
    ape \\
    owl \\
    \hline
  \end{tabular}
\end{center}

We start by renaming the \texttt{animals} table to two tables called \texttt{T1} and \texttt{T2}.
We then do a theta-join on the two new tables, with $\mathtt{T1.zooId} = \mathtt{T2.zooId}$ as a constraint.
Now we have a large table with \texttt{T1} and \texttt{T2} side by side, where the constraint is maintained.
We now do a selection from the joined table with $\mathtt{T1.animalId} = \mathtt{R2.father} \cup \mathtt{T1.animalId} = \mathtt{T2.mother}$.
This results in a table with rows where the child and at least one of the parents are in the same zoo.
Finally, we do a projection of the nickname of \texttt{T1}, and get the following result:

\begin{center}
  \begin{tabular}{ c }
    \hline
    nickname \\  
    \hline
    Uhu \\
    Jahoo \\
    Boo \\
    \hline
  \end{tabular}
\end{center}

