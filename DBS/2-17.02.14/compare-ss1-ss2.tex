\subsection{Comparison of the previous and current solution}

In our first attempt to construct a diagram for the movie database, we used the Enhanced Entity-Relationship (EER) model to construct the relevant information for the database.
In this version of our database, we use the Entity-Relationship (ER) model as described in the course.

In this version we include Chen notation and min-max notation to emphasize to type of relations. 
This is also visualised in form of arrows or no arrows on each connection between relations.
Another clear difference is that we include total participation for some of the relations.
Actually, total and partial participation is expressed as identifying and non-identifying relations in the EER model. This is not covered in the course.
We could also have included weak entities, but we did not find any which should be marked as weak. 

We have removed redundancy, because an actor can also be a director in movies.
We have introduced a relation called Role in which it is clear which role a person has in a given movie.
We also considered an ISA relation between the roles in a movie.
We chose not to use it, because there is nothing different between an actor and a director.

We have only included the necessary primary keys. 
If there was no need for a unique id, we have not included one.
